\title{Syllabus for Algebra-Based Physics-1: Mechanics (PHYS135A-01)}
\author{Dr. Jordan Hanson - Whittier College Dept. of Physics and Astronomy}
\date{\today}
\documentclass[10pt]{article}
\usepackage[a4paper, total={18cm, 27cm}]{geometry}
\usepackage{outlines}
\usepackage[sfdefault]{FiraSans}
\usepackage{hyperref}

\begin{document}
\maketitle

\begin{abstract}
The concepts of algebra-based mechanics will be presented within the context of interactive problem-solving.  First, the concepts of displacement, velocity, and acceleration in one and two dimensions will be introduced, building up to Newton's Laws of motion.  Next, the concepts of friction and rotational motion will be added.  More complex problems will be introduced through the conservation of energy and linear momentum, followed by the rotational equivalents.  The course work will include interactive computational exercises, analytic textbook problems, and lab-based activities.
\end{abstract}
\noindent
\textit{\textbf{Pre-requisites}: None.} \\
\textit{\textbf{Course credits, Liberal Arts Categorization}: 4 Credits, None} \\
\textit{\textbf{Regular course hours}: Monday and Wednesday from 8:50 - 10:50 in SLC 228} \\
\textit{\textbf{Instructor contact information}: jhanson2@whittier.edu, tel. 562.907.5130} \\
\textit{\textbf{Office hours}: Tuesdays at 12:50 and Wednesdays at 15:00 in SLC 212} \\
\textit{\textbf{Attendance/Absence}: In-class activities will serve as attendance (see \textit{\textbf{Grading}}}).  Students needing to reschedule midterms and exams should notify the professor a reasonable time beforehand. \\
\textit{\textbf{Late work policy}: Late work will not be accepted.} \\
\textit{\textbf{Text}: College Physics (openstax.org) -  https://openstax.org/details/books/college-physics} \\
\textit{\textbf{Grading}: There will be four 45-minute quizzes, each examining conceptual understanding in step-by-step problems.  Each midterm is worth 10\% of the final grade.  The weekly homework is worth 20\% of the grade.  Interactive in-class activities will be worth 15\% of the final grade.  Lab groups will present results of a group project worth 10\% of the grade.  The final exam will be held on December 13th for a hald-class period, and will be worth 15\% of the grade.} \\
\textit{\textbf{Homework sets}: Typically 5-8 problems per week, assigned on Monday and collected the following Monday.} \\
\textit{\textbf{ADA Statement on Disability Services}: The Americans with Disabilities Act (ADA) is a federal anti-discrimination statute that provides comprehensive civil rights protection for persons with disabilities. Among other things, this legislation requires that all students with disabilities be guaranteed a learning environment that provides for reasonable accommodation of their disabilities. If you believe you have a disability requiring an accommodation, please contact Disability Services: disabilityservices@whittier.edu, tel. 562.907.4825.} \\
\textit{\textbf{Academic Honesty Policy}: \url{http://www.whittier.edu/academics/academichonesty}} \\
\textit{\textbf{Course Objectives}:}
\begin{itemize}
\item Written expression of quantitative and numerical ideas and arguments.
\item Oral expression of quantitative and numerical ideas and arguments.
\item Problem solving using numerical skills.
\item Mathematical modeling.
\item Logical thinking.
\item Analysis of data and results.
\end{itemize}
\clearpage
\textit{\textbf{Course Outline}:}
\begin{outline}[enumerate]
\1 Week 1 - September 6th through September 8th - \textbf{Chapter 1}
\2 Estimations, approximations, unit-analysis, and coordinate systems
\2 Adding and subtracting vectors, displacement and translational motion
\2 \textit{Mathematics review: geometry, trigonometry}
\1 Week 2 - September 11th through September 15th - \textbf{Chapters 2.1-2.6}
\2 Velocity and acceleration
\2 Accelerration due to gravity near the Earth's surface and other constant accelerations
\2 Common equations of motion for constant acceleration
\1 Week 3 - September 18th through September 22nd - \textbf{Chapters 2.7-2.8, Chapters 3.1,3.4-3.5}
\2 Kinematics with vectors
\2 Projectile motion
\2 Addition of velocities, relative motion
\1 First midterm exam, beginning of week 4
\2 Estimations, approximations, and unit-analysis
\2 Displacement, velocity, and constant acceleration
\2 Vectors
\1 Week 4 - September 25th through September 29th - \textbf{Chapters 4.1-4.4}
\2 Newton's Laws
\2 Examples and free-body diagrams
\1 Week 5 - October 2nd through October 6th - \textbf{Chapters 4.4-4.7}
\2 Newton's Laws, continued
\2 Examples of frictional forces, drag, tension
\1 Second midterm exam, beginning of week 6
\2 Newton's laws and free-body diagrams
\2 Friction, drag, and tension
\1 Week 6 - October 9th through October 13th - \textbf{Chapter 6.1-6.3,6.5-6.6}
\2 Angular displacement, angular velocity and centripetal acceleration
\2 Newton's law of gravity, circular orbits
\2 Kepler's laws
\1 Week 7 - October 16th through October 19th - \textbf{Chapters 7.1-7.4}
\2 Work, kinetic and potential energy
\2 Work-Energy theorem
\2 Conservative forces and potential energy
\1 Mid-semester break, October 20th
\1 Week 8 - October 23rd through October 27th - \textbf{Chapters 7.6-7.8}
\2 Conservation of energy
\2 Power
\2 \textit{Work, energy and power in human beings}
\1 Third midterm exam, beginning of week 9
\2 Gravitation and Kepler's Laws
\2 Work-Energy theorem
\2 Conservation of energy
\1 Week 9 - October 30th through November 3rd - \textbf{Chapters 8.1, 8.3-8.5}
\2 Linear momentum
\2 Conservation of momentum and Newton's second law
\2 Elastic scattering
\2 Inelastic scattering
\1 Week 10 - November 6th through November 10th - \textbf{Chapters 8.6-8.7}
\2 Momentum conservation, continued
\2 Two-dimensional scattering
\2 Rockets
\1 Week 11 - November 13th through November 17th - \textbf{Chapters 10.1-10.4}
\2 Rotational displacement, velocity, and acceleration, continued
\2 Moments of inertia and angular momentum
\2 Kinetic energy of angular motion
\1 Forth midterm exam, beginning of week 12
\2 Conservation of momentum
\2 Scattering in one and two-dimensions
\2 Rotational motion
\1 Week 12 - November 20th and November 21st - \textbf{Chapters 10.5}
\2 Conservation of angular momentum
\1 Week 13 - November 27th through December 1st  - \textbf{Chapter 9}
\2 Torque and Newton's second law, angular case
\2 Statics and torque
\1 Week 14 - December 4th through December 8th - \textbf{Class Presentations and Final Review}
\end{outline}
\end{document}
