\title{Midterm 1 for Calculus-Based Physics-1: Mechanics (PHYS150-01)}
\author{Dr. Jordan Hanson - Whittier College Dept. of Physics and Astronomy}
\date{September 25th, 2017}
\documentclass[10pt]{article}
\usepackage[a4paper, total={18cm, 27cm}]{geometry}
\usepackage{outlines}
\usepackage[sfdefault]{FiraSans}

\begin{document}
\maketitle

\section{Estimation, Approximation, and Unit Analysis}
\begin{enumerate}
\item There is a jar full of candies.  Estimate the number of candies in the jar, if this type of round candy has a radius of $\approx 0.3$ cm, and the jar has a radius of $\approx 12$ cm, and a similar height.  (Remember, the answer only has to be accurate to the correct order of magnitude). \vspace{1.5 cm}
\item An explorer lands on Mars, which has an acceleration due to gravity of $\approx 0.4 g$.  If the explorer drops a bag from shoulder height, how long does it take to reach the ground? \vspace{1.5 cm}
\item Our bodies contain special nerve fibers from the spinal cord to extermities, for quick reactions.  Suppose a curious child, who is 0.75 m tall, touches a flame.  The child's body jerks the hand away after 20 milliseconds (0.02 seconds).  What was the speed of the nerve signal? \\
\begin{itemize}
\item A: 1 m
\item B: 5 s
\item C: 20 m/s
\item D: 100 m/s
\end{itemize}
\end{enumerate}
\section{Displacement, Velocity, and Constant Acceleration}
\begin{enumerate}
\item In the film \textit{The Hunt for Red October}, one scene depicts two Soviet officers at the helm of a submarine they are navigating through an undersea canyon.  Their current speed is 30 kilometers per hour, and the canyon turns 45 degrees to their right 1 kilometer ahead.  In how many seconds must they order the ship to turn before crashing into the side of the canyon?  After the turn, they adjust the speed to 20 kilometers per hour, and travel for 100 seconds.  What is the displacement vector from the original position?\vspace{3cm}
\item A torpedo is dropped in the water 1 km behind the \textit{Red October} by an overflying aircraft, and it accelerates at 3 m/s$^2$, with an initial velocity of 5 m/s.  If the \textit{Red October} does not alter course, but continues at 5 m/s, when will the torpedo reach it?\vspace{3cm}
\end{enumerate}
\section{Vectors and Relative Motion}
\begin{enumerate}
\item The great astronomer Edwin Hubble discovered that all distant galaxies are receding from our Milky Way Galaxy with velocities proportional to their distances. It appears to an observer on the Earth that we are at the center of an expanding universe.  Using the data from Table \ref{tab1}, which depicts distances and velocities of galaxies \textit{relative to the Milky Way}, calculate the velocities relative to galaxy 2.  Would an observer in galaxy 2 also conclude that the universe is expanding (all galaxies moving away from it)?
\begin{table}
\begin{tabular}{| c | c | c |}
\hline
Galaxy & Displacement from Milky Way (Light-years) & Speed (km/s, relative to Milky Way) \\ \hline
1 & -300 & -4500 \\ \hline
2 & -150 & -2200 \\ \hline
3 (Milky Way) & 0 & 0 \\ \hline
4 & 190 & 2830 \\ \hline
5 & 450 & 6700 \\ \hline
\end{tabular}
\caption{\label{tab1} The displacements and velocities \textit{relative to the Milky Way} of five galaxies that lie on a straight line.}
\end{table}
\end{enumerate}
\end{document}