\title{Syllabus for Algebra-Based Physics-2: Electricity, Magnetism, and Modern Physics (PHYS135B-01)}
\author{Dr. Jordan Hanson - Whittier College Dept. of Physics and Astronomy}
\date{\today}
\documentclass[10pt]{article}
\usepackage[a4paper, total={18cm, 27cm}]{geometry}
\usepackage{outlines}
\usepackage[sfdefault]{FiraSans}
\usepackage{hyperref}
\begin{document}
\maketitle

\begin{abstract}
The concepts of algebra-based electricity, magnetism, and modern physics will be presented within the context of interactive problem-solving.  The course will begin with the concepts of electric charge, electrostatics, and electric potential.  Following electrostatics, applications to DC circuits will be covered.  Next, the topics of magnetism and electromagnetism will be covered, concluding with light and optics.  Time-permitting, selected topics in modern physics will be added.  The course work will include interactive computational exercises, analytic textbook problems, group-designed projects, and lab-based activities.
\end{abstract}
\noindent
\textit{\textbf{Pre-requisites}: PHYS-135A.} \\
\textit{\textbf{Course credits, Liberal Arts Categorization}: 4 Credits, None} \\
\textit{\textbf{Regular course hours}: Tuesday and Thursday from 8:50 - 10:50 in SLC 228} \\
\textit{\textbf{Instructor contact information}: jhanson2@whittier.edu, tel. 562.907.5130} \\
\textit{\textbf{Office hours}: } \\
\textit{\textbf{Attendance/Absence}: Students needing to reschedule midterms and exams should notify the professor a reasonable time beforehand. Further attendance issues are left to the discretion of the instructor}.\\ 
\textit{\textbf{Late work policy}: Late work is generally not accepted, but is left to the discretion of the instructor.} \\
\textit{\textbf{Text}: College Physics (openstax.org) -  \url{https://openstax.org/details/books/college-physics}} \\
\textit{\textbf{Grading}: There will be three tests, each examining conceptual understanding in step-by-step problems. Each midterm is worth 15\% of the final grade. The weekly online homework is worth 20\% of the grade. Interactive in-class activities will be worth 10\% of the final grade. Lab groups will present results of a group project worth 10\% of the grade. The final exam will be held on May 10th, 8:00-10:00 am, and will be worth 15\% of the grade.} \\
\textit{\textbf{Grade Settings}: $<60\%$ = F, $\geq 60\%, <70\%$ = D, $\geq 70\%, <80\%$ = C, $\geq 80\%, <90\%$ = B, $\geq 90\%, <100\%$ = A.  Pluses and minuses: 0-3\% minus, 3\%-6\% straight, 6\%-10\% plus (e.g. 79\% = C+, 91\% = A-)} \\
\textit{\textbf{Homework Sets}: Typically 5-10 problems per week, assigned and collected on Tuesdays.} \\
\textit{\textbf{Bonus Essay}: Students may submit an essay on the history of scientific developments covered in the course, due at the end of the semester.  The essay must be 10 pages, address scientific arguments and results, and must include references.  The grade of this paper will replace the lowest midterm grade, if it would raise the final grade.} \\
\textit{\textbf{ADA Statement on Disability Services}: The Americans with Disabilities Act (ADA) is a federal anti-discrimination statute that provides comprehensive civil rights protection for persons with disabilities. Among other things, this legislation requires that all students with disabilities be guaranteed a learning environment that provides for reasonable accommodation of their disabilities. If you believe you have a disability requiring an accommodation, please contact Disability Services: disabilityservices@whittier.edu, tel. 562.907.4825.} \\
\textit{\textbf{Academic Honesty Policy}: \url{http://www.whittier.edu/academics/academichonesty}} \\
\textit{\textbf{Course Objectives}:}
\begin{itemize}
\item To practice written and oral expression of scientifically technical ideas.
\item To solve word problems pertaining to physics and mathematics.
\item To model mathematically electrical systems like DC circuits.
\item To apply logical thinking to conceptually-posed physics problems.
\item To practice scientific experimentation, data analysis, and reporting of results.
\end{itemize}
\clearpage
\small
\textit{\textbf{Course Outline}:}
\begin{outline}[enumerate]
\1 Unit 0: Review of pre-requisite course, 135A
\2 Estimation, approximation, kinematics and Newton's Laws
\2 Work, energy and power
\2 Momentum, linear and angular
\1 Unit 1: Electrostatics - \textbf{Chapters 18 and 19}
\2 The Coulomb Force, and Newton's Second Law for electric charges
\2 Force Fields, comparisons between Newtonian gravity and Coulomb Force
\2 Electric potential, capacitance
\1 Unit 2: Electric Change in Motion - \textbf{Chapters 20 and 21}
\2 Electrical current and Ohm's law, resistors and conductors
\2 DC circuits, power and energy
\2 Human electrical systems
\1 First midterm exam, end of Unit 2
\1 Unit 3: DC circuits and Magnetism 1 - \textbf{Chapters 21 and 22}
\2 Series and Parallel resistors, Kirchhoff's Rules
\2 RC Circuits
\2 Ferromagnets, electromagnets, magnetic fields
\1 \textbf{Spring Break}: March 18th - March 22nd
\1 Unit 4: Magnetism 2 and AC technologies - \textbf{Chapters 22 and 23}
\2 Magnetic fields, forces and torques on moving charged particles and conductors
\2 The Hall effect
\2 Amp\`{e}re's Law: current inducing magnetic fields
\2 Faraday's Law and Lenz's Law
\2 Electric generators and transformers
\1 Second midterm exam, end of Unit 4
\1 Unit 5: Electromagnetic radiation and Optics - \textbf{Chapters 24 and 25}
\2 Maxwell's equations and electromagnetic radiation
\2 The electromagnetic spectrum and energy in light
\2 Optical rays, reflection and refraction
\2 Images and dispersion
\1 Unit 6: Vision and wave optics - \textbf{Chapters 26 and 27}
\2 Human vision
\2 Telescopes and microscopes
\2 Huygen's Principle and the slit-diffraction experiments
\2 Rayleigh criterion and resolution
\2 Films and polarization
\1 Third midterm exam, end of Unit 6
\1 Optional Unit 9 - Modern physics (special relativity and quantum mechanics)
\1 Unit 10 - \textbf{Cumulative Review, group presentations, and final exam}
\end{outline}
\end{document}
