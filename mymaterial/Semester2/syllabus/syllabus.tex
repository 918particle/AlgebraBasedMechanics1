\title{Syllabus for Algebra-Based Physics-2: Electricity, Magnetism, and Modern Physics (PHYS135B-01)}
\author{Dr. Jordan Hanson - Whittier College Dept. of Physics and Astronomy}
\date{\today}
\documentclass[10pt]{article}
\usepackage[a4paper, total={18cm, 27cm}]{geometry}
\usepackage{outlines}
%\usepackage[sfdefault]{FiraSans}
\usepackage{hyperref}
\begin{document}
\maketitle

\begin{abstract}
The concepts of algebra-based electricity, magnetism, and modern physics will be presented within the context of interactive problem-solving.  The course will begin with the concepts of electric charge, electrostatics, and electric potential.  Following electrostatics, applications to DC circuits will be covered.  The course will proceed with the addition of magnetism, induction, and AC circuits.  The course will conclude with geometric and wave optics, and selected topics in modern physics.  The course work will include interactive computational exercises, analytic textbook problems, group-designed projects, and lab-based activities.
\end{abstract}
\noindent
\textit{\textbf{Pre-requisites}: PHYS-135A.} \\
\textit{\textbf{Course credits, Liberal Arts Categorization}: 4 Credits, None} \\
\textit{\textbf{Regular course hours}: Tuesday and Thursday from 8:50 - 10:50 in SLC 228} \\
\textit{\textbf{Instructor contact information}: jhanson2@whittier.edu, tel. 562.907.5130} \\
\textit{\textbf{Office hours}: ?? in SLC 212} \\
\textit{\textbf{Attendance/Absence}: Students needing to reschedule midterms and exams should notify the professor a reasonable time beforehand. Further attendance issues are left to the discretion of the instructor}.\\ 
\textit{\textbf{Late work policy}: Late work will not be accepted.} \\
\textit{\textbf{Text}: College Physics (openstax.org) -  https://openstax.org/details/books/college-physics} \\
\textit{\textbf{Grading}: There will be three midterms, each worth 10\% of the final grade.  The weekly homework is worth 20\% of the grade.  Interactive in-class activities will be worth 15\% of the final grade.  Lab groups will present results of two group-designed projects worth 10\% of the grade each.  The final exam will be held on May 14th, 2018 from 8:00-10:00, and will be worth 15\% of the grade.} \\
\textit{\textbf{Grade Settings}: $<60\%$ = F, $>60\%,\leq 70\%$ = D, $>70\%,\leq80\%$ = C, $>80\%,\leq 90\%$ = B, $<90\%,\leq 100\%$ = A.  Pluses and minuses: 0-3\% minus, 3\%-6\% straight, 6\%-10\% plus (e.g. 79\% = C+, 91\% = A-)} \\
\textit{\textbf{Homework Sets}: Typically 10 problems per week, assigned and collected on Mondays.} \\
\textit{\textbf{Bonus Essay}: Students may submit an essay on the history of scientific developments covered in the course, due at the end of the semester.  The essay must address scientific arguments and results, must include library references, and must have at least 10 pages.  The grade of this paper will replace the lowest midterm score if submitted.} \\
\textit{\textbf{ADA Statement on Disability Services}: The Americans with Disabilities Act (ADA) is a federal anti-discrimination statute that provides comprehensive civil rights protection for persons with disabilities. Among other things, this legislation requires that all students with disabilities be guaranteed a learning environment that provides for reasonable accommodation of their disabilities. If you believe you have a disability requiring an accommodation, please contact Disability Services: disabilityservices@whittier.edu, tel. 562.907.4825.} \\
\textit{\textbf{Academic Honesty Policy}: \url{http://www.whittier.edu/academics/academichonesty}} \\
\textit{\textbf{Course Objectives}:}
\begin{itemize}
\item Written expression of quantitative and numerical ideas and arguments.
\item Oral expression of quantitative and numerical ideas and arguments.
\item Problem solving using numerical skills.
\item Mathematical modeling.
\item Logical thinking.
\item Analysis of data and results.
\end{itemize}
\small
\textit{\textbf{Course Outline}:}
\begin{outline}[enumerate]
\1 Unit 0: Review of pre-requisite course, 135A
\2 Estimation and Approximation
\2 Kinematics and Newton's Laws
\2 Work, Energy and Power
\2 Momentum, Linear and Angular
\1 Unit 1: Electrostatics - \textbf{Chapters 18 and 19}
\2 The Coulomb Force, and Newton's Second Law for electric charges
\2 Force Fields, comparisons between Newtonian gravity and Coulomb Force
\2 Electric potential, capacitors
\1 Unit 2: Electric Change in Motion - \textbf{Chapters 20 and 21}
\2 Electrical current
\2 Ohm's law and resistance, conductors
\2 Power and Energy
\2 Resistors and DC circuits
\2 Human electrical systems
\1 First midterm exam, end of Unit 2
\1 Unit 3: Circuits, DC instruments, and Magnetism 1 - \textbf{Chapters 21 and 22}
\2 Series and Parallel resistors; Kirchhoff's Rules
\2 Voltmeters and Ammeters
\2 RC Circuits
\2 Ferromagnets, electromagnets, magnetic fields
\1 Unit 4: Magnetism 2 and Field Induction 1 - \textbf{Chapters 22 and 23}
\2 Magnetic Fields, forces on moving charged particles
\2 Forces and torques on conductors
\2 The Hall effect
\2 Amp\`{e}re's Law: current inducing magnetic fields
\1 First In-Class Group Presentations, end of Unit 4
\1 Unit 5: Field Induction 2 - \textbf{Chapters 22 and 23}
\2 Induced EMF and magnetic flux
\2 Faraday's Law and Lenz's Law
\2 Electric generators and transformers
\2 Inductance, inductors, RL and RLC circuits
\1 \textbf{Spring Break}: March 19th - March 23rd
\1 Unit 6: Electromagnetic Waves and \textbf{Maxwell's Equations} - \textbf{Chapter 24}
\2 Maxwell's equations: predictions and observations unified
\2 Electromagnetic wave production
\2 The electromagnetic spectrum
\2 Energy in electromagnetic waves
\1 Second midterm exam, end of Unit 6
\1 Unit 7: Optics, Vision, and Optical Instruments - \textbf{Chapters 25 and 26}
\2 Light as a ray
\3 Reflection
\3 Refraction
\3 Total Internal Reflection
\2 Images, Lenses and Mirrors
\2 Human Vision and the human eye
\2 Microscopes and Telescopes
\1 Unit 8: Wave Optics - \textbf{Chapter 27}
\2 Wave interferance
\2 Huygen's Principle and the slit-diffraction experiments
\2 Rayleigh criterion and resolution
\2 Films and polarization
\1 Third midterm exam, end of Unit 8
\1 Unit 9: Special Relativity - \textbf{Chapter 28}
\2 Einstein's Postulates
\2 Modifications and generaliztions to kinematics
\2 Modifications and generaliztions to electromagnetism
\1 Second In-Class Group Presentations, end of Unit 9
\1 Unit 10 - \textbf{Cumultive Review and Selected Topics in Modern Physics}
\end{outline}
\end{document}
