\title{Rubric for the Final Project, PHYS135B}
\author{Dr. Jordan Hanson - Whittier College Dept. of Physics and Astronomy}
\date{\today}
\documentclass[10pt]{article}
\usepackage[a4paper, total={18cm, 27cm}]{geometry}
\usepackage{outlines}
\usepackage[sfdefault]{FiraSans}
\usepackage{hyperref}

\begin{document}
\maketitle

\noindent
\textit{\textbf{Requirements}: 1. Write down an idea for an experiment for yourself, or yourself and a lab partner, that is both doable and involves concepts from the course.  2. Construct the experiment and collect the data.  3. Create a 10-minute presentation on your results (5-10 slides).  4. Give the presentation in the final week of class.} \\
\begin{itemize}
\item\textbf{Project idea}: The project idea should be constructable at home, or it can be a theoretical calculation.  Either way it should be doable for you and your lab partner (if you have one).
\item \textbf{Experiment}: The take-home experiments proposed in the text are a good start for ideas.  Another source is DIY electricity experiments on YouTube (hompolar motors, AC motors etc.). The experiment should be a device or setup that is cheap, safe, and easy to build.  The experiment should be focused on a topic covered this semester.
\item \textbf{Presentation}: The final project can be created in one of two options. Option A: A 10 minute traditional presentation with several minutes for questions. Option B: Digital liberal arts style, in the form of video or digital book form that educates the class on a topic. Regardless of the option, students will all present their work to the class at the
end of the module.
\item \textbf{Speaking}: When a pair of students gives the presentation, each member of the group should give at least part of the presentation.
\end{itemize}
\textit{\textbf{Example outline of the presentation}:}
\begin{outline}[enumerate]
\1 Slide 1: \textit{Measuring the coefficient of static friction} - by Jordan C. Hanson
\1 Slide 2: Introduction: ``The force of friction experienced by a stationary object is proportional to the coefficient of static friction, $\mu_{\rm s}$.  In this experiment, we measure $\mu_{\rm s}$ for a variety of materials.''
\1 Slide 3: ``(Diagram) A textbook is titled at an increasing angle until a given object begins to slide across it.  The angle is measured with a protracter, and the mass of the object is measured with a scale.  The result for $\mu_{\rm s}$ will be given by $\tan\theta$, where $\theta$ is the angle of incline.  The angle must be the maximum angle acheived before the object slides.''
\1 Slide 4: \textit{Tables of data for the angle $\theta$ based on object type are given.}  ``Here is our data.  As you can see...''
\1 Slide 5: ``The predicted coefficients of static friction for the objects are compared to the measured ones.  There are a few discrepancies...but we agree in general with the predictions.''
\1 Slide 6: ``In conclusion, the predicted coefficients of friction were measured with \textbf{standard deviations} in agreement with the global values.''
\end{outline}
\textbf{Grading}: 30\% of the grade will be assigned based on \textit{attention to detail} in the project proposal.  What parts did you need? How is this device built?  Another 30\% will be assigned based on the execution of the experiment.  Are we allowing any unnecessary errors?  Are there any ways we can be more precise?  Finally, 40\% of the grade will be assigned on \textit{how clearly you related the findings to the class.}  Are you plotting or listing the data in such a way that other people can understand it?  Are there unit errors?  \textbf{\textit{Can people read your graphs and tables?}}
\end{document}
