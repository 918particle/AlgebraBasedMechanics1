\title{Study Guide for PHYS135B Module 2, Spring 2021}
\author{Dr. Jordan Hanson - Whittier College Dept. of Physics and Astronomy}
\date{\today}
\documentclass[10pt]{article}
\usepackage[a4paper, total={18cm, 27cm}]{geometry}
\usepackage{outlines}
\usepackage{graphicx}
\begin{document}
\maketitle

\textbf{Instructions:} Work each problem \textit{before} checking your answer with the key (to follow on Moodle). \\ \vspace{0.25cm}

\section{Memory Bank}

\begin{enumerate}
\item $V = (4/3) \pi r^3$ ... The volume of a sphere.
\item $m = \rho V$ ... The relationship between mass $m$, density $\rho$, and volume $V$.
\item $\vec{F} = k \frac{q_1 q_2}{r^2}\hat{r}$ ... Coulomb Force
\item $k = 9 \times 10^{9}$ N C$^{-2}$ m$^{2}$ ... Remember $k = 1/(4\pi \epsilon_0)$.
\item $q_e = 1.6 \times 10^{-19}$ C ... Charge of an electron/proton
\item Atomic mass: the number of grams per mole of a substance
\item $N_A = 6.03 \times 10^{23}$ ... Avagadro's number
\item $\vec{F} = q \vec{E}$ ... Electric field and charge
\item $\vec{E}(z) = \frac{\sigma}{\epsilon_0}\hat{z}$ ... Electric field of two oppositely charge planes each with charge density $\sigma$
\item $\epsilon_0 \approx 8.85 \times 10^{-12}$ F/m
\item $U = q\Delta V$ ... Potential energy and voltage
\item 1 eV: an electron-Volt is the amount of energy one electron gains through 1 V.
\item $V(r) = k\frac{q}{r}$ ... Voltage of a point charge
\item $\vec{E} = -\frac{\Delta V}{\Delta x}$ ... E-field is the slope or change in voltage with respect to distance
\item $V(x) = -E x + V_0$ ... Voltage is linear between two charge planes
\item $Q = C\Delta V$ ... Definition of capacitance
\item $C = \frac{\epsilon_0 A}{d}$ ... Capacitance of a parallel plate capacitor
\item $C_{tot}^{-1} = C_1^{-1} + C_2^{-2}$ ... Adding two capacitors \textit{in series.}
\item $C_{tot} = C_1 + C_2$ ... Adding two capacitors \textit{in parallel.}
\item $i(t) = \Delta Q/\Delta t$ ... Definition of current.
\item $v_d = i/(nqA)$ ... Charge drift velocity in a current $i$ in a conductor with number density $n$ and area $A$.
\item $R_{tot}^{-1} = R_1^{-1} + R_2^{-1}$ ... Adding two capacitors \textit{in parallel.}
\item $R_{tot} = R_1 + R_2$ ... Adding two capacitors \textit{in series.}
\item $\Delta V = I R_{\rm tot}$ ... Ohm's Law
\item $P = I V$ ... Relationship between power, current, and voltage.
\item $V_{\rm C}(t) = \epsilon_1 \left(1 - \exp(-t/\tau)\right)$ ... voltage across the capacitor in an RC series circuit.  The time constant is $\tau = RC$.
\item $i(t) = \frac{\epsilon_1}{R} \exp(-t/\tau)$ ... Current in an RC series circuit.
\item $i_{\rm in} = i_{\rm out}$ ... Kirchhoff's junction rule.
\item $\epsilon_1 + \epsilon_2 + \epsilon_3 + ... = 0$ ... Kirchhoff's loop rule.
\end{enumerate}

\clearpage

\section{Electric Charge and Electric Fields}

\begin{enumerate}
\item (a) Two charges exert $F_{\rm C} = 5.00$ N of force on each other. What will $F_{\rm C}$ be if the distance between them triples? (b) If one charge is $1$ nC, and the other is $2$ nC, what is the distance between them if $F_{\rm C} = 5.00$ N? \\ \vspace{1cm}
\item 
\begin{figure}
\centering
\includegraphics[width=0.3\textwidth]{figures/mill.jpeg}
\caption{\label{fig:mill} The classic Millikan oil drop experiment was a measurement of the charge of an electron.}
\end{figure}
The classic Millikan oil drop experiment was the first to measure accurately the electron charge. Oil drops were suspended against the gravitational force by a vertical electric field. (See Fig. \ref{fig:mill}.) The drops have radius $1.0 \mu$m, and a density of 920 kg/m$^3$. (a) Find the weight of the drop. (b) If the drop has a single excess electron, find the electric field strength needed to balance its weight. \\ \vspace{1.75cm}
\item Suppose two positive, identical charges are located a distance $d$ apart. (a) Sketch the electric field below.  (b) Sketch the electric field if instead one of the charges is negative. \\ \vspace{2.5cm}
\end{enumerate}

\section{Potential Energy and Voltage}

\begin{enumerate}
\item What is the electric field across an 10.00 nm thick human nerve cell membrane if (a) the voltage across it is 50 mV? You may assume a uniform electric field. (b) Suppose this cell membrane is part of a nerve cell.  How much energy would an electron gain if dropped through the 50 mV voltage and accelerated across the cell freely?  Express your anser in electron-Volts (eV). \\ \vspace{2cm}
\item \textbf{Think back to the PhET simulations of parallel lines of charge.}  Suppose a parallel plate capacitor is formed from a positive plate and a negative plate of charge.  The plates' areas $A$ are the same, and the plates' charges ($\pm Q$), and charge densities ($\pm Q/A = \pm \sigma$) are the same as well. (a) Write the expression for the electric field between the plates. (b) Suppose $Q = 1$ nC, and $A = 10$ mm$^2$.  What is the value of the electric field between the plates?  (c) Suppose 0 volts corresponds to the location of the negative plate.  Draw the voltage as a function of distance between the plates.  (d) What is the voltage near the positive plate, if the plates are are separated by a distance $d = 1$ mm? \\ \vspace{2cm}
\end{enumerate}

\section{Capacitors}

\begin{enumerate}
\item What is the capacitance of the capacitor in the previous problem? \\ \vspace{1cm}
\item (a) Consider the same capacitor again, and suppose a second identical capacitor is connected \textit{in parallel} with it.  What is the total capacitance? (b) How much charge would the pair of capacitors store if the voltage across them was 5 volts? \\ \vspace{1.5cm}
\item How much energy in Joules would this charge have if it was all put to work? \\ \vspace{1cm}
\end{enumerate}

\section{Current, Resistance, and DC Circuits}

\begin{enumerate}

\item Three identical resistors $R$ are connected \textit{in parallel}, and powered by an adjustable voltage source. The voltage and \textit{total current} measurements are shown below. Determine the value of $R$.

\begin{figure}[hb]
\centering
\includegraphics[width=0.33\textwidth]{figures/ohm1.png}
\caption{\label{fig:ohm1} A graph of voltage versus current.}
\end{figure}

\vspace{1cm}
\item (a) Using the PHeT tool for DC circuit construction, design a circuit in which a battery with \textit{fixed voltage} lights a bulb, but the bulb brightness can be dimmed or brightened.  \textit{Hint: use other components in series with the bulb.} Draw your design below. (b) Now make a parallel circuit in which two bulbs can be brightened or dimmed independently, and use switches to turn them on or off independently.  Draw your design below. 
\end{enumerate}

\end{document}