\title{Study Guide for Midterm 1 for Algebra-Based Physics-2: Electricity, Magnetism, and Modern Physics (PHYS135B-01)}
\author{Dr. Jordan Hanson - Whittier College Dept. of Physics and Astronomy}
\date{\today}
\documentclass[10pt]{article}
\usepackage[a4paper, total={18cm, 27cm}]{geometry}
\usepackage{outlines}
\usepackage[sfdefault]{FiraSans}
\usepackage{hyperref}
\begin{document}
\maketitle

\begin{enumerate}
\item \textbf{Working with orders magnitude, and approximation.} Use the following information: there are 107 grams per mole of silver, $\approx 6 \times 10^{23}$ atoms per mole, and one conducting electron per silver atom. 
\begin{enumerate}
\item Estimate the number of free electrons in 1 gram of silver. \\ \vspace{1cm}
\item If $10^{10}$ conducting electrons are removed every $10^{-10}$ seconds, how long would it take to remove them all? (Think of this as a current). \\ \vspace{1cm}
\end{enumerate}
\item \textbf{The Coulomb force, and static charge.} The Coulomb force between two charges $q_1$ and $q_2$ separated by a distance $r$ is 
\begin{equation}
 	\vec{F}_C = k\frac{q_1 q_1}{r^2}\hat{r}
\end{equation}
The vector $\hat{r}$ is a unit vector pointing from one charge toward the other, and $k = 9\times 10^9$ N C$^{-2}$m$^{2}$.  Suppose $q_1 = 4.0 \mu$C, and $q_2 = 4.0 \mu$C, and $r=4.0 \mu$m. \\
\begin{enumerate}
\item What is the magnitude of the force between the charges, and in which direction does the force point? \\ \vspace{2cm}
\end{enumerate}
\item \textbf{Drawing electric field lines, 1.} Recall our experience with the PHeT simulation of charges and fields.  (a) Create a charge distribution of two opposite charges $\pm q$. (b) Illustrate the correct electric field between the charge distributions by drawing electric field lines. \\ \vspace{3cm}
\item \textbf{Electric potential and electric field.} Recall that the relationship between a uniform electric field $E$ and the associated change in voltage $V$ is $V = Ed$, where $d$ is a distance.  Two uniformly charged plates with charges $+Q$ and $-Q$ create a uniform electric field $E$ between them.  Let the voltage at the negatively charged plate be 0 V.
\begin{enumerate}
\item 	If the distance between the plates is 80 mm, and the electric field has a value of 0.8 V/mm, what is the voltage at the positive plate? \\ \vspace{1cm}
\end{enumerate}
\item \textbf{Capacitors, and capacitance.} Recall that the charge $Q$ stored on a capacitor is $CV$ for a given potential $V$, and that the unit of capacitance is the \textit{Farad}, F.
\begin{enumerate}
\item How much charge is stored on a capacitor with $C = 0.1\mu$F, if the voltage is $V = 12$ V? \\ \vspace{0.75cm}
\end{enumerate}
\item \textbf{Definition of current, resistance, and Ohm's Law} Recall that \textit{current} is the change in charge per unit time, $I = \Delta Q/\Delta t$, and that the unit of current is the \textit{amp}, A, which is 1 C/s.  Also recall that Ohm's Law is $V=IR$, where $V$ is the voltage, $I$ is the current, and $R$ is the total effective resistance.
\begin{enumerate}
\item How much current flows through a circuit that lights a lightbulb, if the voltage is 24 V, and the lightbulb has a resistance of $100$ Ohms? \\ \vspace{1cm}
\item Recall that the relationship between the power $P$ consumed by a resistor drawing a current $I$ while being given a voltage $V$ is $P=IV$.  How many watts does the light bulb consume? \\ \vspace{1cm}
\vspace{1cm}
\item Draw a graph of voltage versus current for the lightbulb in part (a), assuming the voltage can vary.\\ \vspace{2cm}
\item Suppose the second light bulb is instead connected \textit{in parallel} with the first light bulb.  What is the new current? \\ \vspace{1cm}
\end{enumerate}
\item \textbf{Nerve signals.} Please review the section of Chapter 20 on nerve signal conduction.  Pay special attention to the \textit{action impulse}, which is a voltage versus time.
\end{enumerate}
\end{document}