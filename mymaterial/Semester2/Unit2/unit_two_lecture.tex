\documentclass{beamer}
\usetheme{metropolis}
\usepackage{graphicx}
\usepackage{subfig}
\usepackage{tcolorbox}
\title{Algebra-Based Physics-2: Electricity, Magnetism, and Modern Physics (PHYS135B-01): Unit 2}
\author{Jordan Hanson}
\institute{Whittier College Department of Physics and Astronomy}

\begin{document}
\maketitle

\section{Unit 1 Review}

\begin{frame}{Unit 1 Review}
\textbf{Reading: Chapters 18 and 19}
\begin{enumerate}
\item Charge, mass, the Coulomb force, and the gravitational force
\item Force fields
\item Electric potential and capacitance
\end{enumerate}
\end{frame}

\section{Unit 1 Review Problems}

\begin{frame}{Unit 1 Review Problems}
\small
\alert{Charged black holes}: Suppose two black holes with the same mass are pulled towards each other by gravity.  Each, however, has a slight positive charge.  If the Coulomb force balances with gravity, what is the charge of the black holes?  Each black hole has a mass of $6 \times 10^{30}$ kg, $G = 7 \times 10^{-11}$ m$^3$ s$^{-2}$ kg$^{-1}$, and $\epsilon_0 = 9\times 10^{-12}$ N$^{-1}$ m$^{-2}$ C$^{2}$.
\begin{itemize}
\item A: $5 \times 10^{40}$ C
\item B: $5 \times 10^{30}$ C
\item C: $5 \times 10^{20}$ C
\item D: $5 \times 10^{10}$ C
\end{itemize}
\textit{Is this number surprisingly small, or surprisingly large?}
\end{frame}

\section{Summary}

\begin{frame}{Unit 2 Summary}
\textbf{Reading: Chapters 20 and 21}
\begin{enumerate}
\item Current, Ohm's Law, resistors and conductors
\item DC circuits I
\item Nerve signals
\item \alert{DC circuits II}
\end{enumerate}
\end{frame}

\section{JITT - Reading Quiz Results}

\section{Current}

\begin{frame}{Current}
\underline{Notions of current:}
\begin{itemize}
	\item $I = \frac{\Delta Q}{\Delta t}$ - The derivative of charge
	\item The \textit{movement} of electrons
	\item The \textit{flow} of charge
	\item Number of Coulombs per second (1 Amp = C/s)
\end{itemize}
\underline{There is an interesting problem with the notion of current} \underline{as movement of charges.} \\ \vspace{0.5cm}
\begin{columns}[T]
\begin{column}{0.5\textwidth}
Speed of typical electronic signals: $\approx 10^{8}$ m/s
\end{column}
\begin{column}{0.5\textwidth}
Typical speed of actual charges passing through a conductor under voltage: $\approx 10^{-4}$ m/s
\end{column}
\end{columns} \vspace{0.25cm}
\textbf{Since there is a 12 order of magnitude range, it's probably a good idea to ponder...}
\end{frame}

\begin{frame}{Current}
\small
Are the electrons colliding/interacting to form electrical signals?  Or just moving all together?
\begin{figure}
\includegraphics[width=0.5\textwidth]{figures/current1.png}
\caption{\label{fig:current1} The \textit{drift velocity} is the average velocity of an electron, and current is derived from this average velocity.}
\end{figure}
\url{https://youtu.be/8dgyPRA86K0}
\end{frame}

\begin{frame}{Current}
\small
The answer lies in the fact that we are no longer dealing with \alert{contact forces}, but long-range interactions like the Coulomb force.
\begin{figure}
\includegraphics[width=0.5\textwidth]{figures/current2.png}
\caption{\label{fig:current2} The \textit{drift velocity} is the average velocity of an electron, and current is derived from this average velocity.}
\end{figure}
\textbf{Electrical signals are more like a \alert{wave on a string}}: \\ \url{https://phet.colorado.edu/en/simulation/legacy/wave-on-a-string}
\end{frame}

\begin{frame}{Current}
So we see how electrical signals can move near the speed of light, but we measure the movements of electrons in circuits to be slow.  Can we make a calculation to understand the speed of the electrons? \\ \vspace{0.5cm}
\begin{figure}
\includegraphics[width=0.5\textwidth]{figures/current3.png}
\caption{\label{fig:current3} Consider the volume $V$ of conductor with cross-sectional area $A$ and length $\Delta x$, having $n$ free electrons per unit volume.}
\end{figure}
\end{frame}

\begin{frame}{Current}
An \textbf{amp} is one \textit{Coulomb} per \textit{second}.  The definition of current is
\begin{equation}
I = \frac{\Delta Q}{\Delta t} = \frac{qnA\Delta x}{\Delta t} = q n A v_{\rm d}
\end{equation}
Solving for drift velocity:
\begin{equation}
v_{\rm d} = \frac{I}{q n A}
\end{equation}
Suppose our conductor is a wire with radius $r$ and $A = \pi r^2$.  Substituting,
\begin{equation}
v_{\rm d} = \frac{I}{\pi q n r^2}
\end{equation}
Remember that $q = 1.6 \times 10^{-19}$ C, and $n$ is the number of free electrons \textit{per atom per unit volume}.  How do we get this number?
\end{frame}

\begin{frame}{Current}
\textbf{Number density}: Let's examine copper, a common wire material with one free electron per atom.  Copper has a density of 8800 kg/m$^3$, and $0.06354$ kg/mol.  There are $6.02 \times 10^{23}$ atoms/mol.  How many free electrons per m$^3$ of copper?
\begin{itemize}
\item A: $8.342 \times 10^{26}$ free electrons per kg
\item B: $8.342 \times 10^{27}$ free electrons per kg
\item C: $8.342 \times 10^{28}$ free electrons per kg
\item D: $8.342 \times 10^{29}$ free electrons per kg
\end{itemize}
\end{frame}

\begin{frame}{Current}
Consider a copper wire with radius $r = 2.053$ mm that is carrying 20.0 A of current.  Using $q = 1.6\times 10^{19}$, and $n = 8.342 \times 10^{28}$ electrons/m$^3$, and $v_{\rm d} = I/(\pi q n r^2)$, compute the drift velocity of charge in the wire.  \textit{This is a common situation in household wiring.}
\begin{itemize}
\item A: $4.53 \times 10^{1}$ m/s
\item B: $2.25 \times 10^{-2}$ m/s
\item C: $2.25 \times 10^{-3}$ m/s
\item D: $4.53 \times 10^{-4}$ m/s
\end{itemize}
\end{frame}

\section{Ohm's Law}

\begin{frame}{Current}
\small
Electrons are now moving through our copper wire ($v_{\rm d} \propto I$, $v_{\rm d} \propto A^{-1}$).  What happens when the electrons, which have had some PE converted to KE, encounter objects that are not conductors?  \textbf{They deposit energy and move forward.} \\ \vspace{0.5cm}
\begin{figure}
\centering
\includegraphics[width=0.5\textwidth]{figures/lakes.jpg}
\caption{\label{fig:lakes} Current is comprised of electrons that deposit energy as they move to lower voltages.}
\end{figure}
\textbf{\alert{PhET}}: \url{https://phet.colorado.edu/en/simulation/circuit-construction-kit-dc}
\end{frame}

\begin{frame}{Current}
\textbf{\alert{PhET}}: Create a DC circuit involving a battery, resistor (the brown striped object), a light bulb, and a switch.
\begin{enumerate}
\item Place the battery and connect to it a wire, and attach a resistor to that wire.
\item To the other end of the resistor, connect a switch and leave it open.
\item Connect a light bulb to the other end of the switch, and connect a wire from the light bulb to the battery.
\item The properties of each circuit element can be edited by clicking on the element.
\end{enumerate}
\end{frame}

\begin{frame}{Current}
\begin{figure}
\centering
\includegraphics[width=0.75\textwidth]{figures/PhETBulb.png}
\caption{\label{fig:phetb} Your circuit should resemble this.}
\end{figure}
\end{frame}

\begin{frame}{Current}
\textbf{\alert{PhET}}: Make observations.
\begin{enumerate}
\item What happens to the drift velocity of the electrons as you raise and lower the resistance?  Why do we call light bulbs and the brown striped objects ``resistors?''
\item Since we cannot change the cross-sectional area of the wire independently, we can treat $v_{\rm d} \propto I$.
\item How does the current change if you increase the voltage?
\end{enumerate}
\end{frame}

\begin{frame}{Current}
\textbf{\alert{PhET}}: \textbf{The unit of resistance is the Ohm.  We use the symbol $\Omega$ for Ohms, and $1\Omega = 1$V/A.}
\begin{enumerate}
\item There are two devices available: the \textit{voltmeter}, and the \textit{ammeter}.  Using these devices, measure the voltage drop across the resistor and the light bulb, and the current flowing through the circuit.
\item How are voltage and current and resistance related?  Derive an equation from the data.
\end{enumerate}
\end{frame}

\begin{frame}{Current}
\textbf{Ohm's Law}: Let $V$ be the voltage change across a resistor with resistance $R$, and let $i$ be the current flowing through the resistor.  Ohm's law states that
\begin{equation}
\boxed{
V = iR}
\end{equation}
for materials that fall into the category of \textit{Ohmic}.
\end{frame}

\begin{frame}{Current}
\textbf{\alert{PhET}}: \textbf{How do we deal with more complex circuits?} There must be a way to ``add'' resistors.
\begin{enumerate}
\item Create a circuit that involves just a hairy mess of resistors.  Connect them \textit{in series} and \textit{in parallel}.
\item Calculate the \textit{effective total resistance} by plotting an $i-V$ curve of the system.  Measure $i$ and $V$ by changing the voltage and using the voltmeter and ammeter.
\item What is the effective total resistance of the circuit?  How did you obtain this number from the $i-V$ curve?
\end{enumerate}
\end{frame}

\begin{frame}{Current}
As you may have discovered, \textit{resistors in series} \textbf{add}:
\begin{equation}
R_{\rm tot} = R_1 + R_2 + ...
\end{equation}
\textit{Resistors in parallel} \textbf{add their reciprocals}:
\begin{equation}
\frac{1}{R_{\rm tot}} = \frac{1}{R_1}+\frac{1}{R_2}+...
\end{equation}
\end{frame}

\begin{frame}{Current}
Resitance is not an \textit{intrinsic property} of materials.  Imagine a 0.1 m-long wire (which is a conductor) actually having a small resistance.  What about that same wire, but 1 kilometer long?
\begin{itemize}
\item Electrons lose some fixed energy per unit length in a given material (\textit{Joule heating})
\item Electrons lose more energy if the wire is thinner (\textit{Joule heating})
\end{itemize}
\textbf{Resistivity} $\rho$ is defined in terms of resistance $R$, length $L$ and cross-sectional area $A$ as
\begin{equation}
R = \rho \left( \frac{L}{A} \right)
\end{equation}
\end{frame}

\begin{frame}{Current}
\begin{columns}[T]
\begin{column}{0.5\textwidth}
\begin{figure}
\centering
\includegraphics[width=0.7\textwidth]{figures/rho1.png}
\caption{\label{fig:rho1} Conductor resistivities are in units of $\Omega$ m, and are small but non-zero.}
\end{figure}
\end{column}
\begin{column}{0.5\textwidth}
\begin{figure}
\centering
\includegraphics[width=0.7\textwidth]{figures/rho2.png}
\caption{\label{fig:rho2} Semiconductor resistivities are in units of $\Omega$ m, and are larger.}
\end{figure}
\end{column}
\end{columns}
\end{frame}

\begin{frame}{Current}
Two copper wires need to be attached to carry current to the top floor of a building.  One has a cross-sectional area of 2 mm and is 10 meters long, while the other has a cross-sectional area of 4 mm and is 15 meters long.  What is the total resistance of the two wires attached in series?
\begin{itemize}
\item A: about 1 m$\Omega$
\item B: about 20 m$\Omega$
\item C: about 200 m$\Omega$
\item D: about 2 $\Omega$
\end{itemize}
\end{frame}

\begin{frame}{Current}
Consider the same system.  If we attach a battery and use the wire to feed voltage to some circuit drawing 3.0 A of current, what is the voltage drop due to just the wire?
\begin{itemize}
\item A: about 60 mV
\item B: about 600 mV
\item C: about 6 V
\item D: Current will not flow at all
\end{itemize}
\end{frame}

\begin{frame}{Current}
What would the resistance be if the wire system was 10 times as long?
\begin{itemize}
\item A: about 60 mV
\item B: about 600 mV
\item C: about 6 V
\item D: Current will not flow at all
\end{itemize}
\textit{So you can start to see that resistance matters even for conductors, if the current is traveling for long distances.  Often manufacturers quote the Ohms per foot in wire data sheets.}
\end{frame}

\begin{frame}{Current}
Resistivity depends on temperature in the following way:
\begin{equation}
\rho = \rho_{\rm 0} \left(1 + \alpha \Delta T\right)
\end{equation}
For most conductors, $\alpha$ is small, on the order of $10^{-3}$ $^\circ$C$^{-1}$.
\end{frame}

\begin{frame}{Current}
\begin{figure}
\centering
\includegraphics[width=0.35\textwidth]{figures/rho3.png}
\caption{\label{fig:rho3} Conductor resistivities depend on temperature.}
\end{figure}
\end{frame}

\begin{frame}{Current}
Continuing with the same example of the long copper wires attached together (10.0 m and 15.0 m), if the temperature increases by 50.0 deg C, what is the new resistance?
\begin{itemize}
\item A: 16 m$\Omega$
\item B: 20 m$\Omega$
\item C: 24 m$\Omega$
\item D: 30 m$\Omega$
\end{itemize}
\end{frame}

\begin{frame}{Current}
\textbf{Power} is consumed in resistors, since charges are losing energy and new charges are showing up at a certain rate.  Consider the PE converted to work in a resistor:
\begin{align}
\Delta PE =& \Delta q\Delta V \\
\Delta W =& \Delta q i R \\
\frac{\Delta W}{\Delta t} =& \frac{\Delta q}{\Delta t} i R = i^2 R \\
\frac{\Delta W}{\Delta t} =& i V \\
P =& iV
\end{align}
The formula $P = iV$ shows that the wattage required by some device in a circuit will pull current according to the voltage of the battery.
\end{frame}

\begin{frame}{Current}
How much current is required by a 50 W light bulb if the voltage supplying it is 120 V?
\begin{itemize}
\item A: 420 mA
\item B: 120 mA
\item C: 50 mA
\item D: 50 V
\end{itemize}
\end{frame}

\section{Graphical Analysis of Simple Circuits}

\section{Conclusion}

\begin{frame}{Unit 2 Summary}
\textbf{Reading: Chapters 20 and 21}
\begin{enumerate}
\item Current, Ohm's Law, resistors and conductors
\item DC circuits I
\item Nerve signals
\item \alert{DC circuits II}
\end{enumerate}
\end{frame}

\section{Answers}

\begin{frame}{Answers}
\tiny
\begin{columns}[T]
\begin{column}{0.5\textwidth}
\begin{itemize}
\item $5 \times 10^{30}$ C (double check this)
\item $8.342 \times 10^{28}$ free electrons per kg
\item $4.53 \times 10^{-4}$ m/s
\item about 20 m$\Omega$
\item about 60 mV
\item about 600 mV
\item 24 m$\Omega$
\item 420 mA
\end{itemize}
\end{column}
\begin{column}{0.5\textwidth}
\begin{itemize}
\item 
\end{itemize}
\end{column}
\end{columns}
\end{frame}

\end{document}
