\documentclass{beamer}
\usetheme{metropolis}
\usepackage{graphicx}
\usepackage{subfig}
\usepackage{tcolorbox}
\title{Algebra-Based Physics-2: Electricity, Magnetism, and Modern Physics (PHYS135B-01): Unit 2}
\author{Jordan Hanson}
\institute{Whittier College Department of Physics and Astronomy}

\begin{document}
\maketitle

\section{Unit 1 Review}

\begin{frame}{Unit 1 Review}
\textbf{Reading: Chapters 18 and 19}
\begin{enumerate}
\item Charge, mass, the Coulomb force, and the gravitational force
\item Force fields
\item Electric potential and capacitance
\end{enumerate}
\end{frame}

\section{Unit 1 Review Problems}

\begin{frame}{Unit 1 Review Problems}
\small
\alert{Charged black holes}: Suppose two black holes with the same mass are pulled towards each other by gravity.  Each, however, has a slight positive charge.  If the Coulomb force balances with gravity, what is the charge of the black holes?  Each black hole has a mass of $6 \times 10^{30}$ kg, $G = 7 \times 10^{-11}$ m$^3$ s$^{-2}$ kg$^{-1}$, and $\epsilon_0 = 9\times 10^{-12}$ N$^{-1}$ m$^{-2}$ C$^{2}$.
\begin{itemize}
\item A: $5 \times 10^{40}$ C
\item B: $5 \times 10^{30}$ C
\item C: $5 \times 10^{20}$ C
\item D: $5 \times 10^{10}$ C
\end{itemize}
\textit{Is this number surprisingly small, or surprisingly large?}
\end{frame}

\section{Summary}

\begin{frame}{Unit 2 Summary}
\textbf{Reading: Chapters 20 and 21}
\begin{enumerate}
\item Current, Ohm's Law, resistors and conductors
\item DC circuits I
\item Nerve signals
\item \alert{DC circuits II}
\end{enumerate}
\end{frame}

\section{JITT - Reading Quiz Results}

\section{Current}

\begin{frame}{Current}
\underline{Notions of current:}
\begin{itemize}
	\item $I = \frac{\Delta Q}{\Delta t}$ - The derivative of charge
	\item The \textit{movement} of electrons
	\item The \textit{flow} of charge
	\item Number of Coulombs per second (1 Amp = C/s)
\end{itemize}
There is an interesting problem with the notion of current as movement of charges. \\ \vspace{0.5cm}
\begin{columns}[T]
\begin{column}{0.5\textwidth}
\underline{Speed of typical electronic signals}: $\approx 10^{8}$ m/s
\end{column}
\begin{column}{0.5\textwidth}
\underline{Typical speed of actual charges passing through a conductor under voltage}: $\approx 10^{-4}$ m/s
\end{column}
\end{columns} \vspace{1cm}
\textbf{Since there is a 12 order of magnitude range, it's probably a good idea to ponder...}
\end{frame}

\begin{frame}{Current}
The answer lies in the fact that we are no longer dealing with \alert{contact forces}, but long-range interactions like the Coulomb force. \\
\end{frame}

\begin{frame}{Current}
%\begin{figure}
%\includegraphics[0.7\textwidth]{figures/}
%\caption{\label{fig:} The \textit{drift velocity} is the average velocity of an electron}
%\end{figure}	
\end{frame}


\section{Conclusion}

\begin{frame}{Unit 2 Summary}
\textbf{Reading: Chapters 20 and 21}
\begin{enumerate}
\item Current, Ohm's Law, resistors and conductors
\item DC circuits I
\item Nerve signals
\item \alert{DC circuits II}
\end{enumerate}
\end{frame}

\section{Answers}

\begin{frame}{Answers}
\tiny
\begin{columns}[T]
\begin{column}{0.5\textwidth}
\begin{itemize}
\item 
\end{itemize}
\end{column}
\begin{column}{0.5\textwidth}
\begin{itemize}
\item 
\end{itemize}
\end{column}
\end{columns}
\end{frame}

\end{document}
