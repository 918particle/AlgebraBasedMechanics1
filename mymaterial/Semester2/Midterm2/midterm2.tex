\title{Study Guide for Midterm 1 for Algebra-Based Physics-2: Electricity, Magnetism, and Modern Physics (PHYS135B-01)}
\author{Dr. Jordan Hanson - Whittier College Dept. of Physics and Astronomy}
\date{\today}
\documentclass[10pt]{article}
\usepackage[a4paper, total={18cm, 27cm}]{geometry}
\usepackage{outlines}
\usepackage[sfdefault]{FiraSans}
\usepackage{hyperref}
\usepackage{graphicx}
\begin{document}
\maketitle

\begin{enumerate}
\item \textbf{Applications of circuits, Chapter 21}: (Recall that $P = iV$, $V = iR$, and that two resistors in parallel combine like $R_{tot}^{-1} = R_{1}^{-1} + R_{2}^{-1}$.
\begin{enumerate}
\item A 250 W space heater, a 1000 W microwave, and a 50 W coffee pot are connected to the 120 V outlet.  Will this blow the 12 amp fuse? \\ \\ \\
\item A 200 W toaster, a 1000 W microwave, and a 250 W mini-fridge are connected to the 120 V outlet.  Will this blow the 12 amp fuse? \\ \\ \\
Suppose the mini-fridge in the previous problem were disconnected.
\begin{enumerate}
\item What would the current be? \\ \\ \\
\item What would be the total effective resistance? \\ \\ \\
\item What are the resistances of the separate devices? \\ \\ \\ \\
\end{enumerate}
\end{enumerate}
\item \textbf{Magnetism, Chapter 22}: Recall the form of the Lorentz force is $\vec{F} = q\vec{v} \times \vec{B}$, and that centripetal force is $F = mv^2/r$.
\begin{enumerate}
\item Draw the magnetic field of the Earth, and indicate the trajectory of a positively charged particle from space that enters this field.  Draw a trajectory for a negatively charged particle. \\ \vspace{4cm}
\item Suppose a positively charged particle has a velocity of $\vec{v} = v_0 \hat{i}$, and that it moves through a uniform magnetic field $\vec{B} = B_0 \hat{j}$.  In which direction does the particle accelerate? \\ \\ \\
\item Suppose a positively charged particle has a velocity of $\vec{v} = v_0 \hat{i}$, and that it moves through a uniform magnetic field $\vec{B} = B_0 \hat{k}$.  In which direction does the particle accelerate? \\ \\
\item Suppose a charged particle with charge $q$, mass $m$ and velocity $\vec{v} = 2 \times 10^{8} \hat{x}$ m/s moves in a circle with radius 1.0 m through a magnetic field $\vec{B} = 10^{-3} \hat{y}$ T.  What is the charge-to-mass ratio $q/m$? \\ \vspace{3cm}
\item Suppose we use an electric field to determine that the charge of these particles is positive and twice that of a proton ($q = 2 \times 1.6 \times 10^{-19}$ C).  What is the mass of these particles? \\ \vspace{3cm}
\end{enumerate}
\item \textbf{Magnetism 2, Chapter 23}:  Recall that the definition of magnetic flux through an area $A$ by a B-field $B$ is $\phi = \vec{B} \cdot \vec{A} = BA\cos\theta$.  Recall also Faraday's law $emf = -N \Delta \phi/\Delta t$, and that $\mu_0 = 4\pi \times 10^{-7}$ T A$^{-1}$ m.
\begin{enumerate}
\item Suppose a loop of wire has an area $A$ and is in the x-y plane ($\vec{A} = A \hat{k}$).  A magnetic field $\vec{B}$ creates a flux $\phi$ through the loop of wire.  Create a graph of $\phi$ versus $\theta$, where $\theta$ is the angle between the field and the loop.   Indicate on the graph the angles at which the flux is maximal and where it is zero.\\ \vspace{3cm}
\item If $|B| = 10^{-2}$ T, and $A$ is 4 cm$^2$, what is the maximum $\phi$? \\ \vspace{3cm}
\item If the magnetic field reduces to 0.0 T in 4 ms, what emf is induced in the wire? (Assume $\theta = 0$). \\ \vspace{3cm}
\item Suppose a solenoid shaped wire has $N=1000$ coils, and is $L = 4$ cm long.  Suppose we feed a current $I = 1.0$ amp through the solenoid.  Amp\`{e}re's Law tells us that the magnetic field in the solenoid is $B = \mu_0 (N/L) I$.  What is the magnetic field in this situation?
\end{enumerate}
\end{enumerate}
\end{document}