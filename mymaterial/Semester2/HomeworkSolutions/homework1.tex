\documentclass{beamer}
\usetheme{metropolis}
\usepackage{graphicx}
\usepackage{subfig}
\usepackage{tcolorbox}
\title{Algebra-Based Physics-2: Electricity, Magnetism, and Modern Physics (PHYS135B-01): Homework 1 Solutions}
\author{Jordan Hanson}
\institute{Whittier College Department of Physics and Astronomy}

\begin{document}
\maketitle

\section{Chapter 18}

\begin{frame}{Chapter 18: 7,12,15,30,31.}
\small
\textbf{Exercise 7}: Let the number of protons be $N_p$ and the number of electrons be $N_e$, and $q = 1.6\times 10^{-19}$ C.  For an un-charged object, the net charge must be
\begin{equation}
Q_{Net,i} = qN_{p} - qN_{e,i} = 0
\end{equation}
If there is a net charge $Q_{Net,f}$, then electrons have been removed and $N_{e,f} < N_{e,i}$.  Thus for a charged object,
\begin{equation}
N_{e,f} = N_p - \frac{Q_{Net,f}}{q}
\end{equation}
Dividing both sides by the original number of electrons yields the fraction of remaining electrons, $r$:
\begin{equation}
r = \frac{N_{e,f}}{N_{e,i}} = \frac{N_p}{N_{e,i}} - \frac{Q_{Net,f}}{qN_{e,i}}
\end{equation}
\end{frame}

\begin{frame}{Chapter 18: 7,12,15,30,31.}
\small
\begin{equation}
r = \frac{N_{e,f}}{N_{e,i}} = \frac{N_p}{N_{e,i}} - \frac{Q_{Net,f}}{qN_{e,i}}
\end{equation}
Recall that $N_{e,i} = N_p$.  Substituting, we have
\begin{equation}
r = 1 - \frac{Q_{Net,f}}{qN_p}
\end{equation}
The exercise is asking for the fraction \textit{removed}, which is $1-r$:
\begin{equation}
1-r = \frac{Q_{Net,f}}{qN_p}
\end{equation}
Using $50.0$ g divided by $63.5$ g/mol, and multiplying by 29 gives the total number of protons, $N_p \approx 1.5 \times 10^{25}$.  Thus,
\begin{equation}
1-r = \frac{2\times 10^{-6}~C}{1.6\times 10^{-19} ~C~ 1.5 \times 10^{25}} \approx 10^{-12}
\end{equation}
One electron in a trillion is removed.
\end{frame}

\begin{frame}{Chapter 18: 7,12,15,30,31.}
\small
\textbf{Exercise 12}: This is a scaling problem.  If the force is 5.00 N, and goes as $1/r^2$, then the force must decrease to $1/(3)^2 = 1/9$ of the original value, or $5/9$ N.
\end{frame}

\begin{frame}{Chapter 18: 7,12,15,30,31.}
\small
\textbf{Exercise 15}: Apply the Coulomb force:
\begin{equation}
F = \frac{1}{4\pi\epsilon_0} \frac{q^2}{r^2} = 9 \times 10^9 \times 1^2 / 10^6 \approx 10^4 ~ N
\end{equation}
This is ten times the weight of a grown man, so the electrostatic force is a very strong one, if we are dealing with Coulombs of charge.
\end{frame}

\begin{frame}{Chapter 18: 7,12,15,30,31.}
\small
\textbf{Exercise 30}: a) Use the definition of the electric field $E$ for a point charge $Q$ at a distance $r$:
\begin{align}
E &= \frac{kQ}{r^2} \\
Q &= \frac{r^2 E}{k} \approx 0.07 \mu C
\end{align}
b) This is a scaling problem.  The distance is increasing by a factor of 40, so the field drops by a factor of 1600, to 6.25 N/C.
\end{frame}

\begin{frame}{Chapter 18: 7,12,15,30,31.}
\small
\textbf{Exercise 31}: Set Newton's second law equal to the electric force on a test charge:
\begin{align}
F_C &= F_{Net} = m_p a = q E \\
a &= \frac{qE}{m_p} = \frac{1.6\times 10^{-19}\times 5\times 10^6}{1.67 \times 10^{-27}} ~ m/s^2 \\
a &\approx 5 \times 10^{27+6-19} = 5 \times 10^{14} ~ m/s^2
\end{align}
With this acceleration the proton would reach light speed in about 600 nanoseconds.
\end{frame}

\section{Chapter 19}

\begin{frame}{Chapter 19: 13,15,22,37,47,60.}
\small
\textbf{Exercise 13}: The units of $V/m$ and $N/C$ are equivalent, because energy is charge times voltage:
\begin{equation}
U = qV
\end{equation}
This means a Volt is 1 Joule per Coulomb, and a Joule is a Newton-meter.  Thus, a Volt is a (Newton-meter) per Coulomb.  Dividing both sides by a meter, we see that a Volt per meter is a (Newton-meter) per (Coulomb-meter), or a Newton per Coulomb.
\end{frame}

\begin{frame}{Chapter 19: 13,15,22,37,47,60.}
\small
\textbf{Exercise 15}: a) $V = Ed$, so multiply the two numbers to find 3 kV.  b) Given that the potential rises from 0 to 3 kV in 4 cm, we can think of this as our charged plates in the PHeT simulation.  The voltage increases linearly from 0 V, and so the slope is the electric field.  Thus, $V = 7.5 \times 10^4 \times 10^{-2}$ V = 750 V.
\end{frame}

\begin{frame}{Chapter 19: 13,15,22,37,47,60.}
\small
\textbf{Exercise 22}: If the energy gained is 32 keV, this would mean the voltage is 32 kV for an electron.  However, the ion has a net charge of two times an electron charge (positive or negative).  Thus, the voltage must be 16 kV.  The plates are separated by $V = Ed$, so $E = V/d$ and the result is 8 kV/cm, or $V = 8 \times 10^5$ V/m.
\end{frame}

\begin{frame}{Chapter 19: 13,15,22,37,47,60.}
\small
\textbf{Exercise 37}: Full points for any diagram with the following properties: 1) symmetry about a horizontal line drawn through both points where the charges are located, 2) equipotential lines are orthogonal to electric field lines, and 3) the fact that $V \propto r^{-1}$, meaning that all points on an equipotential line must have an equal amount of $1/r_1 + 1/r_2$, where $r_1$ and $r_2$ are the distances to the two charges.
\end{frame}

\begin{frame}{Chapter 19: 13,15,22,37,47,60.}
\small
\textbf{Exercise 47}: Definition of capacitance: $Q = CV = 8 \times 5.5 \times 10^{-12}$ C, so 44 pC.
\end{frame}

\begin{frame}{Chapter 19: 13,15,22,37,47,60.}
\small
\textbf{Exercise 60}: The two capacitors on the left side add together like $1/C = 1/C_1 + 1/C_2$, so $C = (C_1)(C_2)/(C_1+C_2)$.  This new capacitance, $C$, adds with the capacitor on the right, $C_3$ like $C+C_3$.  Thus, the total is $C_{tot} = (C_1)(C_2)/(C_1+C_2) + C_3 \approx 5.4 \mu F$.
\end{frame}

\end{document}
