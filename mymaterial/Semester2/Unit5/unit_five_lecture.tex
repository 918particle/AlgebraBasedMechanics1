\documentclass{beamer}
\usetheme{metropolis}
\usepackage{graphicx}
\usepackage{subfig}
\usepackage{tcolorbox}
\title{Algebra-Based Physics-2: Electricity, Magnetism, and Modern Physics (PHYS135B-01): Unit 5}
\author{Jordan Hanson}
\institute{Whittier College Department of Physics and Astronomy}

\begin{document}
\maketitle

\section{Unit 4 Review}

\begin{frame}{Unit 4 Summary}
\textbf{Reading: Chapter 22}
\begin{enumerate}
\item Magnets and magnetic fields
\item Force on a moving charge in a magnetic field
\item \textbf{The Hall effect}
\item Magnetic forces on conductors
\item \alert{Amp\`{e}re's Law}
\end{enumerate}
\end{frame}

\section{Summary}

\begin{frame}{Summary}
\begin{enumerate}
\item Maxwell's Equations and Electromagnetic Waves
\item Electromagnetic wave production
\item The Electromagnetic spectrum
\item Waves and Energy
\end{enumerate}
\end{frame}

\section{JITT}

\begin{frame}{JITT}
\textbf{The direction of the electric field shown in each part of Figure 24.5 is that produced by the charge distribution in the wire.  Justify the direction shown in each part, using the Coulomb force law and the definition of $\vec{E} = \vec{F}/q$, where $q$ is positive.}
\end{frame}

\begin{frame}{JITT}
In A, the positive charge is at the bottom and the negative charge is at the top (field has to be upward). In B, the distance between the two charges is zero (no question of direction because the distance becomes zero with the electric field). In C, the distance between the charges is the same as it was in part A but positive charge placed in the downward direction will experience a force that will push it downward. In D, the upward direction is justified (similar to A). 
\end{frame}

\begin{frame}{JITT}
\textbf{Is the direction of the magnetic field shown in Figure 24.6 (a) consistent with the right-hand rule for current (RHR-2) in the
direction shown in the figure?}
\end{frame}

\begin{frame}{JITT}
In (a), using the Right Hand Rule, our thumb points in the course of current. In this case current is upward, so the direction of B should be counter clockwise, which is the direction of our fingers. This appears to be consistent. 
\end{frame}

\begin{frame}{JITT}
Yes, the direction of the magnetic field shown in figure 24.6 is consistent with the right hand rule because a straight up and down wire creates a circular magnetic field around the wire to the right. 
\end{frame}

\begin{frame}{JITT}
\textbf{Why is the direction of the current shown in each part of Figure 24.6 opposite to the electric field produced by the wire’s
charge separation?}
\end{frame}

\begin{frame}{JITT}
The direction of the current is opposite to the electric field because the magnetic field varies with current and propagates away from the antenna at the speed of light, and they are perpendicular to one another and to the direction of propagation.
\end{frame}

\begin{frame}{JITT}
Electric field direction runs in direction of positive test charge, therefore the negative current is in the opposite direction.
\end{frame}

\begin{frame}{JITT}
The current is going upwards at that moment, so it is causing the pos. charge to flow in its own direction, while the neg. charge is going to opposite direction.
\end{frame}

\begin{frame}{JITT}
\textbf{In which situation shown in Figure 24.24 will the electromagnetic wave be more successful in inducing a current in the wire? Explain.}
\end{frame}

\begin{frame}{JITT}
B because the electric field ( E ) shown surrounding the wire is produced by the charge distribution on the wire and the The stronger the E -field created by a separation of charge, the greater the current and, hence, the greater the B -field created.
\end{frame}

\begin{frame}{JITT}
The wire is at rest, so the magnetic field will not induce any current in the wire. Option A will be more successful because we know that an electromagnetic wave consists of electric and magnetic charges. Option A will be more successful due to the directions of the wave which promote waves success. 
\end{frame}

\begin{frame}{JITT}
A because they are perpendicular to one another and create a transverse wave.
\end{frame}

\section{Maxwell's Equations and Electromagnetic Waves}

\begin{frame}{Maxwell's Equations and Electromagnetic Waves}
What is the value $1/\sqrt{\mu_0 \epsilon_0}$?  Do you recognize this value?
\begin{itemize}
\item A: $3 \times 10^{7}$ m/s
\item B: $3 \times 10^{6}$ m/s
\item C: $3 \times 10^{5}$ m/s
\item D: $3 \times 10^{8}$ m/s
\end{itemize}
\end{frame}

\begin{frame}{Maxwell's Equations and Electromagnetic Waves}
Examine the following function:
\begin{equation}
\vec{E}(t) = E_0 \sin\left(k x - \omega t\right) \hat{k}
\end{equation}
\textbf{Work some examples with this function.}  What is the meaning of $k$ and $\omega$?
\end{frame}

\section{Conclusion}

\section{Answers}

\begin{frame}{Answers}
\tiny
\begin{columns}[T]
\begin{column}{0.5\textwidth}
\begin{itemize}
\item $3 \times 10^{8}$ m/s
\end{itemize}
\end{column}
\begin{column}{0.5\textwidth}
\begin{itemize}
\item 
\end{itemize}
\end{column}
\end{columns}
\end{frame}

\end{document}
