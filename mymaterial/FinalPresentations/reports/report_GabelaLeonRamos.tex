\title{Project Overview and Assessment: Jessica Gabela, Nancy Leon, Rubi Ramos}
\author{Dr. Jordan Hanson - Whittier College Dept. of Physics and Astronomy}
\date{\today}
\documentclass[10pt]{article}
\usepackage[a4paper, total={18cm, 27cm}]{geometry}
\usepackage{outlines}
\usepackage[sfdefault]{FiraSans}
\usepackage{hyperref}

\begin{document}
\maketitle

\begin{abstract}
The experiment was a kind of test of the equivalence principle, and the introduction included an explanation of the principle.  The hypothesis, however, is awkwardly worded: the objects do not have constant velocity at any point in the experiment.  It is true, however, that the objects should have the same \textit{acceleration}, and arrive at the ground at the same \textit{time}.  The slide demonstrating predictions is very confusing.  There should be one prediction per object per height.  The method for measuring fall time was clear, aided by the diagram.  The data and calculations do not use proper notation ($v = v_i + gt$, not $v = d t$).  It is not made clear how the raw measurements of tens of milliseconds are put into a calculation that yields the expected 300-400 ms for the drop time.  If the measurements were of speed, in the graphs (even though the axes have units of milliseconds), it should have been a red flag that the different objects were producing different values.  Despite these flaws, the hypothesis was that different objects would fall at the same rate, and somehow the numbers show that at the end, with some statistical fluctuations.
\end{abstract}

\textit{Score} - \textbf{7 of 10 points.}

\textit{Project Assessment}
\begin{outline}[enumerate]
\1 Introduction of Concepts, Hypothesis
\2 There was a qualitative description of the equivalence principle and what it meant for the falling objects.
\1 Explanation of the Experiment, with Diagram or Picture
\2 The diagram expressed the idea of the experiment.
\1 Presentation of Data and Systematics
\2 I could not follow the calculations, which seemed wrong in that the times were of the order a few times $10$ ms, when they should have been hundreds of milliseconds.
\1 Conclusion
\2 The conclusion restated the equivalance principle, and the numbers on one slide seemed to support that claim.  However, the numbers on the other slides seemed wrong by a factor of ten.  This suggests dividing by $g$.  However, dividing by $g$ to obtain time is appropriate if the numerator is a velocity.  Velocities were not measured; times were measured.
\end{outline}
\end{document}
