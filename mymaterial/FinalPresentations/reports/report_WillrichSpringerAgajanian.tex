\title{Project Overview and Assessment: Robert Willrich, Nicholas Springer, Richard Agajanian}
\author{Dr. Jordan Hanson - Whittier College Dept. of Physics and Astronomy}
\date{\today}
\documentclass[10pt]{article}
\usepackage[a4paper, total={18cm, 27cm}]{geometry}
\usepackage{outlines}
\usepackage[sfdefault]{FiraSans}
\usepackage{hyperref}

\begin{document}
\maketitle

\begin{abstract}
This was a carefully organized experiment to obtain the speed of sound, using one receiver and two sources (drums).  The hypothesis was stated quantitatively, along with the relevant concept of average velocity.  Next, the setup was described with diagrams and pictures.  At first collection, the much of the data was not useable.  However, in the procedural changes section, the criterion for deciding which data to keep and which data to discard were made \textit{explicit}.  This is a hallmark of good science: the experimenter is leaving it to the reader to decide whether the criteria are reasonable.  The subset of data was then analyzed appropriately, and the results are two standard deviations or 7.6 percent from the accepted value.
\end{abstract}

\textit{Score} - \textbf{10 of 10 points.}

\textit{Project Assessment}
\begin{outline}[enumerate]
\1 Introduction of Concepts, Hypothesis
\2 This was quantitative and precise.
\1 Explanation of the Experiment, with Diagram or Picture
\2 This was made explicit with diagrams and pictures.
\1 Presentation of Data and Systematics
\2 Some of the data had to be discarded, and the remaining data was analyzed correctly.
\1 Conclusion
\2 The measured speed was found to be within reasonable agreement with the accepted value of the speed of sound.
\end{outline}
\end{document}
