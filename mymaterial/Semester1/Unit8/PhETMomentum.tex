\documentclass{article}
\usepackage{graphicx}
\usepackage[margin=1.5cm]{geometry}
\usepackage{amsmath}
\usepackage{url}

\begin{document}

\title{PhET Activity: Conservation of Momentum}
\author{Prof. Jordan C. Hanson}

\maketitle

\section{Introduction}

\begin{enumerate}
\item Using the PhET activity \textit{Collision Lab}: \url{https://phet.colorado.edu/en/simulations/collision-lab}, we will understand how the conservation of momentum works.  Go to the URL above, and load the Explore 1D tab.
\item Notice that there is a clock at the lower left displaying the time in seconds after the play button is pressed, and that the scale length in the top left is 0.5 meters.  Use these tools to measure the velocity of the balls.
\item Set the mass of each ball to 0.5 kg, and set the velocity of the right-hand ball to zero by clicking and dragging the velcotiy vector.
\item Make the velocity of the left-hand ball non-zero, and click play.  Write down the (a) initial velocity of the left-hand ball, (b) the initial velocity of the right-hand ball, (c) the final velocity of the left-hand ball, and (d) the final velocity of the right-hand ball. \\ \vspace{2cm}
\end{enumerate}

\section{Measurements and Graph}

\begin{enumerate}
\item Using the procedure above, vary the mass of the right-hand ball.  Create a graph below of the final velocity of the right-hand ball in m/s versus mass in kg. \\ \vspace{4cm}
\item Use the \textit{conservation of momentum} to explain the data.  (a) For the case of equal masses, use conservation of momentum to show that the speed of the right-hand ball should be equal to the initial velocity of the approaching one.  (c) For the case in which the right-hand ball is much larger than the aproaching one, show that the intial and final speed of the left-hand ball should be equal.
\end{enumerate}

\end{document}