\documentclass{article}
\usepackage{graphicx}
\usepackage[margin=1.5cm]{geometry}
\usepackage{amsmath}

\begin{document}

\title{Experiment with Drag Forces.}
\author{Prof. Jordan C. Hanson}

\maketitle

\section{Introduction}

In Section 5,2 of the book, we find the following \textbf{Take-Home Experiment}:

\begin{quote}
This interesting activity examines the effect of weight upon terminal velocity. Gather together some nested coffee filters. Leaving them in their original shape, measure the time it takes for one, two, three, four, and five nested filters to fall to the floor from the same height (roughly 2 m).  Note that, due to the way the filters are nested, drag is constant and only mass varies. They obtain terminal velocity quite quickly, so find this velocity as a function of mass.  Plot the terminal velocity, $V$, versus mass. Also plot $v^2$ versus mass.  Which of these relationships is more linear? What can you conclude from these graphs?
\end{quote}

\section{Drag Forces and Setup}

Recall that the drag force $F_D$ is related to the density of air, $\rho$, the drag coefficient $C$, the cross-sectional area $A$, and the velocity $v$ by

\begin{equation}
F_D = \frac{1}{2}C\rho A v^2
\end{equation}

Using a timer on a smartphone or desktop, measure the time required for a coffee filter to float to the ground.  Repeat with two, three, four, and five filters stacked together.  The instructor will give the mass of one coffee filter.

\section{Analysis}

Using a spreadsheet program, create two plots: (a) velocity versus mass, and (b) velocity squared versus mass.  Given what you know about terminal velocity, what should be true of these graphs?  Sketch your results for both graphs below, and don't forget to label the axes and tick marks.

\end{document}
