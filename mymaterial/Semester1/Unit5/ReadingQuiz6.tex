\documentclass{article}
\usepackage{graphicx}
\usepackage[margin=1.5cm]{geometry}
\usepackage{amsmath}

\begin{document}

\title{Warm Up Exercises: Centripetal Force}
\author{Prof. Jordan C. Hanson}

\maketitle

\section{Memory Bank}

\begin{itemize}
\item Let $\Delta\theta$ be the \textit{angular displacement}, $\Delta\theta = \theta_f - \theta_i$.  Let the time duration be $\Delta t = t_f - t_i$.  Let the angular velocity be $\omega = \Delta \theta / \Delta t$.  If $t_i = 0$ seconds and $\theta_i = 0$ degrees, then we can use $\omega$ to write $\theta = \omega t$ (just like $x = v t$).  If an object is rotating with angular velocity $\omega$ on a circle of radius $r$, then the position versus time is:
\begin{equation}
\vec{r}(t) = r\cos(\omega t)\hat{i} + r\sin(\omega t)\hat{j} \label{eq:1}
\end{equation}
\item $v = r\omega$ ... Radial velocity.
\item $a_{\rm C} = v^2/r$ ... Centripetal acceleration.
\item $a_{\rm C} = r \omega^2$ ... Centripetal acceleration.
\item $\vec{F}_{\rm C} = m a_{\rm C}$ ... Centripetal force.
\end{itemize}

\section{Centripetal Force}
\begin{enumerate}
\item Suppose a vehicle traveling at 40 m/s makes a turn with radius $r = 200$ m.  The coefficient of static friction between the tires and the cold, wet road is reduced to 0.5.  Is there enough friction to keep the vehicle in the turn?  Assume the road is flat. \\ \vspace{2cm}
\item Now assume the road is \textit{banked} at an angle of 10 degrees.  The road slopes towards the center of the circle.  Draw a free body diagram that includes the normal force and centripetal force, but neglects friction. \\ \vspace{2cm}
\item Show that the speed $v$, turn radius $r$, and gravitional constant $g$ are related by
\begin{equation}
\tan\theta = \frac{v^2}{r g}
\end{equation}
For the given bank angle and turn radius, what is the maximum speed of the vehicle?
\end{enumerate}

\end{document}
