\documentclass{article}
\usepackage{graphicx}
\usepackage[margin=1.5cm]{geometry}
\usepackage{amsmath}

\begin{document}

\title{Thursday Reading Assessment: Unit 2, Kinematics}
\author{Prof. Jordan C. Hanson}

\maketitle

\section{Memory Bank}

\begin{enumerate}
\item Assume that acceleration is constant: $a = 3.0$ (m/s$^2$), and that $\Delta x = x_f - x_i$
\item $v_f(t) = gt + v_{i}$ (m/s)
\item $x(t) = \frac{1}{2}at^2 + v_{i} t + x_{i}$ (m)
\item $v_f^2 = v_i^2 + 2a\Delta x$ (m/s)$^2$.
\end{enumerate}

\section{Chapter 2 - Kinematics}

\begin{enumerate}
\item Solve Equation 2 for $t$, and just take the magnitude of the vectors. $t=?$ \\ \vspace{2cm}
\item Insert $t$ into Equation 3, and solve for $v_f^2$.  What relationship do you find? \\ \vspace{3cm}
\item \textbf{Example from KNS}: Imagine a sprinter preparing for a race.  He is starting \textit{from rest}, and the race begins at $t=0$.  He accelerates \textit{up to 10 m/s} at a \textit{rate} of 3 m/s$^2$.  \textbf{How far} has he traveled?  Use the fact that we have proved Equation 4. \\ \vspace{2cm}
\item If he travels at 10 m/s for another 20 seconds, what additional distance does he cover?
\end{enumerate}

\end{document}
