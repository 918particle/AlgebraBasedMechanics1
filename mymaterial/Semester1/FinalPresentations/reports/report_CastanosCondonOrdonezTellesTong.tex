\title{Project Overview and Assessment: Joseph Casta\~{n}os, Matthew Condon, Michelle Ordo\~{n}ez, Taylor Telles, and Sara Tong}
\author{Dr. Jordan Hanson - Whittier College Dept. of Physics and Astronomy}
\date{\today}
\documentclass[10pt]{article}
\usepackage[a4paper, total={18cm, 27cm}]{geometry}
\usepackage{outlines}
\usepackage[sfdefault]{FiraSans}
\usepackage{hyperref}

\begin{document}
\maketitle

\begin{abstract}
This was an example of measuring acceleration versus force, and showing that there is a linear relationship between them via Newton's 2nd Law.  The setup was interesting, and the presenters provided a drawing and videos of data collection.  The hypothesis was simple enough, but could have been made more quantitative.  Although it was not initially clear what times and distances were being recorded and used in the velocity measurements, this was made clear during the 	question phase.  This experiment could have been improved by predicting the measured accelerations with the result of a free-body diagram.
\end{abstract}

\textit{Score} - \textbf{9 of 10 points.}

\textit{Project Assessment}
\begin{outline}[enumerate]
\1 Introduction of Concepts, Hypothesis
\2 There was a qualitative hypothesis that was stated clearly.  Using a free-body diagram would have aided understanding.
\1 Explanation of the Experiment, with Diagram or Picture
\2 The explanation of the setup with words, diagrams, and pictures was clear.
\1 Presentation of Data and Systematics
\2 The data was clearly presented, and it was explained in words that distances were chosen to minimize the systematic error of non-linear acceleration after the pulling mass hit the floor.
\1 Conclusion
\2 The conclusion was qualitative but in agreement with the data.
\end{outline}
\end{document}
