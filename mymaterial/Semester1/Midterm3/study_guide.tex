\title{Study Guide for Midterm 3}
\author{Dr. Jordan Hanson - Whittier College Dept. of Physics and Astronomy}
\date{\today}
\documentclass[10pt]{article}
\usepackage[margin=1.5cm]{geometry}
\usepackage{outlines}
\usepackage{graphicx}
\usepackage{amsmath}

\begin{document}
\maketitle

\section{Memory Bank}

\begin{itemize}
\item Unit conversions: 1 km = 1000 m, 1 m = 100 cm, 1 hr = 3600 s, 1 year = $\pi \times 10^7$ s, 1 g/cm$^3$ = 1000 kg/m$^3$.
\item $\vec{x} = a \hat{i} + b\hat{j}$ ... Component form of a two-dimensional vector.
\item $|\vec{x}| = \sqrt{a^2+b^2}$ ... Pythagorean theorem for obtaining vector magnitude.
\item $\theta = \tan^{-1}(b/a)$ ... Obtaining the angle between vector and x-axis.
\item $x(t) = x_i + v t$ ... Velocity is the slope of position versus time.
\item $x(t) = \frac{1}{2} a t^2 + v_i t + x_i$ ... With constant acceleration, position is quadratic.  If $a=0$ this becomes the prior function.
\item $v(t) = v_i + a t$ ... With constant acceleration, acceleration is the slope of velocity.
\item $v^2 = v_i^2 + 2 a \Delta x$ ... The kinematic equation without time, assuming constant acceleration.
\item $\vec{F}_{net} = 0$ ... Newton's First Law, an object with no net force stays at constant velocity, or zero velocity.
\item $\vec{F}_{net} = m\vec{a}$ ... Newton's Second Law.
\item $\vec{F}_{AB} = -\vec{F}_{BA}$ ... Newton's Third Law.
\item $f = \mu N$, $F_D = \frac{1}{2}C\rho A v^2$, $F_D = 6\pi r \eta v$ ... friction, drag in air, drag in viscous fluids.
\item $stress = Y \times strain$, or $F/A = Y (\Delta x / L)$ ... Young's Modulus and elasticity.
\item $s = r \theta$ ... Definition of a \textit{radian}, with arc length $s$ and angle $\theta$.
\item $v = r\omega$, $a = r\alpha$ ... Angular velocity, angular acceleration.
\item $a_C = v^2/r = r\omega^2$ ... Centripetal acceleration.
\item $F_C = m a_C = mv^2/r = mr\omega^2$ ... Centripetal force.
\item $\vec{F}_G = G m_1 m_2/r^2 ~~ \hat{r}$ ... Newton's Law of Gravity.
\item $W = \vec{F} \cdot \vec{d}$ ... Definition of Work, energy.
\item $KE = \frac{1}{2}mv^2$ ... Definition of kinetic energy.
\item $W = \Delta KE$ ... Work-Energy theorem.
\item $U = mgh$ ... Gravitational potential energy.
\item $U = \frac{1}{2}kx^2$ ... Spring potential energy.
\item $P = W/t$ ... Power is work divided by time.
\item 1 kilocalorie, or kcal, is 4184 Joules.
\item $\vec{p} = m\vec{v}$ ... Definition of momentum.
\item $\vec{p}_{i,tot} = \vec{p}_{f,tot}$ ... Momentum conservation for $\vec{F}_{net} = 0$.  Also, $\vec{F}_{net} = \Delta \vec{p} / \Delta t$.
\item $\vec{\tau} = \vec{r}\times \vec{F}$, $|\vec{\tau}| = I \alpha$, with $I = N m r^2$.  $N$ depends on the shape.
\item $\vec{L} = \vec{r} \times \vec{p}$, and $L = I\omega$, so $\Delta L/\Delta t = \tau$.
\end{itemize}

\section{Chapter 7: Work, Energy, and Energy Resources}

\begin{enumerate}
\item A shopper pushes a grocery cart 20.0 m at constant speed on level ground, against a 35.0 N frictional force. He pushes in a direction  25 degrees below the horizontal. (a) What is the work done on the cart by friction? (b) What is the work done on the cart by the gravitational force? (c) What is the work done on the cart by the shopper? (d) Find the force the shopper exerts, using energy considerations. (e) What is the total work done on the cart? \\ \vspace{1.5cm}
\item  Boxing gloves are padded to lessen the force of a blow. (a) Calculate the force exerted by a boxing glove on an opponent’s face, if the glove and face compress 7.50 cm during a blow in which the 7.00-kg arm and glove are brought to rest from an initial speed of 10.0 m/s. (b) Calculate the force exerted by an identical blow in the days when no gloves were used and the knuckles and face would compress only 2.00 cm. (c) Discuss the magnitude of the force with glove on. Does it seem high enough to cause damage even though it is lower than the force with no glove? \\ \vspace{2cm}
\item A large household air conditioner may consume 15.0 kW of power. What is the cost of operating this air conditioner 3.00 h per day for 30.0 days if the cost of electricity is 0.1 dollars per kW h? \\ \vspace{1cm}
\end{enumerate}

\section{Chapter 8: Linear Momentum and Collisions}

\begin{enumerate}
\item Train cars are coupled together by being bumped into one another. Suppose two loaded train cars are moving toward one another, the first having a mass of 150,000 kg and a velocity of 0.300 m/s, and the second having a mass of 110,000 kg and a velocity of −0.120 m/s. (The minus indicates direction of motion.) What is their final velocity? \\ \vspace{1.5cm}
\item \textbf{Elastic interaction:} Two piloted satellites approach one another at a relative speed of 0.250 m/s, intending to dock. The first has a mass of $4.00 \times 10^3$ kg, and the second a mass of $7.50 \times 10^3$ kg. If the two satellites collide elastically rather than dock, what is their final relative velocity? \\ \vspace{1.5cm}
\item \textbf{In-elastic interaction:} Two piloted satellites approaching one another, at a relative speed of 0.250 m/s, intending to dock. The first has a mass of $4.00 \times 10^3$ kg, and the second a mass of $7.50 \times 10^3$ kg. (a) Calculate the final velocity (after docking) by using the frame of reference in which the first satellite was originally at rest. (b) What is the loss of kinetic energy in this inelastic collision? (c) Repeat both parts by using the frame of reference in which the second satellite was originally at rest. Explain why the change in velocity is different in the two frames, whereas the change in kinetic energy is the same in both. \\ \vspace{2cm}
\end{enumerate}

\section{Chapter 10: Rotational Motion and Angular Momentum}

\begin{enumerate}
\item Suppose we catch a fish on a fishing pole.  (a) Draw a diagram including the pole (lever arm), and the weight vector of the fish.  (b) Is the torque going into the page, or out of it?  (c) If the fish has a mass of 0.7 kg, and the pole is 1.4 m long, what is the magnitude of the torque? \\ \vspace{1.5cm}
\item  A soccer player extends her lower leg in a kicking motion by exerting a force with the muscle above the knee in the front of her leg. She produces an angular acceleration of  30.00 rad/s$^2$ and her lower leg has a moment of inertia of 0.750 kg m$^2$. What is the force exerted by the muscle if its effective perpendicular lever arm is 1.90 cm? \\ \vspace{1cm}
\item Suppose you exert a force of 180 N tangential to a 0.280 m radius, 75.0-kg grindstone (a solid disk: $I = \frac{1}{2}MR^2$).  (a) What torque is exerted? (b) What is the angular acceleration assuming negligible opposing friction? (c) What is the angular acceleration if there is an opposing frictional force of 20.0 N exerted 1.50 cm from the axis? \\ \vspace{2cm}
\end{enumerate}

\end{document}