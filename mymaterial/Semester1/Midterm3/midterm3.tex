\title{Midterm 3 for Algebra-Based Physics }
\author{Dr. Jordan Hanson - Whittier College Dept. of Physics and Astronomy}
\date{\today}
\documentclass[10pt]{article}
\usepackage[margin=1.5cm]{geometry}
\usepackage{outlines}
\usepackage{graphicx}
\usepackage{amsmath}

\begin{document}
\maketitle

\section{Memory Bank}

\begin{itemize}
\item Unit conversions: 1 km = 1000 m, 1 m = 100 cm, 1 hr = 3600 s, 1 year = $\pi \times 10^7$ s, 1 g/cm$^3$ = 1000 kg/m$^3$.
\item $\vec{x} = a \hat{i} + b\hat{j}$ ... Component form of a two-dimensional vector.
\item $|\vec{x}| = \sqrt{a^2+b^2}$ ... Pythagorean theorem for obtaining vector magnitude.
\item $\theta = \tan^{-1}(b/a)$ ... Obtaining the angle between vector and x-axis.
\item $x(t) = x_i + v t$ ... Velocity is the slope of position versus time.
\item $x(t) = \frac{1}{2} a t^2 + v_i t + x_i$ ... With constant acceleration, position is quadratic.  If $a=0$ this becomes the prior function.
\item $v(t) = v_i + a t$ ... With constant acceleration, acceleration is the slope of velocity.
\item $v^2 = v_i^2 + 2 a \Delta x$ ... The kinematic equation without time, assuming constant acceleration.
\item $\vec{F}_{net} = 0$ ... Newton's First Law, an object with no net force stays at constant velocity, or zero velocity.
\item $\vec{F}_{net} = m\vec{a}$ ... Newton's Second Law.
\item $\vec{F}_{AB} = -\vec{F}_{BA}$ ... Newton's Third Law.
\item $f = \mu N$, $F_D = \frac{1}{2}C\rho A v^2$, $F_D = 6\pi r \eta v$ ... friction, drag in air, drag in viscous fluids.
\item $stress = Y \times strain$, or $F/A = Y (\Delta x / L)$ ... Young's Modulus and elasticity.
\item $s = r \theta$ ... Definition of a \textit{radian}, with arc length $s$ and angle $\theta$.
\item $v = r\omega$, $a = r\alpha$ ... Angular velocity, angular acceleration.
\item $a_C = v^2/r = r\omega^2$ ... Centripetal acceleration.
\item $F_C = m a_C = mv^2/r = mr\omega^2$ ... Centripetal force.
\item $\vec{F}_G = G m_1 m_2/r^2 ~~ \hat{r}$ ... Newton's Law of Gravity.
\item $W = \vec{F} \cdot \vec{d}$ ... Definition of Work, energy.
\item $KE = \frac{1}{2}mv^2$ ... Definition of kinetic energy.
\item $W = \Delta KE$ ... Work-Energy theorem.
\item $U = mgh$ ... Gravitational potential energy.
\item $U = \frac{1}{2}kx^2$ ... Spring potential energy.
\item $P = W/t$ ... Power is work divided by time.
\item 1 kilocalorie, or kcal, is 4184 Joules.
\item $\vec{p} = m\vec{v}$ ... Definition of momentum.
\item $\vec{p}_{i,tot} = \vec{p}_{f,tot}$ ... Momentum conservation for $\vec{F}_{net} = 0$.  Also, $\vec{F}_{net} = \Delta \vec{p} / \Delta t$.
\item $\vec{\tau} = \vec{r}\times \vec{F}$, $|\vec{\tau}| = I \alpha$, with $I = N m r^2$.  $N$ depends on the shape.
\item $\vec{L} = \vec{r} \times \vec{p}$, and $L = I\omega$, so $\Delta L/\Delta t = \tau$.
\end{itemize}

\section{Chapter 7: Work, Energy, and Energy Resources}

\begin{enumerate}
\item Suppose you push a large piece of furniture with mass 70.0 kg across a floor with frictional coefficient $\mu_{\rm k} = 0.05$ a distance of 8.0 m at constant speed.  a) What is the force of friction?  b) What is the work done by friction on the object?  c) Suppose you are pushing at a 10 degree angle with respect to the horizontal What work do you perform? (d) With what force are you pushing? \\ \vspace{3cm}
\item  Vehicles are designed to cruch linearly to protect the passengers inside. (a) Use the work-energy theorem to calculate the force exerted on a 1500 kg vehicle, if the vehicle is compressed by 1.25 meters when it hits a solid object, decelerating from 9.0 m/s. (b) Compare this to the force exerted on the vehicle in the same situation, but the vehicle is made of older materials that only compress 0.25 meters. (Think about this when people tell you ``bigger cars are safer.''  Depends on the materials.).  \\ \vspace{3cm}
\item A small flatscreen TV can consume 100 Watts of power.  What is the cost of operating this television for 1 hours per day, 6 days per week, for one year?  (Assume \$0.15 per kW h). \\ \vspace{2cm}
\end{enumerate}

\section{Chapter 8: Linear Momentum and Collisions}

\begin{enumerate}
\item A proton with mass $m$ collides with a helium ion of mass $4m$ and temporarily forms a new particle of mass $5m$.  The velocity of the proton is 4\% of the speed of light, to the right, and the velocity of the helium is 2\% of the speed of light, to the left.  What is the velocity of the final particle, as a fraction of the speed of light? \\ \vspace{1.5cm}
\item Two orbiting satellites are attached, and stationary. The first has a mass of $6.00 \times 10^3$ kg, and the second a mass of $3 \times 10^3$ kg. (a) A mechanism is triggered that does 0.5 kJ of work on each satellite, and they fly apart. (a) What is the final velocity of the first craft?  (b) What is the final velocity of the second craft? (c) Is momentum conserved? (d) Is kinetic energy conserved?  Why or why not? \\ \vspace{3cm}
\item \textbf{Conceptual Question}: Recall the laboratory activity we did with the carts on the frictionless rail.  Suppose you have two carts \textit{of equal mass} heading toward each other at \textit{equal velocity}, and each has a magnet loaded on the side facing the other cart.  The magnets repel each other.  When the carts approach, the magnets prevent them from touching and they spring backwards.  Which of the following should be true of the carts' velocities if the collision is elastic?
\begin{itemize}
\item A: One cart should stop and the other will move away at twice the speed. 
\item B: Both carts will stop.
\item C: Both carts will move away in the opposite direction at the same speeds.
\item D: Both carts will move away in the opposite direction at different speeds.
\end{itemize}
\end{enumerate}

\section{Chapter 10: Rotational Motion and Angular Momentum}

\begin{enumerate}
\item Suppose we hang the fish we've caught on a fishing pole.  The first fish has a mass of 0.5 kg, and we hang it at the end of a 1.2 meter fishing pole.  The second fish has a mass of 0.7 kg, and we hang it half-way down the pole.  (a) Draw a diagram including the pole (lever arm), and the weight vector of each fish.  (b) In which direction are the torques?  (c) What is the magnitude of the torque due to each fish?  (d) What is the total torque? \\ \vspace{3cm}
\item  A baseball pitcher throws the ball in a motion where there is rotation of the forearm about the elbow joint as well as other movements.  The linear velocity of the ball relative to the elbow joint goes from 0 m/s to 20.0 m/s in 0.4 seconds, at a distance of 0.480 m from the joint.  The moment of inertia of the forearm is 0.500 kg m$^2$.  What is the torque on the forearm?  (Neglect the mass of the ball).  \\ \vspace{2.5cm}
\end{enumerate}

\end{document}