\title{Study Guide for Midterm 3 for Algebra-Based Physics-1: Mechanics (PHYS135A-01)}
\author{Dr. Jordan Hanson - Whittier College Dept. of Physics and Astronomy}
\date{November 19th, 2017}
\documentclass[10pt]{article}
\usepackage[a4paper, total={18cm, 27cm}]{geometry}
\usepackage{outlines}
\usepackage[sfdefault]{FiraSans}
\usepackage{graphicx}
\usepackage{url}

\begin{document}
\maketitle
\small
\section{Rotational Kinematics and Dynamics}
\begin{enumerate}
\item Express the following angles in radians: (a) $10^{\circ}$ (b) $20^{\circ}$ (c) $30^{\circ}$ (d) $40^{\circ}$ \\ \vspace{1cm}
\item A record is spinning at 45 rpm.  What is the angular velocity in radians per second? \\ \vspace{1cm}
\item A \textit{sling} was an ancient weapon used to hurl polished stones at high velocity at enemies (like David versus Goliath).  Suppose the tangential velocity of the stone is $v=20$ m/s, and the radius is $r=1$ m.  (a) What is the angular velocity? (b) What is the centripetal acceleration at the location indicated by the arrow? (c) How many g's is this? \\ \vspace{1cm}
\item A car is traveling along a flat curved road with radius $r$ and frictional coefficient $\mu$.  (a) Show that if the net force is zero as the car goes through this curve, that $v = \sqrt{\mu rg}$. (b) What is $v$, if $r=600$ m, $g\approx 10$ m/s$^2$, and $\mu = 0.7$? \\ \vspace{0.5cm}
\end{enumerate}
\section{Newton's Law of Gravity}
\begin{enumerate}
\item Set Newton's Law of Gravity ($F_G = G m_1 m_2 / r^2$) equal to the weight force ($w = mg$), and rederive the fact that the acceleration due to gravity near the Earth's surface is $g\approx 9.81$ m/s$^2$. \\ \vspace{1cm}
\item Rederive Kepler's 3rd Law by setting Newton's Law of Gravity equal to the centripetal force of Earth. Using 1 year for the period of Earth's orbit, and $1.5 \times 10^{11}$ m for the radius of Earth's orbit, calculate the mass of the Sun. \\ \vspace{0.5cm}
\end{enumerate}
\section{Work and the Work-Energy Theorem}
\begin{enumerate}
\item If a force $\vec{F} = -3 \hat{i} + 9\hat{j}$ N is applied to an object, and the object is displaced by $\vec{d} = 2\hat{j}$ m, what is the work done? \\ \vspace{1cm}
\item (a) If a person drops a 0.6 kg baseball from a building that is 30 m tall, what is the final speed of the basketball? (b) If a person throws the same basketball 30 m into the air, what is the initial speed of the basketball? \\ \vspace{1cm}
\item Recall the loop-the-loop lab activity done in class, in which a marble rolls through a loop of radius $R$ if rolled from a height $h$.  In your own words, what should the ratio $h/R$ be?  (See \url{http://physics.bu.edu/~redner/211-sp06/class-energy/looptheloop.html} for a good discussion that neglects rotational inertia.)
\end{enumerate}
\end{document}