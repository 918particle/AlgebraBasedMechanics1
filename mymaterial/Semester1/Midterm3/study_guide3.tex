\title{Study Guide for Midterm 3 for Algebra-Based Physics-1: Mechanics (PHYS135A-01)}
\author{Dr. Jordan Hanson - Whittier College Dept. of Physics and Astronomy}
\date{November 6th, 2017}
\documentclass[10pt]{article}
\usepackage[a4paper, total={18cm, 27cm}]{geometry}
\usepackage{outlines}
\usepackage[sfdefault]{FiraSans}
\usepackage{graphicx}

\begin{document}
\maketitle

\section{Rotational Kinematics}
\begin{enumerate}
\item Suppose the radius of a circle 10 cm.  Recall that the speed of the edge of the circle as it rotates is $v = r\omega$, where r is the radius and $\omega$ is the angular velocity.  If the angular velocity is 100 rotations per minute, what is $v$?
\textbf{Solution}: 200 rotations per minute is $\omega = 100(2\pi)/60$ radians per second.  Since $r = 0.1$ m, we have $v = r\omega = 100(2\pi)/60/10 = \pi/3$ m/s.
\item An object begins a spin a 1 rotation per second, but ends up with 4 rotations per second.  If the change takes 1.5 seconds to complete, what is the average angular acceleration? \textbf{Solution}: Angular acceleration, like linear acceleration, is the change in speed divided by the change in time.  We have $(4-1)/1.5 = 2$ rotations per second-squared.
\item A centrifuge is spinning a sample at a radius of 5 cm with an angular velocity of 60 rotations per minute.  What is the speed $v$ of the sample?  \textbf{Solution}: $v = r\omega$, so $v = 0.05(60)(2\pi)/60 = \pi/10$ m/s. \textit{Please remember that $\omega$ should be in radians per second, which is why I convert from rotations per minute to radians per second with $2\pi/60$.}
\end{enumerate}
\section{Centripetal Acceleration and Centripetal Force}
\begin{enumerate}
\item A vehicle with mass 800 kg is going around a flat turn of radius 300 m.  The static friction coefficient between the tires and the pavement is 0.1.  What is the maximum speed with which the car may turn without sliding?  \textbf{Solution}: Centripetal force is being supplied by static friction between the tires and the road: $\mu m g = mv^2/r$, or $\mu g = v^2/r$.  Thus, $v = \sqrt{\mu rg} = \sqrt{300(10)/10} = \sqrt{300}$ m/s = 17 m/s.
\item Suppose that the car now encounters a banked curve, with a bank angle $\theta$.  By breaking the normal force into two components, show that the bank angle, velocity, and turn radius are related by $\tan\theta = \frac{v^2}{rg}$ (neglecting friction). \textbf{Solution}: See lecture notes unit 5, slide 22.
\end{enumerate}
\section{Newton's Law of Gravity, and the Solar System}
\begin{enumerate}
\item Suppose one massive object orbits another.  In your own words, explain why Newton's Third Law tells us that the force of gravity on the little one by the big one is the same as the the force on the big one by the little one.
\item Memorize how to derive 9.8 m/s$^2$ from Newton's Law of Gravity and Newton's Second Law.  See text, p. 217.
\item What would the acceleration be on another planet, if it were identical to Earth, but had a radius twice as large?  \textbf{Solution}: Since the gravity law goes as one over the radius squared, the acceleration would drop by a factor of 4, or 9.8/4 m/s$^2$.
\end{enumerate}
\end{document}