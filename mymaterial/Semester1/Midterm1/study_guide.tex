\title{Study Guide for Midterm 1}
\author{Dr. Jordan Hanson - Whittier College Dept. of Physics and Astronomy}
\date{\today}
\documentclass[10pt]{article}
\usepackage[a4paper, total={18cm, 27cm}]{geometry}
\usepackage{outlines}
\usepackage[sfdefault]{FiraSans}
\usepackage{graphicx}
\usepackage{amsmath}

\begin{document}
\maketitle

\section{Estimation and Unit Conversion}
\begin{enumerate}
\item Which of the following is most likely the speed of a runner on a track?
\begin{itemize}
\item A: 0.5 m/s
\item B: \textbf{5 m/s}
\item C: 50 m/s
\item D: 500 m/s
\end{itemize}
\item Convert the speed you chose in to kilometers per hour. \\
\begin{equation}
5 \left(\frac{m}{s}\right) \left(\frac{1 km}{1000 m}\right) \left(\frac{3600 s}{1 hr}\right) = 5 \times 3.6 km/hr = 18 km/hr
\end{equation}
\item Water flows through a pipe at a rate of 1000 cm$^3$/s.  What is this rate in m$^3$/hour? \\
\begin{equation}
10^3 \left(\frac{cm^3}{s}\right) \left(\frac{1 m}{100 cm}\right)^3 \left(\frac{3600 s}{1 hr}\right) = 3.6 \times 10^{-6+3+3} \left(\frac{m^3}{hr}\right) = 3.6 \left(\frac{m^3}{hr}\right) 
\end{equation}
\item One \textit{knot} is about 0.51 m/s.  A submarine travels at 20 knots, and another submarine travels at 25 knots.  What is the difference in speed, in meters per second? \\ \\
The difference in speed, \textit{in knots}, is 5 knots.  So $5 kn \times (0.51 m/s / kn) = 0.255 m/s$.
\end{enumerate}

\section{Displacement, Velocity, and Constant Acceleration Vectors}

\begin{enumerate}
\item An object has an initial position of 3 m, and a final position of -4 m, after 3.5 seconds elapses.  What is the average velocity? \\ 
\begin{equation}
v_{ave} = \frac{x_f-x_i}{t_f - t_i} = \frac{-4 - 3}{3.5 - 0} ~ (m/s) = -7/3.5 ~ (m/s) = -2.0 ~ (m/s)
\end{equation}
\item Suppose the position of an object is described by the following equation: $x(t) = 3.0 t + 5.0$ m.  Which of the following is true of the velocity and acceleration?
\begin{itemize}
\item A: Velocity is positive, acceleration is negative.
\item B: Velocity is negative, acceleration is positive.
\item C: \textbf{Velocity is positive, acceleration is zero.} (Remember, if velocity is a \textit{linear} function, acceleration is zero).
\item D: Velocity is negative, acceleration is zero.
\end{itemize}
\item If $x(t) = 3.0 t + 5.0$ m, what is the displacement between $t=1.0$ sec and $t=5.0$ sec? What is the acceleration?
\begin{itemize}
\item A: 8 m, 0 m/s$^2$
\item B: 12 m, 2 m/s$^2$
\item C: \textbf{12 m, 0 m/s$^2$} (It has to be one of the answers with no acceleration, and plug-in to find $x_f - x_i = x(5) - x(1) = 12 m$.
\item D: 8 m, 2 m/s$^2$
\end{itemize}
\item A basketball is shot horizontally from the top of a 100 m-tall building.  The initial vertical velocity is 0 m/s, and the initial horizontal velocity is 3 m/s.  How far away from the edge of the building does the ball land?  (You can assume that $g = -10$ m/s$^2$ for this problem). \\ \\
To find the horizontal displacement, we'd need to know the horizontal velocity (given, 3 m/s) and the \textit{time}.
\begin{equation}
\Delta x = v_x \Delta t \label{eq:1}
\end{equation}
We don't know $\Delta t$ yet.  Let's assume $t_i = 0$, so that $\Delta t = t_f - t_i = t_f$.  How do we get $t_f$?
\begin{equation}
y(t) = \frac{1}{2}at^2 + v_{i,y} t + y_{i}
\end{equation}
This equation is true, since we are applying it to the \textit{vertical direction only.}  Since the object is falling, we have an acceleration of $\vec{a} = -g \hat{j} ~ m/s^2$.  Also, $v_{i,y} = 0$ m/s because the ball is shot \textit{horizontally}, meaning it has no vertical velocity initially.  If the final y-position is at the ground, then 
\begin{align}
y_f - y_i &= -\frac{1}{2}gt_f^2 \\
0-h &= -\frac{1}{2}gt_f^2 \\
t_f &= \sqrt{2h/g}
\end{align}
Now we have the $t_f$, so we can plug it in to  Eq. \ref{eq:1}:
\begin{equation}
\Delta x = v_x \sqrt{2h/g} = 2\sqrt{20} ~ m
\end{equation}
\item What is the final velocity of the ball? \\ \\
To find the final velocity, we need both components of the velocity, $v_x$ and $v_y$.  But $v_x$ is \textit{constant} the entire time, because there is no horizontal acceleration.  To find $v_{y,f}$ at the time of landing, we can use
\begin{equation}
v_{y,f} = v_{i,y} + a t = -gt_f = -g \sqrt{2h/g}
\end{equation}
Remember, $v_{i,y} = 0$ because the ball was shot horizontally, and $a = -g$ (acceleration is down).  Now we have $v_{x,f}$ and $v_{y,f}$, which are components of the velocity vector.  If we know the components of the velocity vector, we can use Pythagorean theorem to solve for the magnitude:
\begin{equation}
|v| = \sqrt{v_{x,f}^2 + v_{y,f}^2} = \sqrt{v_{x,f}^2 + g^2 (2h/g)} = \sqrt{v_{x,f}^2 + 2gh} \approx 45 ~ m/s
\end{equation}
\end{enumerate}

\section{Vectors}
\begin{enumerate}
\item Let $\vec{x}_f = (3.0,-3.0)$ m, and $\vec{x}_i = (3.0,3.0)$ m.  What is $\Delta \vec{x} = \vec{x}_f - \vec{x}_i$? \\ \\
Remember to subtract x's from x's and y's from y's:
\begin{equation}
\Delta \vec{x} = (3.0-3.0, -3.0 - 3.0) ~ m = (0.0,-6.0) ~ m
\end{equation}
\item A jet fighter (Maverick) has an initial speed of 100 m/s, at a 60 degree angle with respect to horizontal.  Another fighter (Jester) has an initial speed of 100 m/s, but at a 45 degree angle with respect to horizontal.  What is the velocity of Maverick, minus the velocity of Jester?  \textit{Hint: it's not 0 m/s.  Build the velocity vector for each fighter first.} \\ \\
First, the velocity vector of Maverick: the hypoteneuse of the triangle is 100 m/s, and the angle is 60 degrees.  Thus, the x-component is $100 \cos(60^{\circ})$ m/s, and the y-component is $100 \sin(60^{\circ})$ m/s, so 
\begin{equation}
\vec{v}_M = (100/2,\sqrt{3}(100)/2) ~ m/s = (50,50\sqrt{3}) ~ m/s
\end{equation}
The same logic applies to the velocity vector of Jester, except it's 45 degrees instead of 60.
\begin{equation}
\vec{v}_J = (100/\sqrt{2},100/\sqrt{2}) ~ m/s
\end{equation}
\item If Maverick accelerates to a velocity of $v = (100,100)$ m/s, what is his speed? \\ \\
Use Pythagorean theorem: $\sqrt{100^2+100^2} = \sqrt{2 \times 10^2} = 100 \sqrt{2} ~ m/s$.
\item Multiply them via the dot-product.  Evaluate the dot product $\vec{x}_1 \cdot \vec{x}_2$, if $\vec{x}_1 = (0,1)$ m, and $\vec{x}_2 = (2,5)$ m. \\ \\
$(0,1) \cdot (2,5) ~ m^2 = 0\times 2 + 1\times 5 ~ m^2 = 5 ~ m^2 / s^2$
\end{enumerate}
\end{document}