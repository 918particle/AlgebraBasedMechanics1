\title{Study Guide for Midterm 1}
\author{Dr. Jordan Hanson - Whittier College Dept. of Physics and Astronomy}
\date{\today}
\documentclass[10pt]{article}
\usepackage[a4paper, total={18cm, 27cm}]{geometry}
\usepackage{outlines}
\usepackage[sfdefault]{FiraSans}
\usepackage{graphicx}
\usepackage{amsmath}

\begin{document}
\maketitle

\section{Estimation and Unit Conversion}
\begin{itemize}
\item Below are some examples of estimation and unit conversion.
\begin{enumerate}
\item Which of the following is most likely the speed of a runner on a track?
\begin{itemize}
\item A: 0.5 m/s
\item B: 5 m/s
\item C: 50 m/s
\item D: 500 m/s
\end{itemize}
\item Convert the speed you chose in to kilometers per hour. \\ \vspace{0.5cm}
\item Water flows through a pipe at a rate of 1000 cm$^3$/s.  What is this rate in m$^3$/hour? \\ \vspace{1cm}
\item One \textit{knot} is about 0.51 m/s.  A submarine travels at 20 knots, and another submarine travels at 25 knots.  What is the difference in speed, in meters per second? \\ \vspace{1cm}
\end{enumerate}
\end{itemize}

\section{Displacement, Velocity, and Constant Acceleration Vectors}
The definition of average velocity and acceleration are
\begin{align}
\vec{v}_{ave} & = \frac{\vec{x}_{f} - \vec{x}_{i}}{t_{f} - t_{i}} = \frac{\Delta \vec{x}}{\Delta t} \label{eq:1} \\
\vec{a}_{ave} & = \frac{\vec{v}_{f} - \vec{v}_{i}}{t_{f} - t_{i}} = \frac{\Delta \vec{v}}{\Delta t}
\label{eq:2}
\end{align}
The numerator of Eq. \ref{eq:1} is in general a vector called \textit{the displacement}: $\Delta \vec{x}$, describing the change in position of an object.  If an object experiences constant acceleration, $\vec{a}$, the following equations apply:
\begin{align}
x(t) &= x_0 + v_0 t + \frac{1}{2} a t^2 \label{eq:kinematics_const_acc1} \\
v(t) &= v_0 + a t \label{eq:kinematics_const_acc2} \\
a(t) &= a \label{eq:kinematics_const_acc3} \\
v^2 &= v_0^2 + 2a\Delta x \label{eq:kinematics_const_acc4}
\end{align}
\clearpage

Let's practice solving problems with these equations, starting with Eq. \ref{eq:1} and \ref{eq:2}.
\begin{enumerate}
\item An object has an initial position of 3 m, and a final position of -4 m, after 3.5 seconds elapses.  What is the average velocity? \\ \hspace{0.75cm}
\item Suppose the position of an object is described by the following equation: $x(t) = 3.0 t + 5.0$ m.  Which of the following is true of the velocity and acceleration?
\begin{itemize}
\item A: Velocity is positive, acceleration is negative.
\item B: Velocity is negative, acceleration is positive.
\item C: Velocity is positive, acceleration is zero.
\item D: Velocity is negative, acceleration is zero.
\end{itemize}
\item If $x(t) = 3.0 t + 5.0$ m, what is the displacement between $t=1.0$ sec and $t=5.0$ sec? What is the acceleration?
\begin{itemize}
\item A: 8 m, 0 m/s$^2$
\item B: 12 m, 2 m/s$^2$
\item C: 12 m, 0 m/s$^2$
\item D: 8 m, 2 m/s$^2$
\end{itemize}
\end{enumerate}
Now let's practice using Eqs. \ref{eq:kinematics_const_acc1}-\ref{eq:kinematics_const_acc4}.  
\begin{enumerate}
\item A basketball is shot horizontally from the top of a 100 m-tall building.  The initial vertical velocity is 0 m/s, and the initial horizontal velocity is 3 m/s.  How far away from the edge of the building does the ball land?  (You can assume that $g = -10$ m/s$^2$ for this problem). \\ \vspace{2cm}
\item What is the final velocity of the ball? \\ \vspace{1cm}
\end{enumerate}

\section{Vectors}
You must be able to do the following with vectors:
\begin{itemize}
\item Add and subtract them.
\begin{enumerate}
\item Let $\vec{x}_f = (3.0,-3.0)$ m, and $\vec{x}_i = (3.0,3.0)$ m.  What is $\Delta \vec{x} = \vec{x}_f - \vec{x}_i$? \\ \vspace{1cm}
\item A jet fighter (Maverick) has an initial speed of 100 m/s, at a 60 degree angle with respect to horizontal.  Another fighter (Jester) has an initial speed of 100 m/s, but at a 45 degree angle with respect to horizontal.  What is the velocity of Maverick, minus the velocity of Jester?  \textit{Hint: it's not 0 m/s.  Build the velocity vector for each fighter first.} \\ \vspace{1.5cm}
\end{enumerate}
\item Compute their magnitude.  If Maverick accelerates to a velocity of $v = (100,100)$ m/s, what is his speed? \\ \vspace{1cm}
\item Multiply them via the dot-product.  Evaluate the dot product $\vec{x}_1 \cdot \vec{x}_2$, if $\vec{x}_1 = (0,1)$ m, and $\vec{x}_2 = (2,5)$ m. \\ \vspace{1cm} 
\end{itemize}
\end{document}