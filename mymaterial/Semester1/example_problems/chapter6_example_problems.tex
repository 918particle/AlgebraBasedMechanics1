\documentclass{beamer}
\usetheme{metropolis}
\usepackage{graphicx}
\usepackage{subfig}
\title{Algebra-Based Physics-1: Mechanics (PHYS135A-01): Example Problems File}
\date{September 6th - September 8th, 2017}
\author{Jordan Hanson}
\institute{Whittier College Department of Physics and Astronomy}

\begin{document}
\maketitle

\section{Chapter 6}

\begin{frame}{Examples from Chapter 6}
\textbf{Conceptual Question 2.} \textit{Can centripetal acceleration change the speed of circular motion?  Explain.}
\end{frame}

\begin{frame}{Examples from Chapter 6}
\textbf{Conceptual Question 2.} \alert{Answer:} No, it only changes the direction.  Figure 6.8 in the text shows that the velocity only changes direction, not magnitude, for uniform circular motion.
\end{frame}

\begin{frame}{Examples from Chapter 6}
\textbf{Conceptual Question 6.} \textit{Race car drivers routinely cut corners as shown in Figure 6.32. Explain how this allows the curve to be taken at the greatest speed.}
\end{frame}

\begin{frame}{Examples from Chapter 6}
\textbf{Conceptual Question 6.} \alert{Answer:} Path 2 in the figure has a larger radius of curvature.  Widening the radius of curvature lowers the necessary centripetal force.  Assuming the tires are providing the maximum possible static friction, a larger radius of curvature allows for a larger velocity while maintaining traction (static friction).
\end{frame}

\begin{frame}{Examples from Chapter 6}
\textbf{Conceptual Question for Newton's Law of Gravitation.} \textit{Two objects of equal mass are drifting toward each other.  If the acceleration experienced by each is 5g when the objects are 5000 km apart, what is the acceleration when they are 2500 km apart?}
\end{frame}

\begin{frame}{Examples from Chapter 6}
\textbf{Conceptual Question for Newton's Law of Gravitation.} \alert{Answer:} If the distance decreases by a factor of two, the force of gravity must increase by a factor of four.  Remember that by Newton's Third Law, the force of the first object on the second is equal and opposite to the force of the second object on the first.  But the force is equal to $m \vec{a}$, so if the masses are equal then the accelerations are equal.  The acceleration will increase by a factor of 4.0, from 5g to 20g.
\end{frame}

\section{Answers}

\begin{frame}{Answers}
\begin{columns}[T]
\begin{column}{0.5\textwidth}
\begin{itemize}
\item ...
\end{itemize}
\end{column}
\begin{column}{0.5\textwidth}
\begin{itemize}
\item ...
\end{itemize}
\end{column}
\end{columns}
\end{frame}

\end{document}
