\title{Midterm 2 for Algebra-Based Physics-1: Mechanics (PHYS135A-01)}
\author{Dr. Jordan Hanson - Whittier College Dept. of Physics and Astronomy}
\date{October 16th, 2017}
\documentclass[10pt]{article}
\usepackage[a4paper, total={18cm, 27cm}]{geometry}
\usepackage{outlines}
\usepackage[sfdefault]{FiraSans}
\usepackage{graphicx}

\begin{document}
\maketitle

\section{Vectors and Newton's Laws}
\begin{enumerate}
\item Let $\vec{F}_{\rm 1} = \frac{3}{4}\hat{i} + 1\hat{j}$ N, and $\vec{F}_{\rm 2} = -1\hat{i} + \frac{3}{4}\hat{j}$ N.  a) Give the magnitude of each force.  b) What is the net force?  c) What is the angle between these two forces? \vspace{2.0 cm}
\item Imagine you are sitting in an airplane that has just lifted off with an acceleration vector 20 degrees with respect to horizontal.  Draw a free-body diagram corresponding to you, showing all forces acting on you.
\vspace{2.0 cm}
\item Imagine you are riding a skateboard down a hill (no friction), and the incline angle is 30 degrees.  Draw a free-body diagram corresponding to you, showing all forces acting on you.
\vspace{2.0 cm}
\end{enumerate}
\section{Young's Modulus}
\begin{enumerate}
\item Someone in the laboratory hands us a piece of metal, and needs to know what kind of metal it is.  We decide to measure the Young's modulus, $Y$.  The piece of metal is a cylinder with cross-sectional area $A = 4\pi \times 10^{-4}$ m$^2$, and length $L = 0.1$ m.  We apply a force $F=10^4$ N to squeeze the piece of metal, and the length changes by $x = 10^{-5}$ m.  Young's modulus $Y$ is defined so that:
\begin{equation}
\frac{x}{L} = \frac{p}{Y} = \frac{F/A}{Y}
\end{equation}
What is Y, for this material?  How does this value compare to aluminum or iron?
\begin{itemize}
\item $8 \times 10^{8}$ Pa
\item $80 \times 10^{9}$ N
\item $80 \times 10^{9}$ Pa
\item $8 \times 10^{9}$ Pa
\end{itemize}
\end{enumerate}
\clearpage
\section{Frictional Forces}
\begin{enumerate}
\item There is a spill of a mystery toxic liquid on a shop floor, and no one wants to touch it.  Someone gets the bright idea that they can identify it by the coefficient of kinetic friction and a steel plate.  Draw a free body diagram corresponding to a steel plate sliding along the liquid/floor, with friction decelerating it.\vspace{2.5cm}
\item What is the coefficient of kinetic friction, $\mu_{\rm k}$, if a steel plate with an initial speed of 2 m/s comes to a stop after 1.0 seconds, assuming $g = 10$ m/s$^2$?  (Use the definition of acceleration $\Delta v/\Delta t = a$).
\begin{itemize}
\item 0.1
\item 0.2
\item 0.5
\item 1.2
\end{itemize}
\item Suppose they get a sample of the mystery liquid in a vile.  They assume the drag force is given by Stoke's Law, $F_{\rm D} = 6\pi r \eta v$, where $v$ is the velocity of a particle moving through they fluid, $r$ is the radius of the particle, and $\eta$ is the \textit{viscosity}.  They drop a bead with $r = 1$ mm and a mass of 2 grams into the fluid, and observe the bead sink with a constant (terminal) velocity of 1 m/s.  What is the viscosity of the fluid?  Units: kg/(m s).
\begin{itemize}
\item $10/(3\pi)$ kg/(m s)
\item $1/(3\pi)$ kg/(m s)
\item $20$ kg/(m s)
\item $10$ kg/(m s)
\end{itemize}
\end{enumerate}
\end{document}