\title{Syllabus for Algebra-Based Physics-2: Electricity, Magnetism, and Modern Physics (PHYS135B-01)}
\author{Dr. Jordan Hanson - Whittier College Dept. of Physics and Astronomy}
\date{\today}
\documentclass[10pt]{article}
\usepackage[a4paper, total={18cm, 27cm}]{geometry}
\usepackage{outlines}
\usepackage[sfdefault]{FiraSans}
\usepackage{hyperref}
\begin{document}
\maketitle

\begin{abstract}
The concepts of algebra-based electricity, magnetism, and modern physics will be presented within the context of interactive problem-solving.  The course will begin with the concepts of electric charge, electrostatics, and electric potential.  Following electrostatics, applications to DC circuits will be covered.  Next, the topics of magnetism and electromagnetism will be covered, concluding with light and optics.  Time-permitting, selected topics in modern physics will be added.  The course work will include interactive computational exercises, analytic textbook problems, group-designed projects, and lab-based activities.
\end{abstract}
\noindent
\textit{\textbf{Pre-requisites}: PHYS-135A.} \\
\textit{\textbf{Course credits, Liberal Arts Categorization}: 4 Credits, None} \\
\textit{\textbf{Regular course hours}: Tuesday and Thursday from 8:50 - 10:50 in SLC 228} \\
\textit{\textbf{Instructor contact information}: jhanson2@whittier.edu, tel. 562.907.5130} \\
\textit{\textbf{Office hours}: Mondays from 8:00-12:00, SLC 212.} \\
\textit{\textbf{Attendance/Absence}: Students needing to reschedule midterms and exams should notify the professor a reasonable time beforehand. Further attendance issues are left to the discretion of the instructor}.\\ 
\textit{\textbf{Late work policy}: Late work is generally not accepted, but is left to the discretion of the instructor.} \\
\textit{\textbf{Text}: College Physics (openstax.org) -  \url{https://openstax.org/details/books/college-physics}} \\
\textit{\textbf{Grading}: There will be three tests, each examining conceptual understanding in step-by-step problems. Each midterm is worth 15\% of the final grade. The weekly online homework is worth 20\% of the grade. Interactive in-class activities will be worth 10\% of the final grade. Lab groups will present results of a group project worth 10\% of the grade. The final exam will be held on May 11th, 8:00-10:00 am, and will be worth 15\% of the grade.} \\
\textit{\textbf{Grade Settings}: $<60\%$ = F, $\geq 60\%, <70\%$ = D, $\geq 70\%, <80\%$ = C, $\geq 80\%, <90\%$ = B, $\geq 90\%, <100\%$ = A.  Pluses and minuses: 0-3\% minus, 3\%-6\% straight, 6\%-10\% plus (e.g. 79\% = C+, 91\% = A-)} \\
\textit{\textbf{Homework Sets}: Typically 5-10 problems per week, assigned and collected on Tuesdays.  See \url{http://goeta.link/USB06CA-FF35D5-1ZU} for online homework setup.} \\
\textit{\textbf{Bonus Essay}: Students may submit an essay on the history of scientific developments covered in the course, due at the end of the semester. The essay must be 10 pages, address scientific arguments and results, and must include references. The grade of this paper will replace the lowest midterm grade. Students wishing to submit must notify the professor no later than one week after the second midterm.} \\
\textit{\textbf{Thursday science presentations.} Optionally, students may present a recent scientific article or publication to the class on Thursdays for a bonus point on the most recent homework or midterm. Limit one point per student per Thursday.} \\
\textit{\textbf{ADA Statement on Disability Services}: The Americans with Disabilities Act (ADA) is a federal anti-discrimination statute that provides comprehensive civil rights protection for persons with disabilities. Among other things, this legislation requires that all students with disabilities be guaranteed a learning environment that provides for reasonable accommodation of their disabilities. If you believe you have a disability requiring an accommodation, please contact Disability Services: disabilityservices@whittier.edu, tel. 562.907.4825.} \\
\textit{\textbf{Academic Honesty Policy}: \url{http://www.whittier.edu/academics/academichonesty}} \\
\textit{\textbf{Course Objectives}:}
\begin{itemize}
\item To practice written and oral expression of scientifically technical ideas.
\item To solve word problems pertaining to physics and mathematics.
\item To construct mathematical models of electrical systems like DC circuits.
\item To apply logical thinking to conceptually-posed physics problems.
\item To practice scientific experimentation, data analysis, and reporting of results.
\end{itemize}
\clearpage
\small
\textit{\textbf{Course Outline}:}
\begin{outline}[enumerate]
\1 \textbf{Unit 0:} Review of pre-requisite courses, and electrostatics.
\2 Unit analysis, kinematics, and Newton's Laws
\2 Work and energy, momentum
\2 Electrostatics, I - \textbf{Chapters - 18.1 - 18.5}
\3 The Coulomb Force, and Newton's Second Law for electric charges
\3 The concept of an electric field
\3 Tuesday reading quiz: chapter 18.3
\2 Electrostatics, II, electric potential - \textbf{Chapters 18.6 - 18.8, 19.1 - 19.3}
\3 Charge and electric fields in biology
\3 Potential energy and charge: voltage
\3 Potential energy and fields, point charges
\3 Tuesday reading quiz: chapter 19.2
\1 \textbf{Unit 1:} Capacitors, current and DC circuits
\2 Capacitors and capacitance - \textbf{Chapters 19.4 - 19.7}
\3 Equipotential lines
\3 Capacitance and capacitors
\3 Capacitors in series and in parallel, energy considerations
\3 Tuesday reading quiz: - chapter 19.6
\2 Current and DC circuits - \textbf{Chapters 20.1 - 20.4, 20.7}
\3 DC current and resistance, Ohm's law
\3 Energy and power in DC current
\3 Biological example: human nerve conduction
\3 Tuesday reading quiz: chapter 20.2
\1 \textbf{First midterm, end of Unit 1, February 27th, 2020.} The first midterm will be shorter than the second and third ones, focusing on the Coulomb force and electric fields, voltage, and capacitance.
\1 \textbf{Unit 2:} DC Circuits with resistors in series and parallel, RC circuits
\2 DC circuit basics - \textbf{Chapters 21.1 - 21.4}
\3 Resistors in series and parallel, electromotive force (EMF)
\3 Kirchhoff's rules
\3 Voltmeters and ammeters
\3 Tuesday reading quiz: chapter 21.3
\1 \textbf{Unit 3:} Magnetism I
\2 Magnetostatics I - \textbf{Chapters 22.1 - 22.5}
\3 Magnets, ferromagnetic and electromagnetic
\3 Magnetic fields and field lines, force on moving charge
\3 Magnetic applications I
\3 Tuesday reading quiz: chapter 22.4
\2 Magnetostatics II - \textbf{Chapters 22.6 - 22.11}
\3 The Hall effect
\3 Force on current carrying conductor, torque on current loop
\3 Amp\`{e}re's Law: magnetic fields created by current
\3 Tuesday reading quiz: 22.6
\1 \textbf{Second midterm, end of Unit 3, March 26th, 2020.} The second midterm will cover DC circuits and Kirchhoff’s rules, mag-
netic force on moving charge and current, and applications of magnetic torque like the mass spectrometer.
\1 \textbf{Unit 4:} Magnetism II
\2 Magnetic induction - \textbf{Chapters 23.1 - 23.5, 23.7}
\3 Induced EMF, magnetic flux
\3 Faraday's Law
\3 Motional EMF and generators, transformers
\3 Tuesday reading quiz: chapter 23.2
\1 \textbf{Unit 5:} Electromagnetic waves
\2 Maxwell's equations and EM wave production: conceptual understanding of universal physics - \textbf{Chapters 24.1 - 24.4}
\3 The electromagnetic spectrum
\3 Energy in electromagnetic waves
\3 Tuesday reading quiz: 24.2
\1 \textbf{Third midterm, end of Unit 5, April 23rd, 2020.} The third midterm will cover magnetic induction and Faraday's law, generators and transformers, and radiation.
\1 \textbf{Unit 6:} Cumulative Review, group presentations, and final exam
\2 No reading quizzes
\2 Group presentations
\3 Worth 10\% of the final grade
\3 Given as a group
\3 10-15 minute presentation with slides or board work
\3 Final exam reviews will be given the last week of class, or potentially faculty reading day
\3 The final exam will be held on May 11th, 8:00-10:00, and will be worth 15\% of the grade.
\end{outline}
\end{document}
