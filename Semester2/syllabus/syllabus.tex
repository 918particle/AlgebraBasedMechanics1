\title{Syllabus for Algebra-Based Physics-2: Electricity, Magnetism, and Modern Physics (PHYS135B-01)}
\author{Dr. Jordan Hanson - Whittier College Dept. of Physics and Astronomy}
\date{\today}
\documentclass[10pt]{article}
\usepackage[a4paper, total={18cm, 27cm}]{geometry}
\usepackage{outlines}
\usepackage{hyperref}
\begin{document}
\maketitle

\begin{abstract}
The concepts of algebra-based electricity, magnetism, and modern physics will be presented within the context of interactive problem-solving.  The course will begin with the concepts of electric charge, electrostatics, and electric potential.  Following electrostatics, applications to DC circuits will be covered.  Next, the topics of magnetism and electromagnetism will be covered, concluding with light and optics.  Time-permitting, selected topics in modern physics will be added.  The course work will include interactive computational exercises, analytic textbook problems, group-designed projects, and lab-based activities.
\end{abstract}
\noindent
\textit{\textbf{Pre-requisites}: PHYS135A} \\
\textit{\textbf{Course credits, Liberal Arts Categorization}: 4 Credits, None} \\
\textit{\textbf{Regular course hours and location}: Tuesday and Thursday, 8:50 - 10:50, SLC 232.} \\
\textit{\textbf{Instructor contact information}: 
\begin{itemize}
\item Discord: 918particle
\item Email: jhanson2@whittier.edu
\item Office: SLC 212
\item Cell: 562.351.0047
\item YouTube Channel: \url{www.youtube.com/918particle}
\item Book online appointments: \url{https://fgucmvjkylvmgqfsco.10to8.com}
\end{itemize}}
\noindent
\textit{\textbf{Office hours}: Booking service to schedule meeting: \url{10to8.com} as above, and indicate in-person or online.} \\
\textit{\textbf{Attendance/Absence}: Students needing to reschedule midterms must notify the professor a few days in advance.} \\ 
\textit{\textbf{Late work policy}: Late work is generally not accepted, but is left to the discretion of the instructor.} \\
\textit{\textbf{Text}: Go to \url{https://openstax.org/details/books/college-physics-2e} for access (free).} \\
\textit{\textbf{Homework}: } \\
\textit{\textbf{Grading}: The course grade will be a weighted average of assignment scores, and the weights are listed in Tab. \ref{tab:grades}.}

\begin{table}[h]
\centering
\begin{tabular}{| c | c |}
\hline
Item & Percentage \\ \hline \hline
Daily exercises & 10 \% \\ \hline
Homework sets and labs & 30 \% \\ \hline
Midterm & 20 \% \\ \hline
Final & 20 \% \\ \hline
Final Project + Presentation & 20\% \\ \hline
\end{tabular}
\caption{\label{tab:grades} These are the grade weights for each assignment. \textbf{Grade Settings}: $\geq 60\%, <70\%$ = D, $\geq 70\%, <80\%$ = C, $\geq 80\%, <90\%$ = B, $\geq 90\%, <100\%$ = A. Pluses and minuses: 0-3\% minus, 3\%-6\% straight, 6\%-10\% plus (e.g. 79\% = C+, 91\% = A-).}
\end{table}
\clearpage
\noindent
\textit{\textbf{Statement on Disability Services}: Whittier College is committed to make learning experiences as accessible as possible. If you experience physical or academic barriers due to a disability, you are encouraged to contact Student Disability Services (SDS) to discuss options. To learn more about academic accommodations, please email disabilityservices@whittier.edu.} \\
\textit{\textbf{Mental Health Resources:} Counseling services for enrolled students are offered at no charge and will be offered remotely during periods of remote learning. When we return to campus, in person and remote/telehealth services will be offered. Schedule an appointment by emailing counselingcenter@whittier.edu  or by phone, 562-907-4239 (between 8am – 5pm; M-F). For support after 5pm, you may call the After-Hours RN Telephone Advice Line at 562.454.4548 (press option 1); for mental health emergencies contact Campus Safety at 562-907-4211; Digital mental health platform at \url{https://you.whittier.edu/}} \\ \\
\noindent
\textit{\textbf{Course Policy Updates}: Our course combines in-person discussions and activities with online Zoom meetings and asynchronous pre-recorded content.
\begin{enumerate}
\item Class will meet via Zoom until February 21st.  There will also be asynchronous activities distributed through Moodle that do not require us to meet over Zoom.
\item \textbf{Students may opt-out of taking the final exam.}  The final exam is optional.  If one does not take it, the assignment weights for the final grade will be those given in Tab. \ref{tab:grades} (right).
\item The final project can be created in one of two options.  \textbf{Option A}: A 10 minute traditional presentation with several minutes for questions.  \textbf{Option B}: Digital liberal arts style, in the form of video or digital book form that educates the class on a topic.  Regardless of the option, students will all present their work to the class at the end of the module.
\end{enumerate}}
\noindent
\textit{\textbf{Course Objectives}:}
\begin{itemize}
\item To practice written and oral expression of scientifically technical ideas.
\item To solve word problems pertaining to physics and mathematics.
\item To construct mathematical models of electrical systems like DC circuits.
\item To apply logical thinking to conceptually-posed physics problems.
\item To practice scientific experimentation, data analysis, and reporting of results.
\end{itemize}

\textit{\textbf{Course Outline}:}
\begin{outline}[enumerate]
\1 \textbf{Unit 0:} Review of pre-requisite courses, and electrostatics.
\2 Unit analysis, kinematics, and Newton's Laws
\2 Work and energy, momentum
\2 Electrostatics, I - \textbf{Chapters - 18.1 - 18.5}
\3 The Coulomb Force, and Newton's Second Law for electric charges
\3 The concept of an electric field
\2 Electrostatics, II, electric potential - \textbf{Chapters 18.6 - 18.8, 19.1 - 19.3}
\3 Charge and electric fields in biology
\3 Potential energy and charge: voltage
\3 Potential energy and fields, point charges
\1 \textbf{Unit 1:} Capacitors, current and DC circuits
\2 Capacitors and capacitance - \textbf{Chapters 19.4 - 19.7}
\3 Equipotential lines
\3 Capacitance and capacitors
\3 Capacitors in series and in parallel, energy considerations
\2 Current and DC circuits - \textbf{Chapters 20.1 - 20.4, 20.7}
\3 DC current and resistance, Ohm's law
\3 Energy and power in DC current
\3 Biological example: human nerve conduction
\1 \textbf{Unit 2:} DC Circuits with resistors in series and parallel, RC circuits
\2 DC circuit basics - \textbf{Chapters 21.1 - 21.4}
\3 Resistors in series and parallel, electromotive force (EMF)
\3 Kirchhoff's rules
\3 Voltmeters and ammeters
\1 \textbf{Unit 3:} Magnetism I
\2 Magnetostatics I - \textbf{Chapters 22.1 - 22.5}
\3 Magnets, ferromagnetic and electromagnetic
\3 Magnetic fields and field lines, force on moving charge
\3 Magnetic applications I
\2 Magnetostatics II - \textbf{Chapters 22.6 - 22.11}
\3 The Hall effect
\3 Force on current carrying conductor, torque on current loop
\3 Amp\`{e}re's Law: magnetic fields created by current
\1 \textbf{Unit 4:} Magnetism II
\2 Magnetic induction - \textbf{Chapters 23.1 - 23.5, 23.7}
\3 Induced EMF, magnetic flux
\3 Faraday's Law
\3 Motional EMF and generators, transformers
\1 \textbf{Unit 5:} Cumulative Review, group presentations, and final exam
\2 No warm-up exercises
\2 Group presentations, both option A and B
\3 Given as a pair or team
\3 10-15 minute duration with questions
\3 Whiteboard is available
\3 Final exam is optional (see Tab. \ref{tab:grades}).
\end{outline}
\textbf{The five pieces of technology driving the course:}
\begin{enumerate}
\item OpenStax Textbook, College Physics: \url{https://openstax.org/details/books/college-physics}.  Sign up at openstax.org for a free account, and you will be able to save custom highlights.  The book can also be downloaded as a PDF or accessed via the smartphone app on Android OS and iOS.
\item OpenStax Tutor, for homework and reading assignments: Course link, \url{https://tutor.openstax.org/enroll/235830/PHYS135B-Spring-2022}.  There is a complete getting started guide on Moodle.
\item Pivot Interactives laboratory software: \url{pivotinteractives.com}. Tutorial: \url{https://youtu.be/ARh_nPriGzk}.  Class key: ck-ab00f0ae.  Join class: \url{https://app.pivotinteractives.com/join-class?classKey=ck-ab00f0ae}.
\item Online booking service for office hours, 10to8.com: \url{10to8.com}.  Book an online appointment here: \url{https://fgucmvjkylvmgqfsco.10to8.com}.  Appointments are automatically synced with instructor schedule and last 30 minutes via Zoom (same ID and password as class time).
\item Moodle: We will use modules native to Moodle for sharing lecture notes, quizzes and submitting project files. 
\end{enumerate}
\end{document}
