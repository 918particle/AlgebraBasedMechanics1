\title{Syllabus for Algebra-Based Physics II: Electricity, Magnetism, and Modern Physics (PHYS135B-01)}
\author{Dr. Jordan Hanson - Whittier College Dept. of Physics and Astronomy}
\date{\today}
\documentclass[10pt]{article}
\usepackage[a4paper, total={18cm, 27cm}]{geometry}
\usepackage{outlines}
\usepackage{hyperref}
\begin{document}
\maketitle

\begin{abstract}
The concepts of algebra-based electromagnetism, optics, and modern physics will be presented within the context of interactive problem-solving.  The course will begin with the concepts of electric charge, electrostatics, electric potential, and capacitance.  Next, current and DC circuits will be covered.  Magnetostatics and magnetic induction follow DC circuits, concluding with AC circuits.  Finally, seleted topics in electromagnetic waves, optics, and modern medicine will be presented.  The course work will include interactive computational exercises, analytic textbook problems, group-designed projects, and lab-based activities.
\end{abstract}
\noindent
\textit{\textbf{Pre-requisites}: PHYS135A} \\
\textit{\textbf{Course credits, Liberal Arts Categorization}: 4 Credits, None} \\
\textit{\textbf{Regular course hours and location}: Tuesday and Thursday, 8:50 - 10:50, SLC 232.} \\
\textit{\textbf{Instructor contact information}: Discord: 918particle, Email: jhanson2@whittier.edu, Office: SLC 212, YouTube Channel: \url{www.youtube.com/918particle}, Book online appointments: \url{https://fgucmvjkylvmgqfsco.10to8.com}.}
\noindent
\textit{\textbf{Office hours}: Booking service to schedule meeting: \url{10to8.com} as above, and indicate in-person or online.} \\
\textit{\textbf{Attendance/Absence}: Students needing to reschedule midterms must notify the professor a few days in advance.} \\ 
\textit{\textbf{Late work policy}: Late work is generally not accepted, but is left to the discretion of the instructor.} \\
\textit{\textbf{Text}: College Physics, 2nd Ed. \url{https://openstax.org/details/books/college-physics-2e}. Open-access.} \\
\textit{\textbf{Homework}: Problem sets will be assigned on Fridays, and due after one week.} \\
\textit{\textbf{Grading}: The course grade will be a weighted average of assignment scores, and the weights are listed in Tab. \ref{tab:grades}.}
\begin{table}
\centering
\begin{tabular}{| c | c | c |}
\hline
\textbf{Assignment} & \textbf{Weight} & \textbf{Date} \\ \hline
Daily exercises & 10 \% & Completed during class\\ \hline
Homework sets and labs & 30 \% & Weekly on Fridays \\ \hline
First Midterm & 20 \% & March 8th, 2024 (take-home style on Units 0-3)\\ \hline
Second Midterm & 20 \% & April 26th, 2024 (take-home style on Units 4-5) \\ \hline
Final Project Presentation & 20\% & April 23rd and 25th, 2024 (in class) \\ \hline
\end{tabular}
\caption{\label{tab:grades} These are the grade weights for each assignment. The final project presentation can take two forms.  \textbf{Option A}: A 10-15 minute traditional presentation with several minutes for questions.  \textbf{Option B}: A video in digital storytelling format using WeVideo, also 10-15 minutes long.}
\end{table}
\noindent
\textit{\textbf{Grade Settings}: $\geq 60\%, <70\%$ = D, $\geq 70\%, <80\%$ = C, $\geq 80\%, <90\%$ = B, $\geq 90\%, <100\%$ = A. Pluses and minuses: 0-3\% minus, 3\%-6\% straight, 6\%-10\% plus (e.g. 79\% = C+, 91\% = A-).} \\
\textit{\textbf{ADA Statement on Disability Services}: Whittier College is committed to make learning experiences as accessible as possible. If you experience physical or academic barriers due to a disability, you are encouraged to contact Student Disability Services (SDS) to discuss options. To learn more about academic accommodations, email disabilityservices@whittier.edu, call (562) 907-4825, or go to SDS which is located on the ground floor of Wardman Library.} \\
\textit{\textbf{Academic Honesty:} \url{https://www.whittier.edu/policies/academic/honesty}} \\
\noindent
\textit{\textbf{Course Objectives}:}
\begin{itemize}
\item To solve word problems in the subjects of physics and mathematics
\item To construct mathematical models of physical systems
\item To match mathematical models of physical systems to experimental results
\item To apply logical thinking to conceptually-posed physics problems
\item To practice scientific experimentation, data analysis, and reporting of results
\item To practice written and oral expression of scientifically technical ideas
\item To understand how to use measurement devices and electronics tools
\end{itemize}
\clearpage
\twocolumn
\textit{\textbf{Course Outline}:}
\begin{outline}[enumerate]
\1 \textbf{Unit 0:} Review of pre-requisite courses, and electrostatics.
\2 Unit analysis, kinematics, and Newton's Laws
\2 Work and energy, momentum
\2 Electrostatics, I - \textbf{Chapters - 18.1 - 18.5}
\3 The Coulomb Force, and Newton's Second Law for electric charges
\3 The concept of an electric field
\2 Electrostatics, II, electric potential - \textbf{Chapters 19.1 - 19.3}
\3 Potential energy and charge, voltage
\3 Potential energy and fields, point charges
\3 Electrostatics in biology
\1 \textbf{Unit 1:} Capacitors, current and DC circuits
\2 Capacitors and capacitance - \textbf{Chapters 19.4 - 19.7}
\3 Equipotential lines, capacitance, and capacitors
\3 Capacitors in series and in parallel, energy considerations
\2 Current and DC circuits - \textbf{Chapters 20.1 - 20.5, 20.7}
\3 DC current and resistance, Ohm's law
\3 Energy and power in DC current
\3 AC current and waveforms
\3 Biological example: human nerve conduction
\1 \textbf{Unit 2:} DC Circuits with resistors in series and parallel, RC circuits
\2 DC circuit basics - \textbf{Chapters 21.1 - 21.4, 21.6}
\3 Resistors in series and parallel, electromotive force (EMF)
\3 Kirchhoff's rules
\3 Voltmeters and ammeters
\3 RC circuits
\1 \textbf{Unit 3:} Magnetism I
\2 Magnetostatics I - \textbf{Chapters 22.1 - 22.4}
\3 Magnets, ferromagnetic and electromagnetic
\3 Magnetic fields and field lines, force on moving charge
\3 Magnetic applications I: fusion reactors
\2 Magnetostatics II - \textbf{Chapters 22.7 - 22.9}
\3 Force on current carrying conductor, torque on current loop
\3 Amp\`{e}re's Law: magnetic fields created by current
\3 Magnetic applications II: mass spectrometry
\1 \textbf{First Midterm - March 8th, 2024.}
\2 Take-home style, covers Units 0-3
\1 \textbf{Unit 4:} Magnetism II
\2 Magnetic induction - \textbf{Chapters 23.1 - 23.5, 23.7, 23.9}
\3 Induced EMF, magnetic flux
\3 Faraday's Law
\3 Motional EMF and generators, transformers
\2 AC circuits - \textbf{Chapters 23.10 - 23.12}
\3 RL circuits
\3 RLC circuits
\1 \textbf{Unit 5:} Waves, Optics, Medical Physics
\2 Electromagnetic waves - \textbf{Chapters 24.1 - 24.4}
\3 Maxwell's Equations
\3 Electromagnetic wave production
\3 Electromagnetic spectrum and energy
\2 Geometric optics - \textbf{Chapters 25.1 - 25.3, 25.6}
\3 Ray-tracing
\3 Reflection
\3 Refraction
\3 Lens optics
\2 Wave optics - \textbf{Chapters 27.1 - 27.3}
\3 Wave interferance
\3 Wave diffraction
\3 Double slit experiments
\2 Nuclear physics in medicine - \textbf{32.1 - 32.4}
\3 Diagnostics and medical imaging
\3 Biological effects of ionizing radiation
\3 Therapeutic uses of ionizing radiation
\3 Food irradiation
\1 \textbf{Unit 6:} Second midterm, final project presentations
\2 \textbf{Second midterm on April 25th, 2024.}
\3 Take-home style, covers Units 4 and 5
\2 Final project presentations
\3 Presented via option A or B (see Tab. \ref{tab:grades}).
\3 Given on April 23rd and 25th in class
\end{outline}
\end{document}
