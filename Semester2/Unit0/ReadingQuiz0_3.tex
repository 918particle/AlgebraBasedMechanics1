\documentclass{article}
\usepackage{graphicx}
\usepackage[margin=1.5cm]{geometry}
\usepackage{amsmath}

\begin{document}

\title{Thursday Reading Assessment: Unit 0, Review of 135A}
\author{Prof. Jordan C. Hanson}

\maketitle

\section{Memory Bank}

\begin{itemize}
\item $\vec{p} = m\vec{v}$ ... Definition of momentum involving vectors.
\item $\vec{p}_{\rm tot,i} = \vec{p}_{\rm tot,f}$ ... Conservation of momentum.
\end{itemize}

\section{Warm-Up Exercises}

\begin{enumerate}
\item Suppose a physical therapy patient is asked to shove a medicine ball forward off the edge of a table to help rebuild the strength of their shoulders.  The medicine ball weighs 7 kg.  (a) If the patient is able to give the ball a speed of 1 m s$^{-1}$, what is the momentum of the ball? (b) If the patient gives the ball the same momentum by rolling it, and it strikes elastically a ball with a mass of 3.5 kg, what will be the velocity of the second ball? \\ \vspace{1.5cm}
\item (a) Estimate the area of our classroom, in m$^2$. (b) Estimate the volume of our classroom, in m$^3$.  (c) If 100 people entered this classroom, how much volume would each person have? \\ \vspace{1.5cm}
\item Perform the following unit conversions:
\begin{itemize}
\item Convert 120 cm to m:
\item Convert 500 cm$^2$ to m$^2$:
\item One ``atmosphere'' of pressure, or 1 atm, is equal to 101325 Pascals, or Pa.  A Pascal is defined as 1 N m$^{-2}$.  Convert 610 Pa to atm. (This is roughly the air pressure on Mars). \\ \vspace{1cm}
\end{itemize}
\item Let $\vec{x} = 0.5 \hat{i} - 0.5\hat{j}$, and $\vec{y} = -0.5\hat{i} + 0.5\hat{j}$. (a) Calculate $\vec{x} + \vec{y}$.  (b) Calculate $\vec{x} - \vec{y}$.
\end{enumerate}

\end{document}
