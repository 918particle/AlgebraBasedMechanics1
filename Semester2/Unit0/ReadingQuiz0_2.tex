\documentclass{article}
\usepackage{graphicx}
\usepackage[margin=1.5cm]{geometry}
\usepackage{amsmath}

\begin{document}

\title{Wednesday Reading Assessment: Unit 0, Review of 135A}
\author{Prof. Jordan C. Hanson}

\maketitle

\section{Memory Bank}

\begin{itemize}
\item $\vec{v}_{\rm ave} = \Delta\vec{x}/\Delta t$ ... Definition of average velocity involving vectors.
\item $x(t) = \frac{1}{2}at^2 + v_{\rm i} t + x_{\rm i}$ ... One-dimensional displacement with constant acceleration.
\item $v(t) = v_{\rm i} + a t$ ... One-dimensional velocity with constant acceleration. 
\item $\vec{F}_{\rm net} = m\vec{a}$ ... Newton's 2nd Law.
\item $W = \vec{F} \cdot \vec{x}$ ... Definition of work involving vectors.
\item $W = Fx\cos\theta$ ... Definition of work, with $\theta$ as angle between force and displacement.
\item $KE = \frac{1}{2} m v^2$ ... Kinetic energy.
\item $W = \Delta KE = KE_{\rm f} - KE_{\rm i}$ ... Work energy theorem.
\item $KE_{\rm i} + PE_{\rm i} = KE_{\rm f} + PE_{\rm f}$ ... Energy conservation.
\item $\vec{p} = m\vec{v}$ ... Definition of momentum involving vectors.
\item $\vec{p}_{\rm tot,i} = \vec{p}_{\rm tot,f}$ ... Conservation of momentum.
\item $\tau = I\alpha$ ... Netwon's 2nd Law for rotating objects, with torque, moment of inertia, and angular acceleration.
\item $KE_{\rm rot} = \frac{1}{2}I \omega^2$ ... Rotational kinetic energy.
\item $L = I \omega$ ... Angular momentum.
\end{itemize}

\clearpage

\section{Warm-Up Exercises}

\begin{enumerate}
\item Suppose a ship sails at 20 km per hour for 3 hours to the West, and then at 20 km per hour for 2 hours to the South.  (a) Assuming that the ship starts at the origin of a 2D coordinate system, what is the final location of the ship? (b) What is the average velocity? \\ \vspace{2.5cm}
\item Suppose an athlete starts a race from rest at $t=0$ with a constant acceleration of $a = 2.5$ m s$^{-2}$.  (a) How long before the speed of the runner is 5 m s$^{-1}$? (b) If the runner has a mass of 60 kg, what is the kinetic energy? (c) How much work or energy was required to reach this velocity? \\ \vspace{2.5cm}
\item Suppose two children each pull a toy in opposite directions.  One pulls to the right with a force of 10 N, while the other pulls to the left with a force of 8 N.  The toy weighs 0.5 kg. (a) What is the magnitude and direction of the acceleration of the toy? (b) If the toy begins at rest, where is the toy after 2 seconds? (c) If the kids drop this same toy from a height of 10 meters, what will be the final velocity just before it hits the ground? \\ \vspace{2.5cm}
\item Suppose a physical therapy patient is asked to shove a medicine ball forward off the edge of a table to help rebuild the strength of their shoulders.  The medicine ball weighs 7 kg.  (a) If the patient is able to give the ball a speed of 1 m s$^{-1}$, what is the momentum of the ball? (b) If the patient gives the ball the same momentum by rolling it, and it strikes elastically a ball with a mass of 3.5 kg, what will be the velocity of the second ball? \\ \vspace{2.5cm}
\item Suppose the medicine ball in the previous problem has a mass of 3.5 kg, and a diameter of 10 cm.  The moment of inertia for a solid sphere is $I = \frac{2}{5} m r^2$. (a) What is the angular momentum of the ball if it is spun at 1 rotation per second? (b) What is the rotational kinetic energy?
\end{enumerate}

\end{document}
