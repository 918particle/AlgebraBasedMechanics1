\documentclass[12pt,twocolumn]{article}
\usepackage[margin=1.5cm]{geometry}
\usepackage{amsmath}
\usepackage{graphicx}
\usepackage{hyperref}
\usepackage{sectsty}
\title{Midterm 1}
\author{Prof. Jordan C. Hanson}
\sectionfont{\fontsize{12}{15}\selectfont}

\begin{document}
\maketitle
\small

\section{Unit 0: Electrostatics I and II}

\noindent
\begin{enumerate}
\item A 50 gram copper wire has a net charge of 2.00 $\mu$C. What fraction of the copper’s electrons has been removed? (Each atom has 29 protons, and the atomic mass is 63.5.) \\ \\
\textit{Divide 50 grams by the atomic weight to obtain the number of moles.  Then, multiply by Avogadro's number to obtain the number of atoms.  Next, multiply by the atomic number (electrons per atom) to obtain total electrons in 50 grams of copper.  Now convert the 2 $\mu$C into number of electrons via the charge of 1 electron.  The ratio of removed electrons to total electrons is on the order of $9\times 10^{-13}$.}
\item A test charge of +2 $\mu$C is placed halfway between a charge of +6 $\mu$C and another of +4 $\mu$C separated by 10 cm. (a) What is the magnitude of the net force on the test charge? (b) What is the direction of this force (away from or toward the +6 $\mu$C charge)? \\ \\
\textit{(a) Draw a free body diagram, and find that the two forces are opposing one another.  With equal distances, the stronger force is just the one generated by the larger charge.  We can show that $F_{\rm Net} = k q \Delta q/r^2$, where $\Delta q$ is the difference between the left and right charges: 2 $\mu$C.  The result is 14.4 N. (b) The force points away from the $q = 6$ $\mu$C charge.}
\item What is the force on the charge located at $x=8.00$ cm in Fig. \ref{fig:e-field_1}(a) given that $q=1.00$ $\mu$C? \\ \\
\textit{Use the same logic as the previous problem, balancing Coulomb forces.  The charge at $x=3$ cm is located 5 cm from the test charge.  The charge at $x=11$ cm is located 3 cm from the test charge.  The 5 cm and 3 cm separations are what enter the denominators of the Coulomb force equations.  Note that the attractive force by the charge on the left points left, while the attractive force by the charge on the right points to the right.  The result is 12.8 N.}
\begin{figure}[hb]
\centering
\includegraphics[width=0.49\textwidth]{e-field_1.jpeg}
\caption{\label{fig:e-field_1} Linear arrangement of charges.}
\end{figure}
\item Find the total electric field at $x=11.00$ cm in Fig. \ref{fig:e-field_1}(b). \\ \\
\textit{Assume $q=1$ $\mu$C.  The superposition principle says the electric fields add like vectors, so the positive test charge experiences a negative electric field from the first charge, and positive electric fields from the right three charges.  The result is strikingly large: $4.1 \times 10^7$ V/m.  This is large enough to break down air and cause a spark.}
\item Determine the direction of the force on $q$ in Fig. \ref{fig:e-field_2}, given that $q_a=q_b=+7.50$ $\mu$C and $q_c = q_d = -7.50$ $\mu$C. (b) Calculate the force on the charge $q$, given that the square is 10.0 cm on a side and $q=2.00$ $\mu$C. \\ \\
\textit{(a) Down.  The positive charges each create an E-field that cancels the horizontal component of the E-field of the other charge.  Similarly, the negative charges each create an E-field that cancels the horizontal component of the E-field of the other negative charge. (b) 76.4 N.}
\begin{figure}
\centering
\includegraphics[width=0.25\textwidth]{e-field_2.jpeg}
\caption{\label{fig:e-field_2} 2D arrangement of charges.}
\end{figure} \vspace{3cm}
\item (a) An evacuated tube uses an accelerating voltage of 40 kV to accelerate electrons to hit a copper plate and produce x rays. Non-relativistically, what would be the maximum speed of these electrons? (b)  Show that units of V/m and N/C for electric field strength are indeed equivalent. \\ \\
\textit{Non-relativistically, we can assume the energy given to the electron by the voltage is $U = qV$, without accounting for the rest mass of the electron or the ``boost factor.''  Set $U = KE = 1/2 m v^2$, and solve for $v$.  We find $\approx 1.2 \times 10^{8}$ m s$^{-1}$.  This is about one third of the speed of light. (b) Start with 1 N/C, then note that 1 J is 1 V/C.  1 J is 1 N m.  Multiply the top and bottom of 1 N/C by 1 m to find 1 N m/C m.  This is 1 J/C m, but 1 J/C is 1 V, so we find 1 V/m.}
\item The electric field strength between two parallel conducting plates separated by 4.00 cm is $7.50 \times 10^4$ V m$^{-1}$. (a) What is $\Delta V$ between the plates? (b) The plate with the lowest potential is taken to be at zero volts. What is the potential 1.00 cm from that plate (and 3.00 cm from the other)? (c) The voltage across a membrane forming a cell wall is 80.0 mV and the membrane is 9.00 nm thick. What is the electric field strength?\footnote{The value is surprisingly large, but correct.} \\ \\
\textit{(a) 3000 Volts.  (b) 750.0 Volts. (c) 8.89 mV/nm, or $8.89 \times 10^{6}$ V m$^{-1}$.}
\item A doubly charged ion is accelerated to an energy of 32.0 keV by the electric field between two parallel conducting plates separated by 2.00 cm. What is the electric field strength between the plates? \\ \\
\textit{Using $\Delta U = q \Delta V$, we find $\Delta V = 16$ kV.  Using $E = \Delta V/\Delta x$, we find $E = 8$ kV cm$^{-1}$, or $8 \times 10^5$ V m$^{-1}$}
\item In one of the classic nuclear physics experiments at the beginning of the 20th century, an alpha particle was accelerated toward a gold nucleus, and its path was substantially deflected by the Coulomb interaction. If the energy of the doubly charged alpha nucleus was 5.00 MeV, how close to the gold nucleus (79 protons) could it come before being deflected? \\ \\
\textit{The key to this problem is energy conservation.  The potential energy of a point charge $q$ in the potential $\Delta V$ of a nucleus $q_N$ is $\Delta U = q \Delta V = q(k q_N/r)$.  Treating $r$ like the distance of closest approach means all of the kinetic energy is converted to potential energy.  Thus, we insert the kinetic energy 5 MeV as $\Delta U$, and calculate $r$.  The result is about $45 \times 10^{-15}$ m, or 45 fm.}
\end{enumerate}

\section{Unit 1: Capacitors, Current, and DC \\ circuits}

\noindent
\begin{enumerate}
\item What capacitance is needed to store 3.00 $\mu$C of charge at a voltage of 120 V? \\ \\ \textit{25 nF.}
\item (a) What is the energy stored in the 10.0 $\mu$F capacitor of a heart defibrillator charged to $9.00\times 10^3$ V? (b) Find the amount of stored charge. (c) In open heart surgery, a much smaller amount of energy will defibrillate the heart.  What voltage is applied to the 8.00 $\mu$F capacitor of a heart defibrillator that stores 40.0 J of energy? (d) Find the amount of stored charge. \\ \\
\textit{(a) 405 J. (b) 0.09 C. (c) 3160 Volts.  (d) 0.025 C.}
\item To build up the charge and energy required in part (a) of the previous problem, an AED designer decides to split the charge among four capacitors in parallel.  Determine the required capacitance of each individual capacitor, and the charge stored on each, if the voltage remains $9.00 \times 10^3$ V.  Why would the designer choose not to connect the capacitors in series? \\ \\
\textit{2.5 microfarads per capacitor, each with 0.0225 C of charge per capacitor.  Connecting them in series would reduce total capacitance.}
\item If a 1.0 mm diameter copper wire can have a resistance of no more than 2.0 $\Omega$, (at 20 degrees C), how long can it be? \\ \\
\textit{Solving $R = \rho L/A$ for the length gives 91.3 m.  The resistivity $\rho$ may be found in chapter 20, table 20.1.}
\item An LED is connected in series with a 1 k$\Omega$ resistor.  A 3.0V battery is connected to the resistor, the LED follows the resistor, and the LED is then connected to ground.  The negative terminal of the battery is also connected to ground.  (a) What current flows from the battery, if the LED resistance is $3$ $\Omega$? (b) How much power is consumed by the LED? (c) How many Coulombs of charge flow through the LED in 10 minutes? \\ \\
\textit{(a) 2.99 mA accounting for the LED resistance, 3 mA without it. (b) About 9 mW.  (c) 1.8 C.}
\end{enumerate}

\section{Unit 2: DC circuits with resistors in \\ series and parallel, RC circuits}

\begin{figure}
\centering
\includegraphics[width=0.4\textwidth,trim=0cm 13.5cm 5.5cm 0cm,clip=true]{parallel.jpeg}
\caption{\label{fig:para} A DC circuit with three resistors.}
\end{figure}
\begin{figure}
\centering
\includegraphics[width=0.13\textwidth,trim=0cm 0.5cm 4cm 0cm,clip=true]{series_batt.jpg}
\includegraphics[width=0.18\textwidth,trim=0cm 0.5cm 5.5cm 0.41cm,clip=true]{parallel_batt.jpg}
\caption{\label{fig:batt} (Left) A DC circuit with two batteries in series, and a resistance R = 0.5 k$\Omega$. (Bottom) A DC circuit with two batteries in parallel, and a resistance R = 0.5 k$\Omega$.}
\end{figure}
\begin{enumerate}
\item In Fig. \ref{fig:para}, let $R_1 = 10$ k$\Omega$, $R_2 = 5$ k$\Omega$, and $R_{\rm tot} = 2$ k$\Omega$.  (a) What is the resistance of $R_3$? (b) If the battery has $\Delta V = 12$ V, what current flows from the battery? (c) What are the individual currents, $I_1$, $I_2$, and $I_3$? \\ \\
\textit{(a) Recall that $R_{\rm tot}^{-1} = R_{1}^{-1} + R_2^{-1} + R_3^{-1}$.  Solving for $R_3$, we find $R_3 = 5$ k$\Omega$. (b) 6 mA.  (c) $I_1 = 1.2$ mA, $I_2 = I_3 = 2.4$ mA.}
\item Consider Fig. \ref{fig:batt}. (a) Assuming no internal resistance, calculate the current and power through the resistance $R$ if each battery has 1.5 V in the series circuit, and 3 V in the parallel circuit. (b) Now repeat part (a) for each circuit, assuming all batteries have an internal resitance of $5$ $\Omega$. \\ \\
\textit{(a) For the series circuit, we find 6 mA and 18 mW.  For the parallel circuit, we find 6 mA, and 18 mW. (b) For the series circuit, we find 5.88 mA, and 17.3 mW.  For the parallel circuit, we find 5.97 mA, and 17.8 mW.}
\item A child's electronic toy is supplied by three 1.58-V alkaline cells having internal resistances of 0.02 $\Omega$ in series with a 1.53-V carbon-zinc dry cell having a 0.10 $\Omega$ internal resistance. The load resistance is 10.00 $\Omega$. (a) Draw a circuit diagram of the toy and its batteries. (b) What current flows? (c) How much power is supplied to the load? (d) What is the internal resistance of the dry cell if it goes bad, resulting in only 0.500 W being supplied to the load? \\ \\
\textit{(a) The diagram should be three identical batteries plus internal resistances, followed by one different battery and resistance, followed by the load resistance, all connected in series. (b) 617 mA.  (c) 3.8 W.  (d) About 18 $\Omega$.}
\item A heart pacemaker fires 72 times a minute, each time a 25.0-nF capacitor is charged (by a battery in series with a resistor) to 0.632 of its full voltage. What is the value of the resistance? \\ \\
\textit{We have to convert the 72 beats per min into the time between beats, which is 60/72 seconds per beat.  Then let $\tau = 60/72$ seconds, and $\tau = RC$.  We find $R = 33$ M$\Omega$.}
\item An ECG monitor must have an RC time constant less than $1.00 \times 10^2$ $\mu$s to be able to measure variations in voltage over small time intervals. If the resistance of the circuit (due mostly to that of the patient’s chest) is 1.00 k$\Omega$, what is the maximum capacitance of the circuit? \\ \\
\textit{Using $\tau = RC$, we find $C_{\rm max} = 0.1$ $\mu$F.}
\end{enumerate}

\section{Unit 3: Magnetism I}
\begin{figure}
\centering
\includegraphics[width=0.4\textwidth,trim=0cm 2.8cm 0cm 0cm,clip=true]{Bfield.jpeg}
\caption{\label{fig:B} Three cases involving the particle velocity $\vec{v}$, and $\vec{B}$ field.}
\end{figure}
\begin{figure}
\centering
\includegraphics[width=0.4\textwidth]{coils.jpeg}
\caption{\label{fig:B2} Three currents that create B-fields.}
\end{figure}

\noindent
\begin{enumerate}
\item Consider Fig. \ref{fig:B}. Fill in Tab. \ref{tab:ex} of directions below for the Lorentz force, assuming a \textbf{negatively charged} particle.  Let $\hat{i}$ represent right, $\hat{j}$ represent up, and $\hat{k}$ represent out of the page.
\begin{table}
\centering
\begin{tabular}{| c | c | c | c |}
\hline
Case & v direction & B direction & F direction \\ \hline
(a) & $-\hat{j}$ & $\hat{k}$ & $\hat{i}$ \\ \hline
(b) & $\hat{j}$ & $\hat{i}$ & $\hat{k}$ \\ \hline
(c) & $\hat{i}$ & $-\hat{k}$ & $-\hat{j}$ \\ \hline
\end{tabular}
\caption{\label{tab:ex} Table of directions for to Fig. \ref{fig:B}. \textbf{Recall the charge is negative.}}
\end{table}
\item An electron moving at $4.00\times 10^3$ m s$^{-1}$ in a 1.25-T magnetic field experiences a magnetic force of $1.40\times 10^{-16}$ N. What angle does the velocity of the electron make with the magnetic field? There are two possible answers. \\ \\
\textit{Using $F = qvB\sin\theta$, inserting the data, and solving for $\theta$, gives $\theta = -10.1$ degrees.}
\item (a) An oxygen-16 ion with a mass of $2.66\times 10^{-26}$ kg travels at $5.00\times 10^6$ m/s perpendicular to a 1.20-T magnetic field, which makes it move in a circular arc with a 0.231-m radius. What positive charge is on the ion? (b) What is the ratio of this charge to the charge of an electron? (c) Discuss why the ratio found in (b) should be an integer. (d) A mass spectrometer is being used to separate common oxygen-16 from the much rarer oxygen-18, taken from a sample of old glacial ice. (The relative abundance of these oxygen isotopes is related to climatic temperature at the time the ice was deposited.) The ratio of the masses of these two ions is 16 to 18.  Assuming the ions have the same charge, what would the radius of the circular arc be for the oxygen-18?  \textit{Hint: this is a scaling problem.} \\ \\
\textit{(a) Setting the Lorentz force equal to the centripetal force and solving for the charge gives $4.8 \times 10^{-19}$ C. (b) The charge is equivalent to 3 electrons. (c) The ratio should be the integer 3, because three electrons have been released from the oxygen-16 ion, giving it a positive charge. (d) We know that radius is directly proportional to mass in this situation, so 0.231 meters times 18/16 gives 0.260 meters for the rarer, heavier isotope.}
\item What force is exerted on the water in an MHD drive utilizing a 25.0-cm-diameter tube, if 100-A current is passed across the tube that is perpendicular to a 2.00-T magnetic field? (The relatively small size of this force indicates the need for very large currents and magnetic fields to make practical MHD drives.) \\ \\
\textit{50 N, using $F = ILB$.}
\item For this exercise, we are designing an electric motor.  Calculate the B-field strength needed on a 200-turn square loop 20.0 cm on a side to create a maximum torque of 300 N m if the loop has 25.0 A of current.  \\ \\
\textit{The relevant formula for the maximum torque is $\tau_{\rm max} = NIAB$, where $N$ is the number of loops, $I$ is the current, $A$ is the area of the loops, and $B$ is the magnetic field strength.  The result is $B = 1.5$ T.}
\item Consider Fig. \ref{fig:B2}.  Fill in the B-field directions in Tab. \ref{tab:ex2} using Amp\`{e}re's Law, and the appropriate right-hand rule. Let $\hat{i}$ represent right, $\hat{j}$ represent up, and $\hat{k}$ represent out of the page.
\begin{table}
\centering
\begin{tabular}{| c | c |}
\hline
Case & B direction \\ \hline
(a) & $\hat{k}$ \\ \hline
(b) & $\hat{i}$ \\ \hline
(c) & $-\hat{i}$ \\ \hline
\end{tabular}
\caption{\label{tab:ex2} Table of directions for to Fig. \ref{fig:B2}. \textbf{Recall the right hand rule for the B-field near a wire.}}
\end{table}
\item Calculate the size of the magnetic field 20 m below a high voltage power line. The line carries 450 MW at a voltage of 300,000 V. \\ \\
\textit{The formula for the B-field near a long straight wire is $B = \mu_0 I/2\pi r$, where $\mu_0$ is the magnetic constant, $I$ is the current, and $r$ is the distance from the wire.  Inserting data and solving for the field gives $0.15 \times 10^{-4}$ T, or 0.15 Gauss.  To find the current, use $P = IV$.}
\item The B-field in the tokamak reactor in Fig. \ref{fig:B3} is given by $B = \mu_0 N I / (2\pi r)$, where $N$ is the total number of loops, $I$ is the current, and $r$ is the radius at which we evaluate the B-field.  (a) Design your own reactor by specifying practical values of $N$, $I$, and $r$ that achieve a 1.0 T B-field at a radius of 5.0 m.  (b) With what frequency will a proton circle the toroidal B-field (1.0 T)?  \textit{These ideas will help activate fusion reactors.} \\ \\
\textit{I chose to solve the given formula for I, and I chose $N = 10^5$ and $r=5$ meters.  This gives a current of 250 Amps.  This is a difficult but achievable configuration. (b) Set the centripetal force equal to the Lorentz force, with $F_{\rm C} = m r \omega^2$.  Suppose the velocity is 1 percent the speed of light.  Given the mass and charge of a proton, and knowing that $\omega = 2\pi f$, we find f = 7.6 MHz.}
\end{enumerate}
\begin{figure}[hb]
\centering
\includegraphics[width=0.33\textwidth]{tokamak.png}
\caption{\label{fig:B3} A generalized diagram of a tokamak.}
\end{figure}
\end{document}