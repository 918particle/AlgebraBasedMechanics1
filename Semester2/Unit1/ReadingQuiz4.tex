\documentclass{article}
\usepackage{graphicx}
\usepackage[margin=1.5cm]{geometry}
\usepackage{amsmath}

\begin{document}

\title{Thursday Reading Assessment: Unit 1, DC Circuits}
\author{Prof. Jordan C. Hanson}

\maketitle

\section{Memory Bank}

\begin{itemize}
\item $C = A\epsilon_0/d$ ... Capacitance of a parallel-plate capacitor.
\item $\epsilon_0 = 8.85 \times 10^{-12}$ F/m
\item $C_{tot}^{-1} = C_1^{-1} + C_2^{-1}$ ... Total capacitance of two capacitors in series.
\item $I = \Delta Q/\Delta t$ ... Definition of current.
\item $\Delta V = I R$ ... Ohm's Law, relating resistance $R$, current $I$, and voltage $\Delta V$.
\end{itemize}

\section{Capacitors and Capacitance}

\begin{enumerate}
\item Suppose you have a parallel plate capacitor with $A = 1$ mm$^2$, and $d = 0.1$ mm.  What is the capacitance? \\ \vspace{0.5cm}
\item Now suppose two identical capacitors from the previous problem are added \textit{in series.}  What is the total capacitance? \\ \vspace{0.5cm}
\item Which of the following should be the formula for the capacitance of a coaxial cable of length $L$?  The inner and outer radii are $R_1$ and $R_2$, respectively.  \textit{Think about the units of $\epsilon_0$.}
\begin{itemize}
\item A: $C = (2\pi\epsilon_0 R_1 L)/\ln(R_2/R_1)$.
\item B: $C = (2\pi\epsilon_0 L)/\ln(R_2/R_1)$.
\item C: $C = (2\pi\epsilon_0 R_1 R_2 L)/\ln(R_2/R_1)$.
\item D: $C = (2\pi\epsilon_0 L)$.
\end{itemize}
\end{enumerate}

\section{Current, Resistance, and Ohm's Law}

\begin{enumerate}
\item The definition of one ``amp'' of current is 1 Coulomb per second.  How many milliamps of current correspond to $1$ $\mu$C of charge flowing down a 10 cm wire in 1 hour? \\ \vspace{1cm}
\item Suppose a 12 V battery drives 100 mA of current through a resistor with resistance $R$.  Calculate the value of $R$. \\ \vspace{1cm}
\item If $R$ in the system in the previous exercise was doubled, and $\Delta V$ kept constant, what would be the new current?
\end{enumerate}

\end{document}
