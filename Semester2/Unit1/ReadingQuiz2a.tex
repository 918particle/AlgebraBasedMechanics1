\documentclass{article}
\usepackage{graphicx}
\usepackage[margin=1.5cm]{geometry}
\usepackage{amsmath}

\begin{document}

\title{Thursday Reading Assessment: Review of Electric Fields and Scaling Problems}
\author{Prof. Jordan C. Hanson}

\maketitle

\section{Memory Bank}

\begin{itemize}
\item $\vec{E} = kq/r^2 \hat{r}$ ...The electric field of a point charge. 
\end{itemize}

\section{Balancing Electric Fields}

\begin{enumerate}
\item Suppose a charge $q_1$ is positive and located at the origin.  Suppose another charge $-q_2$ is negative, and located a distance $x_0$ to the right of the origin.  Where can a third charge be placed such that the net force on it is zero? (Assume $q_1 = 5\mu$C and $-q_2 = -3\mu$C, and $x_0 = 0.25$).\\ \vspace{4cm}
\item Suppose we observe a charge $2q$ from a distance $r$.  Let the E-field observed be $E_0$.  Which of the following is the new E-field if we change the distance to $2 r$, but the charge changes to $4 r$?
\begin{itemize}
\item A: $\frac{1}{2} E_0$
\item B: $ E_0$
\item C: $2 E_0$
\item D: $\frac{3}{2} E_0$
\end{itemize}
\end{enumerate}

\end{document}
