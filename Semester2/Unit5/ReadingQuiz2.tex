\documentclass{article}
\usepackage{graphicx}
\usepackage[margin=1.5cm]{geometry}
\usepackage{amsmath}

\begin{document}

\title{Tuesday Reading Assessment: Unit 5, Electromagnetism}
\author{Prof. Jordan C. Hanson}

\maketitle

\section{Memory Bank}

\begin{itemize}
\item $c/n = f \lambda$ ... Relationship between the speed of light, $c$, the index of refraction, $n$, the frequency of an electromagnetic wave, $f$, and its wavelength, $\lambda$.
\item $c = 3.0 \times 10^8$ m s$^{-1}$ ... The speed of light in a vacuum, within 1 percent error.
\item $\sin(x)\cos(y) = \frac{1}{2}\left(\sin(x+y) + \sin(x-y)\right)$ ... Trigonometric identity.
\item $\Delta x = v \Delta t$ ... The relationship between displacement, constant velocity, and time duration.
\end{itemize}

\section{Electromagnetic Spectrum and Waves}

\begin{enumerate}
\item The frequency of a stream of monochromatic optical photons is $4.3 \times 10^{14}$ Hz.  (a) What is the wavelength of this light in air ($n \approx 1$)?  (b) What is the wavelength in glass, with $n = 1.5$? \\ \vspace{2cm}
\item Suppose a plane is flying 500 meters above an ice shelf in Antarctica.  A radar pulse is fired directly downwards at the ice.  The signal enteres the ice, travels for another 500 meters, reflects from the ocean, and returns to the plane.  How long before the signal returns?  The index of refraction for RF (radio-frequency) waves in ice is 1.78, but it is 1.0 in air. \\ \vspace{2cm}
\item Verify the trigonometric identity in the Memory Bank for $x = \pi/2$, $y = 0$, and $x = 0$, $y = \pi/2$.
\end{enumerate}

\end{document}
