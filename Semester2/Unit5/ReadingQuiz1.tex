\documentclass{article}
\usepackage{graphicx}
\usepackage[margin=1.5cm]{geometry}
\usepackage{amsmath}

\begin{document}

\title{Wednesday Reading Assessment: Unit 5, Electromagnetism}
\author{Prof. Jordan C. Hanson}

\maketitle

\section{Memory Bank}

\begin{itemize}
\item $c = f\lambda$ ... The relationship between speed of light, $c$, the frequency, $f$, and the wavelength, $\lambda$.
\item $c = 299,792,458$ m s$^{-1}$, or $c \approx 0.3$ m ns$^{-1}$.
\item $\bar{S} = E^2/(2c\mu_0)$ ... Average \textbf{intensity}, or W m$^{-2}$, of electromagnetic energy.
\end{itemize}

\section{Electromagnetic Waves and Energy}

\begin{enumerate}
\item (a) What is the wavelength of a 1 GHz radio wave? (b) What is the frequency of a 900 MHz cell phone signal? (c) What is the speed of light in ice, if we observe a frequency of 100 MHz, and a wavelength of 1.69 m? \\ \vspace{2cm}
\item In part (c) of the previous exercise, what is the ratio of the speed of light in ice to the speed found in the memory bank? \textit{This is known as the index of refraction.}\\ \vspace{2cm}
\item What is the intensity of a wave created by an E-field with magnitude 1000 V m$^{-1}$ at the source? \\ \vspace{2cm}
\item How do you think the previous result scales with distance from the source?
\end{enumerate}

\end{document}
