\documentclass{beamer}
\usetheme{metropolis}
\usepackage{graphicx}
\usepackage{subfig}
\usepackage{tcolorbox}
\usepackage{mathtools}
\title{Algebra-Based Physics-2: Electricity, Magnetism, and Modern Physics (PHYS135B-01): Unit 5}
\author{Jordan Hanson}
\institute{Whittier College Department of Physics and Astronomy}

\begin{document}
\maketitle

\section{Summary}

\begin{frame}{Unit 5 Summary}
\begin{enumerate}
\item Electromagnetic waves - \textbf{Chapters 24.1 - 24.4}
\begin{itemize}
\item Maxwell's Equations
\item Electromagnetic wave production
\item Electromagnetic spectrum and energy
\end{itemize}
\item Geometric optics - \textbf{Chapters 25.1 - 25.3, 25.6}
\begin{itemize}
\item Ray-tracing
\item Reflection
\item Refraction
\item Lens optics
\end{itemize}
\end{enumerate}
\end{frame}

\begin{frame}{Unit 5 Summary}
\begin{enumerate}
\item Wave optics - \textbf{Chapters 27.1 - 27.3}
\begin{itemize}
\item Wave interferance
\item Wave diffraction
\item Double slit experiments
\end{itemize}
\item Nuclear physics in medicine - \textbf{32.1 - 32.4}
\begin{itemize}
\item Diagnostics and medical imaging
\item Biological effects of ionizing radiation
\item Therapeutic uses of ionizing radiation
\item Food irradiation
\end{itemize}
\end{enumerate}
\end{frame}

\section{Electromagnetic waves: Maxwell's Equations}

\begin{frame}{Electromagnetic waves: Maxwell's Equations}
\textbf{\alert{Maxwell's Equations}} are a set of four main ideas describing \textit{the entirety} of electromagnetism.
\begin{itemize}
\item \textbf{Electric fields}, or the force per unit test charge, originate on positive charges and terminate on negative charges. The force is related to the permittivity of free space, $\epsilon_0$.  Electric fields give rise to Coulomb's law, or Gauss's law for electricity.
\item \textbf{Magnetic fields}, or the force per unit test current per unit length, are continuous, having no beginning or end. No magnetic monopoles are known to exist.  The strength of the magnetic force is related to the permeability of free space, $\mu_0$.  Magnetic fields give rise to Gauss's law for magnetism.
\end{itemize}
\end{frame}

\begin{frame}{Electromagnetic waves: Maxwell's Equations}
\textbf{\alert{Maxwell's Equations}} are a set of four main ideas describing \textit{the entirety} of electromagnetism.
\begin{itemize}
\item A \textit{changing} \textbf{magnetic field} induces an electromotive force (emf) and, hence, an \textbf{electric field}. The direction of the emf opposes the change.  This is Faraday's law of induction, and includes Lenz's law.
\item \textbf{Magnetic fields} are generated by \textit{moving charges} (current) or by \textit{changing} \textbf{electric fields}. This is Amp\`{e}re's law, enhanced with the idea that changing \textbf{electric fields} without current or charge induce \textbf{magnetic fields}.
\end{itemize}
\end{frame}

\begin{frame}{Electromagnetic waves: Maxwell's Equations}
Amp\`{e}re's Law is enhanced with the idea that changing \textbf{electric fields} without current or charge induce \textbf{magnetic fields}.
\begin{figure}
\centering
\includegraphics[width=0.275\textwidth]{figures/displacement_current.png}
\caption{\label{fig:disp_current} When a current creates a B-field, which surface bounding the B-field line is relevant?}
\end{figure}
\end{frame}

\begin{frame}{Electromagnetic waves: Maxwell's Equations}
\small
The addition to Amp\`{e}re's Law is called \textit{the displacement current}:
\begin{equation}
I_{\rm d} = \epsilon_0 \frac{\Delta \Phi_{\rm E}}{\Delta t}
\end{equation}
The electric flux is $\Phi_{\rm E} = \vec{E} \cdot \vec{A}$.  Assume we are dealing with surface $S_2$, meaning $I = 0$.  Amp\`{e}re's Law gives
\begin{align}
B 2\pi r &= \mu_0\left(I + I_{\rm d}\right) \\
B 2\pi r &= \mu_0\left(0 + I_{\rm d}\right) \\
B 2\pi r &= \mu_0\left(\epsilon_0 \frac{\Delta \Phi_E}{\Delta t} \right) = \mu_0\epsilon_0 A\left(\frac{\Delta E}{\Delta t}\right)
\end{align}
For a parallel-plate capacitor,
\begin{equation}
E = V/d = Q/(Cd) = Qd/(\epsilon_0 A d) = Q/(\epsilon_0 A)
\end{equation}
\end{frame}

\begin{frame}{Electromagnetic waves: Maxwell's Equations}
Insert the magnitude of the E-field into Amp\`{e}re's Law to find:
\begin{align}
B 2\pi r &= \mu_0\epsilon_0 A\left(\frac{\Delta E}{\Delta t}\right) \\
B 2\pi r &= \mu_0\epsilon_0 A \left(\frac{\Delta E}{\epsilon_0 A \Delta t}\right) \\
B &= \frac{\mu_0}{2\pi r} \frac{\Delta Q}{\Delta t} \\
\Aboxed{ B &= \frac{\mu_0 I}{2\pi r} }
\end{align}
A \textit{changing} \textbf{\alert{E-field}} is responsible for the B-field of a capacitor.
\end{frame}

\begin{frame}{Electromagnetic waves: Maxwell's Equations}
What is the B-field generated 1 cm laterally from a capacitor in an RC circuit that charges from 0 to 10 nJ in 1 $\mu$s?
\begin{itemize}
\item A: 20 $\mu$T
\item B: 20 mT
\item C: 20 nT
\item D: 20 pT
\end{itemize}
\end{frame}

\begin{frame}{Electromagnetic waves: Maxwell's Equations}
What is the energy stored in the capacitor, if the capacitance is $C = 10$ pF?
\begin{itemize}
\item A: 5 $\mu$J
\item B: 5 mJ
\item C: 5 nJ
\item D: 5 pJ
\end{itemize}
\small
Recall that $U = \frac{1}{2} \frac{Q^2}{C}$.
\end{frame}

\begin{frame}{Electromagnetic waves: Maxwell's Equations}
\small
But \textit{where} is the energy stored in a capacitor?  \textbf{The E-field.}  Consider that we proved the stored energy is
\begin{equation}
U_{\rm C} = \frac{1}{2} C V^2
\end{equation}
The voltage only exists because of the arrangement of charges and the field, and we know that $V = E d$.  Also, the volume is $Ad$.  Thus,
\begin{align}
U_{\rm C} &= \frac{1}{2}C E^2 d^2 \\
U_{\rm C} &= \frac{1}{2}\left(\frac{\epsilon_0 A}{d}\right) E^2 d^2 \\
\frac{U_{\rm C}}{Ad} &= \frac{1}{2}\epsilon_0 E^2 \\
\Aboxed{\epsilon_{\rm C} &= \frac{1}{2}\epsilon_0 E^2}
\end{align}
\end{frame}

\begin{frame}{Electromagnetic waves: Maxwell's Equations}
Amp\`{e}re's Law is enhanced with the idea that changing \textbf{electric fields} without current or charge induce \textbf{magnetic fields}.
\begin{figure}
\centering
\includegraphics[width=0.275\textwidth]{figures/displacement_current.png}
\caption{\label{fig:disp_current2} When a current creates a B-field, which surface bounding the B-field line is relevant?}
\end{figure}
\end{frame}

\begin{frame}{Electromagnetic waves: Maxwell's Equations}
What if there was a solenoid inductor $(N = 1)$ next to the capacitor, waiting to catch the B-field and become charged (via Faraday's Law)?  The solenoid will produce some current $I$ to create the \textit{opposite} B-field:
\begin{align}
L &= \frac{\mu_0 N^2 A}{d} \\
U_{\rm L} &= \frac{1}{2} L I^2 \\
U_{\rm L} &= \frac{1}{2} \frac{\mu_0 A}{d} I^2 \\
B &= \mu_0 \frac{N}{d} I = \mu_0 \frac{I}{d} \\
I^2 &= \frac{d^2B^2}{\mu_0^2}
\end{align}
\end{frame}

\begin{frame}{Electromagnetic waves: Maxwell's Equations}
What if there was a solenoid inductor $(N = 1)$ next to the capacitor, waiting to catch the B-field and become charged (via Faraday's Law)?  The solenoid will produce some current $I$ to create the \textit{opposite} B-field:
\begin{align}
U_{\rm L} &= \frac{1}{2}\frac{\mu_0 A}{d}\frac{d^2B^2}{\mu_0^2} = \frac{1}{2}\frac{B^2 Ad}{\mu_0} \\
\frac{U_{\rm L}}{Ad} &= \frac{B^2}{2\mu_0} \\
\Aboxed{\epsilon_{\rm L} &= \frac{1}{2\mu_0} B^2}
\end{align}
Suppose the inductor catches \textit{all} the energy from the capacitor, so that $\epsilon_{\rm C} = \epsilon_{\rm L}$?
\end{frame}

\begin{frame}{Electromagnetic waves: Maxwell's Equations}
\small
If that is true, then
\begin{align}
\epsilon_{\rm C} &= \epsilon_{\rm L} \\
\frac{1}{2}\epsilon_0 E^2 &= \frac{1}{2\mu_0} B^2 \\
\frac{E}{B} &= \frac{1}{\sqrt{\epsilon_0 \mu_0}}
\end{align}
Show that the units of $E/B$ are m s$^{-1}$.  \textit{Hint: recall $F = qE$, and $F = qvB$.}  Knowing that the ratio on the left hand side is a velocity:
\begin{equation}
\boxed{v = \frac{1}{\sqrt{\epsilon_0 \mu_0}}} \label{eq:c}
\end{equation}
Equation \ref{eq:c} represents the \textbf{\alert{speed of light.}}  Now imagine the inductor charging a second capacitor, and that capacitor charging some second inductor ... the energy starts to propagate.
\end{frame}

\begin{frame}{Electromagnetic waves: Maxwell's Equations}
\textbf{\alert{We should be able to observe}} this effect in the lab.  
\begin{figure}
\centering
\includegraphics[width=0.7\textwidth]{figures/spark.png}
\caption{\label{fig:spark} Heinrich Hertz demonstrated the \textit{spark gap} RLC circuit.}
\end{figure}
\footnotesize
\begin{itemize}
\item The RLC circuit on the left side is set to resonate.
\item The transformer changes the signal to a high voltage that makes a spark in loop 1.
\item The RLC circuit in the tuner is set to the same resonance frequency.
\item Sparks are induced \textit{even though the circuits are not connected with conductors.}
\end{itemize}
\end{frame}

\begin{frame}{Electromagnetic waves: Maxwell's Equations}
\begin{figure}
\centering
\includegraphics[width=0.5\textwidth]{figures/spark.png}
\caption{\label{fig:spark2} The transmitter and receiver are connected to RLC circuits with the same resonance frequency.}
\end{figure}
If the transmitter and receiver resonance frequency are the same:
\begin{align}
f_{\rm L} &= f_{\rm R} \\
\frac{1}{2\pi \sqrt{L_1 C_1}} &= \frac{1}{2\pi \sqrt{L_2 C_2}} \\
L_1 C_1 &= L_2 C_2 
\end{align}
\end{frame}

\begin{frame}{Electromagnetic waves: Maxwell's Equations}
If the transmitter and receiver resonance frequency are the same:
\begin{equation}
L_{\rm TX} C_{\rm TX} = L_{\rm RX} C_{\rm RX}
\end{equation}
If the transmitter (TX) inductance is 1 mH, and the TX capacitance is 0.1 mF, and the receiver (RX) capacitance is 10 mF, what is the RX inductance?
\begin{itemize}
\item A: 1 mH
\item B: 0.1 mH
\item C: 0.01 mH
\item D: 0.001 mH
\end{itemize}
\footnotesize
Hint: treat this as a scaling problem.
\end{frame}

\begin{frame}{Electromagnetic waves: Maxwell's Equations}
If the transmitter (TX) inductance is 1 mH, and the TX capacitance is 0.1 mF, and the receiver (RX) capacitance is 0.2 mF, what is the RX inductance?
\begin{itemize}
\item A: 5 mH
\item B: 0.5 mH
\item C: 0.05 mH
\item D: 0.005 mH
\end{itemize}
\footnotesize
Hint: treat this as a scaling problem.
\end{frame}

\begin{frame}{Electromagnetic waves: Maxwell's Equations}
\small
That electromagnetic fields can \textit{propagate} was strong evidence that they are wavelike.  All waves that obey the ``wave equation'' share a relationship between the speed, $v$, frequency $f$, and the \textit{wavelength} $\lambda$:
\begin{equation}
v = f\lambda
\end{equation}
The wavelength is the displacement between wave peaks, and $1/f = T$ is the period in time between peaks.  If the speed is $v = 1/\sqrt{\epsilon_0 \mu_0}$, and the resonance frequency corresponds to a capacitance of 0.2 $\mu$F and inductance of 0.5 $\mu$H, what is the wavelength?
\begin{itemize}
\item A: 3750 m
\item B: 375 m
\item C: 37.5 m
\item D: 3.75 m
\end{itemize}
\end{frame}

\begin{frame}{Electromagnetic waves: Maxwell's Equations}
If the speed of light is $3 \times 10^{8}$ m/s, what is this same speed in m/ns?
\begin{itemize}
\item A: 30 m/ns
\item B: 3 m/ns
\item C: 0.3 m/ns
\item D: 0.03 m/ns
\end{itemize}
\end{frame}

\begin{frame}{Electromagnetic waves: Maxwell's Equations}
What is the frequency of electromagnetic radiation with a wavelength comparable to the length of a person ($\approx 1$ m)?
\begin{itemize}
\item A: 0.3 GHz
\item B: 3 GHz
\item C: 300 MHz
\item D: 3000 MHz
\end{itemize}
\footnotesize
Note: is there any reason to expect limitations on the wavelengths and frequencies of electromagnetic waves?
\end{frame}

\section{Electromagnetic waves: Electromagnetic wave production}

\begin{frame}{Electromagnetic waves: Electromagnetic wave production}
\begin{figure}
\centering
\includegraphics[width=0.55\textwidth]{figures/gen1.png}
\caption{\label{fig:gen1} An AC voltage source corresponds to electrons oscilatting, which leads to an oscilatting field.}
\end{figure}
\end{frame}

\begin{frame}{Electromagnetic waves: Electromagnetic wave production}
\begin{figure}
\centering
\includegraphics[width=0.9\textwidth]{figures/gen2.png}
\caption{\label{fig:gen2} The oscillating E-field generates an orthogonal B-field.}
\end{figure}
\end{frame}

\begin{frame}{Electromagnetic waves: Electromagnetic wave production}
\begin{figure}
\centering
\includegraphics[width=0.75\textwidth]{figures/gen3.png}
\caption{\label{fig:gen3} The oscillating B-field generates an orthogonal E-field, continuing the process.}
\end{figure}
\end{frame}

\begin{frame}{Electromagnetic waves: Electromagnetic wave production}
The wave \textbf{\alert{moves energy}} in the direction of the green arrow.
\begin{figure}
\centering
\includegraphics[width=0.35\textwidth]{figures/gen3.png}
\caption{\label{fig:gen4} The oscillating B-field generates an orthogonal E-field, continuing the process.}
\end{figure}
The flux of energy per unit area in this case is (after some length mathematics)
\begin{equation}
\vec{S} = \frac{1}{\mu_0} \vec{E} \times \vec{B}
\end{equation}
\end{frame}

\begin{frame}{Electromagnetic waves: Electromagnetic wave production}
The wave \textbf{\alert{moves energy}} in the direction of $\vec{S}$.  The direction is orthogonal to the E and B, with a magnitude $EB/\mu_0$.  The peak B-value is $B = E/c$.  Both B and E are sinusoids with the same $f$ and $\phi$.  Thus, $S \propto \sin^2(2\pi ft +\phi)$, and the average of this is $1/2$.  This makes the average \textbf{\alert{intensity}}
\begin{equation}
\bar{S} = \frac{1}{2c\mu_0}E^2
\end{equation}
The units of intensity are W m$^{-2}$.  This formula is useful:
\begin{itemize}
\item Calculate the RX power at a radio given the field strength at the radio station, and the distance to the radio station.
\item Calculate the brightness of a star observed at Earth, given the field strength of the light at the star.
\end{itemize}
\end{frame}

\begin{frame}{Electromagnetic waves: Electromagnetic wave production}
Suppose a microwave in the kitchen generates 1 kW of power, projected onto a 10cm x 10cm area at a distance of 1 m.  What is the intensity (power per unit area)?
\begin{itemize}
\item C: 1 kW
\item D: 100 kW
\item A: 1 kW m$^{-2}$
\item B: 100 kW m$^{-2}$
\end{itemize}
\end{frame}

\begin{frame}{Electromagnetic waves: Electromagnetic wave production}
Suppose a microwave in the kitchen generates 1 kW of power, projected onto a 10cm x 10cm area at a distance of 1 m.  How long does it take the energy to travel the 1 meter?
\begin{itemize}
\item C: 0.333 ns
\item D: 3.33 ns
\item A: 33.3 ns
\item B: 333 ns
\end{itemize}
\end{frame}

\begin{frame}{Electromagnetic waves: Electromagnetic wave production}
Suppose a microwave in the kitchen generates 1 kW of power, projected onto a 10cm x 10cm area at a distance of 1 m.  What is the peak E-field at the source?
\begin{itemize}
\item C: 870 V/m
\item D: 8700 V
\item A: 8700 V/m
\item B: 870 V
\end{itemize}
\end{frame}

\section{Electromagnetic waves: Electromagnetic spectrum and energy}

\section{Geometric optics: Ray-tracing}

\section{Geometric optics: Reflection}

\section{Geometric optics: Refraction}

\section{Geometric optics: Lens optics}

\section{Wave optics: Wave interferance}

\section{Wave optics: Wave diffraction}

\section{Wave optics: Double slit experiments}

\section{Nuclear physics in medicine: Diagnostics and medical imaging}

\section{Nuclear physics in medicine: Biological effects of ionizing radiation}

\section{Nuclear physics in medicine: Therapeutic uses of ionizing radiation}

\section{Nuclear physics in medicine: Food irradiation}

\section{Conclusion}

\begin{frame}{Unit 5 Summary}
\begin{enumerate}
\item Electromagnetic waves - \textbf{Chapters 24.1 - 24.4}
\begin{itemize}
\item Maxwell's Equations
\item Electromagnetic wave production
\item Electromagnetic spectrum and energy
\end{itemize}
\item Geometric optics - \textbf{Chapters 25.1 - 25.3, 25.6}
\begin{itemize}
\item Ray-tracing
\item Reflection
\item Refraction
\item Lens optics
\end{itemize}
\end{enumerate}
\end{frame}

\begin{frame}{Unit 5 Summary}
\begin{enumerate}
\item Wave optics - \textbf{Chapters 27.1 - 27.3}
\begin{itemize}
\item Wave interferance
\item Wave diffraction
\item Double slit experiments
\end{itemize}
\item Nuclear physics in medicine - \textbf{32.1 - 32.4}
\begin{itemize}
\item Diagnostics and medical imaging
\item Biological effects of ionizing radiation
\item Therapeutic uses of ionizing radiation
\item Food irradiation
\end{itemize}
\end{enumerate}
\end{frame}

\end{document}
