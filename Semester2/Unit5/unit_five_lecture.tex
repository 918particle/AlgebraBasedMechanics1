\documentclass{beamer}
\usetheme{metropolis}
\usepackage{graphicx}
\usepackage{subfig}
\usepackage{tcolorbox}
\usepackage{amsmath,mathtools}
\title{Algebra-Based Physics-2: Electricity, Magnetism, and Modern Physics (PHYS135B-01): Unit 5}
\author{Jordan Hanson}
\institute{Whittier College Department of Physics and Astronomy}

\begin{document}
\maketitle

\section{Summary}

\begin{frame}{Unit 5 Summary}
\begin{enumerate}
\item Electromagnetic waves - \textbf{Chapters 24.1 - 24.4}
\begin{itemize}
\item Maxwell's Equations
\item Electromagnetic wave production
\item Electromagnetic spectrum and energy
\end{itemize}
\item Geometric optics - \textbf{Chapters 25.1 - 25.3, 25.6}
\begin{itemize}
\item Ray-tracing
\item Reflection
\item Refraction
\item Lens optics
\end{itemize}
\end{enumerate}
\end{frame}

\begin{frame}{Unit 5 Summary}
\begin{enumerate}
\item Wave optics - \textbf{Chapters 27.1 - 27.3}
\begin{itemize}
\item Wave interferance
\item Wave diffraction
\item Double slit experiments
\end{itemize}
\item Nuclear physics in medicine - \textbf{32.1 - 32.4}
\begin{itemize}
\item Diagnostics and medical imaging
\item Biological effects of ionizing radiation
\item Therapeutic uses of ionizing radiation
\item Food irradiation
\end{itemize}
\end{enumerate}
\end{frame}

\section{Electromagnetic waves: Maxwell's Equations}

\begin{frame}{Electromagnetic waves: Maxwell's Equations}
\textbf{\alert{Maxwell's Equations}} are a set of four main ideas describing \textit{the entirety} of electromagnetism.
\begin{itemize}
\item \textbf{Electric fields}, or the force per unit test charge, originate on positive charges and terminate on negative charges. The force is related to the permittivity of free space, $\epsilon_0$.  Electric fields give rise to Coulomb's law, or Gauss's law for electricity.
\item \textbf{Magnetic fields}, or the force per unit test current per unit length, are continuous, having no beginning or end. No magnetic monopoles are known to exist.  The strength of the magnetic force is related to the permeability of free space, $\mu_0$.  Magnetic fields give rise to Gauss's law for magnetism.
\end{itemize}
\end{frame}

\begin{frame}{Electromagnetic waves: Maxwell's Equations}
\textbf{\alert{Maxwell's Equations}} are a set of four main ideas describing \textit{the entirety} of electromagnetism.
\begin{itemize}
\item A \textit{changing} \textbf{magnetic field} induces an electromotive force (emf) and, hence, an \textbf{electric field}. The direction of the emf opposes the change.  This is Faraday's law of induction, and includes Lenz's law.
\item \textbf{Magnetic fields} are generated by \textit{moving charges} (current) or by \textit{changing} \textbf{electric fields}. This is Amp\`{e}re's law, enhanced with the idea that changing \textbf{electric fields} without current or charge induce \textbf{magnetic fields}.
\end{itemize}
\end{frame}

\begin{frame}{Electromagnetic waves: Maxwell's Equations}
Amp\`{e}re's Law is enhanced with the idea that changing \textbf{electric fields} without current or charge induce \textbf{magnetic fields}.
\begin{figure}
\centering
\includegraphics[width=0.275\textwidth]{figures/displacement_current.png}
\caption{\label{fig:disp_current} When a current creates a B-field, which surface bounding the B-field line is relevant?}
\end{figure}
\end{frame}

\begin{frame}{Electromagnetic waves: Maxwell's Equations}
\small
The addition to Amp\`{e}re's Law is called \textit{the displacement current}:
\begin{equation}
I_{\rm d} = \epsilon_0 \frac{\Delta \Phi_{\rm E}}{\Delta t}
\end{equation}
The electric flux is $\Phi_{\rm E} = \vec{E} \cdot \vec{A}$.  Assume we are dealing with surface $S_2$, meaning $I = 0$.  Amp\`{e}re's Law gives
\begin{align}
B 2\pi r &= \mu_0\left(I + I_{\rm d}\right) \\
B 2\pi r &= \mu_0\left(0 + I_{\rm d}\right) \\
B 2\pi r &= \mu_0\left(\epsilon_0 \frac{\Delta \Phi_E}{\Delta t} \right) = \mu_0\epsilon_0 A\left(\frac{\Delta E}{\Delta t}\right)
\end{align}
For a parallel-plate capacitor,
\begin{equation}
E = V/d = Q/(Cd) = Qd/(\epsilon_0 A d) = Q/(\epsilon_0 A)
\end{equation}
\end{frame}

\begin{frame}{Electromagnetic waves: Maxwell's Equations}
Insert the magnitude of the E-field into Amp\`{e}re's Law to find:
\begin{align}
B 2\pi r &= \mu_0\epsilon_0 A\left(\frac{\Delta E}{\Delta t}\right) \\
B 2\pi r &= \mu_0\epsilon_0 A \left(\frac{\Delta E}{\epsilon_0 A \Delta t}\right) \\
B &= \frac{\mu_0}{2\pi r} \frac{\Delta Q}{\Delta t} \\
\Aboxed{ B &= \frac{\mu_0 I}{2\pi r} }
\end{align}
A \textit{changing} \textbf{\alert{E-field}} is responsible for the B-field of a capacitor.
\end{frame}

\begin{frame}{Electromagnetic waves: Maxwell's Equations}
What is the B-field generated 1 cm laterally from a capacitor in an RC circuit that charges from 0 to 10 nJ in 1 $\mu$s?
\begin{itemize}
\item A: 20 $\mu$T
\item B: 20 mT
\item C: 20 nT
\item D: 20 pT
\end{itemize}
\end{frame}

\begin{frame}{Electromagnetic waves: Maxwell's Equations}
What is the energy stored in the capacitor, if the capacitance is $C = 10$ pF?
\begin{itemize}
\item A: 5 $\mu$J
\item B: 5 mJ
\item C: 5 nJ
\item D: 5 pJ
\end{itemize}
\small
Recall that $U = \frac{1}{2} \frac{Q^2}{C}$.
\end{frame}

\begin{frame}{Electromagnetic waves: Maxwell's Equations}
\small
But \textit{where} is the energy stored in a capacitor?  \textbf{The E-field.}  Consider that we proved the stored energy is
\begin{equation}
U_{\rm C} = \frac{1}{2} C V^2
\end{equation}
The voltage only exists because of the arrangement of charges and the field, and we know that $V = E d$.  Also, the volume is $Ad$.  Thus,
\begin{align}
U_{\rm C} &= \frac{1}{2}C E^2 d^2 \\
U_{\rm C} &= \frac{1}{2}\left(\frac{\epsilon_0 A}{d}\right) E^2 d^2 \\
\frac{U_{\rm C}}{Ad} &= \frac{1}{2}\epsilon_0 E^2 \\
\Aboxed{\epsilon_{\rm C} &= \frac{1}{2}\epsilon_0 E^2}
\end{align}
\end{frame}

\begin{frame}{Electromagnetic waves: Maxwell's Equations}
Amp\`{e}re's Law is enhanced with the idea that changing \textbf{electric fields} without current or charge induce \textbf{magnetic fields}.
\begin{figure}
\centering
\includegraphics[width=0.275\textwidth]{figures/displacement_current.png}
\caption{\label{fig:disp_current2} When a current creates a B-field, which surface bounding the B-field line is relevant?}
\end{figure}
\end{frame}

\begin{frame}{Electromagnetic waves: Maxwell's Equations}
What if there was a solenoid inductor $(N = 1)$ next to the capacitor, waiting to catch the B-field and become charged (via Faraday's Law)?  The solenoid will produce some current $I$ to create the \textit{opposite} B-field:
\begin{align}
L &= \frac{\mu_0 N^2 A}{d} \\
U_{\rm L} &= \frac{1}{2} L I^2 \\
U_{\rm L} &= \frac{1}{2} \frac{\mu_0 A}{d} I^2 \\
B &= \mu_0 \frac{N}{d} I = \mu_0 \frac{I}{d} \\
I^2 &= \frac{d^2B^2}{\mu_0^2}
\end{align}
\end{frame}

\begin{frame}{Electromagnetic waves: Maxwell's Equations}
What if there was a solenoid inductor $(N = 1)$ next to the capacitor, waiting to catch the B-field and become charged (via Faraday's Law)?  The solenoid will produce some current $I$ to create the \textit{opposite} B-field:
\begin{align}
U_{\rm L} &= \frac{1}{2}\frac{\mu_0 A}{d}\frac{d^2B^2}{\mu_0^2} = \frac{1}{2}\frac{B^2 Ad}{\mu_0} \\
\frac{U_{\rm L}}{Ad} &= \frac{B^2}{2\mu_0} \\
\Aboxed{\epsilon_{\rm L} &= \frac{1}{2\mu_0} B^2}
\end{align}
Suppose the inductor catches \textit{all} the energy from the capacitor, so that $\epsilon_{\rm C} = \epsilon_{\rm L}$?
\end{frame}

\begin{frame}{Electromagnetic waves: Maxwell's Equations}
\small
If that is true, then
\begin{align}
\epsilon_{\rm C} &= \epsilon_{\rm L} \\
\frac{1}{2}\epsilon_0 E^2 &= \frac{1}{2\mu_0} B^2 \\
\frac{E}{B} &= \frac{1}{\sqrt{\epsilon_0 \mu_0}}
\end{align}
Show that the units of $E/B$ are m s$^{-1}$.  \textit{Hint: recall $F = qE$, and $F = qvB$.}  Knowing that the ratio on the left hand side is a velocity:
\begin{equation}
\boxed{v = \frac{1}{\sqrt{\epsilon_0 \mu_0}}} \label{eq:c}
\end{equation}
Equation \ref{eq:c} represents the \textbf{\alert{speed of light.}}  Now imagine the inductor charging a second capacitor, and that capacitor charging some second inductor ... the energy starts to propagate.
\end{frame}

\begin{frame}{Electromagnetic waves: Maxwell's Equations}
\textbf{\alert{We should be able to observe}} this effect in the lab.  
\begin{figure}
\centering
\includegraphics[width=0.7\textwidth]{figures/spark.png}
\caption{\label{fig:spark} Heinrich Hertz demonstrated the \textit{spark gap} RLC circuit.}
\end{figure}
\footnotesize
\begin{itemize}
\item The RLC circuit on the left side is set to resonate.
\item The transformer changes the signal to a high voltage that makes a spark in loop 1.
\item The RLC circuit in the tuner is set to the same resonance frequency.
\item Sparks are induced \textit{even though the circuits are not connected with conductors.}
\end{itemize}
\end{frame}

\begin{frame}{Electromagnetic waves: Maxwell's Equations}
\begin{figure}
\centering
\includegraphics[width=0.5\textwidth]{figures/spark.png}
\caption{\label{fig:spark2} The transmitter and receiver are connected to RLC circuits with the same resonance frequency.}
\end{figure}
If the transmitter and receiver resonance frequency are the same:
\begin{align}
f_{\rm L} &= f_{\rm R} \\
\frac{1}{2\pi \sqrt{L_1 C_1}} &= \frac{1}{2\pi \sqrt{L_2 C_2}} \\
L_1 C_1 &= L_2 C_2 
\end{align}
\end{frame}

\begin{frame}{Electromagnetic waves: Maxwell's Equations}
If the transmitter and receiver resonance frequency are the same:
\begin{equation}
L_{\rm TX} C_{\rm TX} = L_{\rm RX} C_{\rm RX}
\end{equation}
If the transmitter (TX) inductance is 1 mH, and the TX capacitance is 0.1 mF, and the receiver (RX) capacitance is 10 mF, what is the RX inductance?
\begin{itemize}
\item A: 1 mH
\item B: 0.1 mH
\item C: 0.01 mH
\item D: 0.001 mH
\end{itemize}
\footnotesize
Hint: treat this as a scaling problem.
\end{frame}

\begin{frame}{Electromagnetic waves: Maxwell's Equations}
If the transmitter (TX) inductance is 1 mH, and the TX capacitance is 0.1 mF, and the receiver (RX) capacitance is 0.2 mF, what is the RX inductance?
\begin{itemize}
\item A: 5 mH
\item B: 0.5 mH
\item C: 0.05 mH
\item D: 0.005 mH
\end{itemize}
\footnotesize
Hint: treat this as a scaling problem.
\end{frame}

\begin{frame}{Electromagnetic waves: Maxwell's Equations}
\small
That electromagnetic fields can \textit{propagate} was strong evidence that they are wavelike.  All waves that obey the ``wave equation'' share a relationship between the speed, $v$, frequency $f$, and the \textit{wavelength} $\lambda$:
\begin{equation}
v = f\lambda
\end{equation}
The wavelength is the displacement between wave peaks, and $1/f = T$ is the period in time between peaks.  If the speed is $v = 1/\sqrt{\epsilon_0 \mu_0}$, and the resonance frequency corresponds to a capacitance of 0.2 $\mu$F and inductance of 0.5 $\mu$H, what is the wavelength?
\begin{itemize}
\item A: 3750 m
\item B: 375 m
\item C: 37.5 m
\item D: 3.75 m
\end{itemize}
\end{frame}

\begin{frame}{Electromagnetic waves: Maxwell's Equations}
If the speed of light is $3 \times 10^{8}$ m/s, what is this same speed in m/ns?
\begin{itemize}
\item A: 30 m/ns
\item B: 3 m/ns
\item C: 0.3 m/ns
\item D: 0.03 m/ns
\end{itemize}
\end{frame}

\begin{frame}{Electromagnetic waves: Maxwell's Equations}
What is the frequency of electromagnetic radiation with a wavelength comparable to the length of a person ($\approx 1$ m)?
\begin{itemize}
\item A: 0.3 GHz
\item B: 3 GHz
\item C: 300 MHz
\item D: 3000 MHz
\end{itemize}
\footnotesize
Note: is there any reason to expect limitations on the wavelengths and frequencies of electromagnetic waves?
\end{frame}

\section{Electromagnetic waves: Electromagnetic wave production}

\begin{frame}{Electromagnetic waves: Electromagnetic wave production}
\begin{figure}
\centering
\includegraphics[width=0.55\textwidth]{figures/gen1.png}
\caption{\label{fig:gen1} An AC voltage source corresponds to electrons oscilatting, which leads to an oscilatting field.}
\end{figure}
\end{frame}

\begin{frame}{Electromagnetic waves: Electromagnetic wave production}
\begin{figure}
\centering
\includegraphics[width=0.9\textwidth]{figures/gen2.png}
\caption{\label{fig:gen2} The oscillating E-field generates an orthogonal B-field.}
\end{figure}
\end{frame}

\begin{frame}{Electromagnetic waves: Electromagnetic wave production}
\begin{figure}
\centering
\includegraphics[width=0.75\textwidth]{figures/gen3.png}
\caption{\label{fig:gen3} The oscillating B-field generates an orthogonal E-field, continuing the process.}
\end{figure}
\end{frame}

\begin{frame}{Electromagnetic waves: Electromagnetic wave production}
The wave \textbf{\alert{moves energy}} in the direction of the green arrow.
\begin{figure}
\centering
\includegraphics[width=0.35\textwidth]{figures/gen3.png}
\caption{\label{fig:gen4} The oscillating B-field generates an orthogonal E-field, continuing the process.}
\end{figure}
The flux of energy per unit area in this case is (after some length mathematics)
\begin{equation}
\vec{S} = \frac{1}{\mu_0} \vec{E} \times \vec{B}
\end{equation}
\end{frame}

\begin{frame}{Electromagnetic waves: Electromagnetic wave production}
The wave \textbf{\alert{moves energy}} in the direction of $\vec{S}$.  The direction is orthogonal to the E and B, with a magnitude $EB/\mu_0$.  The peak B-value is $B = E/c$.  Both B and E are sinusoids with the same $f$ and $\phi$.  Thus, $S \propto \sin^2(2\pi ft +\phi)$, and the average of this is $1/2$.  This makes the average \textbf{\alert{intensity}}
\begin{equation}
\bar{S} = \frac{1}{2c\mu_0}E^2
\end{equation}
The units of intensity are W m$^{-2}$.  This formula is useful:
\begin{itemize}
\item Calculate the RX power at a radio given the field strength at the radio station, and the distance to the radio station.
\item Calculate the brightness of a star observed at Earth, given the field strength of the light at the star.
\end{itemize}
\end{frame}

\begin{frame}{Electromagnetic waves: Electromagnetic wave production}
Suppose a microwave in the kitchen generates 1 kW of power, projected onto a 10cm x 10cm area at a distance of 1 m.  What is the intensity (power per unit area)?
\begin{itemize}
\item C: 1 kW
\item D: 100 kW
\item A: 1 kW m$^{-2}$
\item B: 100 kW m$^{-2}$
\end{itemize}
\end{frame}

\begin{frame}{Electromagnetic waves: Electromagnetic wave production}
Suppose a microwave in the kitchen generates 1 kW of power, projected onto a 10cm x 10cm area at a distance of 1 m.  How long does it take the energy to travel the 1 meter?
\begin{itemize}
\item C: 0.333 ns
\item D: 3.33 ns
\item A: 33.3 ns
\item B: 333 ns
\end{itemize}
\end{frame}

\begin{frame}{Electromagnetic waves: Electromagnetic wave production}
Suppose a microwave in the kitchen generates 1 kW of power, projected onto a 10cm x 10cm area at a distance of 1 m.  What is the peak E-field at the source?
\begin{itemize}
\item C: 870 V/m
\item D: 8700 V
\item A: 8700 V/m
\item B: 870 V
\end{itemize}
\end{frame}

\section{Electromagnetic waves: Electromagnetic spectrum and energy}

\begin{frame}{Electromagnetic waves: Electromagnetic spectrum and energy}
\small
\begin{figure}
\centering
\includegraphics[width=0.95\textwidth]{figures/spectrum.png}
\caption{\label{fig:spectrum} \textbf{\alert{The electromagnetic spectrum}} maps signal types to wavelength (top) and frequency (bottom).}
\end{figure}
\begin{itemize}
\item Visible spectrum: more than $10^{14}$ Hz, 400-700 nm wavelengths
\item Radio waves: $[10^{-1} - 10^4]$ MHz
\end{itemize}
\end{frame}

\begin{frame}{Electromagnetic waves: Electromagnetic spectrum and energy}
\small
\textbf{\alert{Amplitude modulation (AM)}} is a technology that allows audio transmission over the EM spectrum.
\begin{figure}
\centering
\includegraphics[width=0.85\textwidth]{figures/AM.png}
\caption{\label{fig:radio} (a) The carrier wave. (b) The audio spectrum. (c) The \textit{modulated} carrier wave.}
\end{figure}
\end{frame}

\begin{frame}{Electromagnetic waves: Electromagnetic spectrum and energy}
\footnotesize
\begin{columns}[T]
\begin{column}{0.5\textwidth}
The carrier is a pure sinusoid at a single frequency, $f_{\rm c}$, with amplitude $A$:
\begin{equation}
c(t) = A\sin(2\pi f_{\rm c} t) \label{eq:carrier}
\end{equation}
Let the modulating audio signal (at a given frequency) be
\begin{equation}
m(t) = Am\cos(2\pi f_{\rm m} t + \phi) \label{eq:modulation}
\end{equation}
Audio waves are not electromagnetic, and audio frequencies are orders of magnitude smaller than radio frequencies.  If there were a way to \textit{mix} (multiply) these signals \textbf{\alert{as voltages}}, then we get
\begin{equation}
y(t) = \left[1+\frac{m(t)}{A}\right]c(t) \label{eq:mix}
\end{equation}
\end{column}
\begin{column}{0.5\textwidth}
Do you remember the following trigonometric identity?
\begin{multline}
\sin(A)\cos(B) = \\ \frac{1}{2}\left(\sin(A+B) + \sin(A-B)\right) \label{eq:ident}
\end{multline}
\textbf{Group exercise:}
Substitute Eqs. \ref{eq:carrier} and \ref{eq:modulation} into \ref{eq:mix}, and use the trigonometric identity in Eq. \ref{eq:ident} to simplify the result.
\begin{enumerate}
\item Look for three waves: the carrier, and two additional ones at two different frequencies.
\item Draw a picture of the spectrum, the amplitude versus frequency of the signal.
\end{enumerate}
\end{column}
\end{columns}
\end{frame}

\begin{frame}{Electromagnetic waves: Electromagnetic spectrum and energy}
\footnotesize
\begin{columns}[T]
\begin{column}{0.5\textwidth}
The AM mixing yields three waves:
\begin{itemize}
\item The original carrier
\item A wave with $f_{\rm c} + f_{\rm m}$
\item A wave with $f_{\rm c} - f_{\rm m}$
\end{itemize}
To re-capture the audio, we must \textit{demodulate}, or reverse the process.  \\ \vspace{0.5cm}\textbf{How do we create} $m(t)$, and how do we modulate and demodulate it?
\begin{itemize}
\item Parallel LC circuits that act as resonators
\item Diodes, devices that allow current to flow only one way
\end{itemize}
\end{column}
\begin{column}{0.5\textwidth}
\begin{figure}
\centering
\includegraphics[width=0.95\textwidth]{figures/AMSpec.pdf}
\includegraphics[width=0.75\textwidth]{figures/AMSpec2.pdf}
\caption{\label{fig:amspec} \footnotesize (Top) An example of a single audio frequency mixed into an AM signal. (Bottom) An audio spectrum mixed into an AM signal.}
\end{figure}
\end{column}
\end{columns}
\end{frame}

\begin{frame}{Electromagnetic waves: Electromagnetic spectrum and energy}
\footnotesize
\begin{columns}[T]
\begin{column}{0.35\textwidth}
\textbf{How do we create} $m(t)$, and how do we modulate and demodulate it?
\begin{itemize}
\item Parallel LC circuits that act as resonators
\item Diodes, devices that allow current to flow only one way
\end{itemize}
\begin{figure}
\centering
\includegraphics[width=0.95\textwidth]{figures/diode.png}
\caption{\label{fig:diode} \footnotesize Circuit diagram for the diode.}
\end{figure}
\end{column}
\begin{column}{0.65\textwidth}
\begin{figure}
\centering
\includegraphics[width=0.95\textwidth]{figures/AMSpec3.pdf}
\caption{\label{fig:amspec2} \footnotesize (Upper left) The audio signal is converted to a voltage via a microphone. (Lower left) The radio carrier signal oscillates at a higher frequency. (Middle) The LC resonator and diode mix the two signals. (Right) The final amplitude is modulated by the audio.}
\end{figure}
\end{column}
\end{columns}
\end{frame}

\begin{frame}{Electromagnetic waves: Electromagnetic spectrum and energy}
Suppose an audio signal exists primarily at 10 kHz, and we need an AM carrier at 1.4 MHz.  If this AM carrier is mixed with the audio signal, what frequencies will exist in the final signal?
\begin{itemize}
\item A: 1390 and 1400 kHz
\item B: 1400 kHz only
\item C: 1390, 1400, and 1410 kHz
\item D: 1390 and 1410 kHz
\end{itemize}
\end{frame}

\begin{frame}{Electromagnetic waves: Electromagnetic spectrum and energy}
Suppose an audio signal exists primarily at 10 kHz, and we need an AM carrier at 1.4 MHz.  If this AM carrier is mixed with the audio signal, what is the \textit{bandwidth} required?  That is, how much of the EM spectrum is occupied by the final signal?
\begin{itemize}
\item A: 10 kHz
\item B: 20 kHz
\item C: 1400 kHz
\item D: 1420 kHz
\end{itemize}
\end{frame}

\begin{frame}{Electromagnetic waves: Electromagnetic spectrum and energy}
Suppose an audio signal exists primarily at 10 kHz, and we need an AM carrier at 1.4 MHz.  If, in our mixer, $L = 10$ $\mu$H is required, what value must we choose for $C$?
\begin{itemize}
\item A: 1.3 mF
\item B: 2.6 $\mu$F
\item C: 1.3 $\mu$F
\item D: 1.3 nF
\end{itemize}
\end{frame}

\begin{frame}{Electromagnetic waves: Electromagnetic spectrum and energy}
\begin{figure}
\centering
\includegraphics[width=0.65\textwidth]{figures/RX.pdf}
\includegraphics[width=0.65\textwidth]{figures/SuperHet.pdf}
\caption{\label{fig:rx} \footnotesize Circuit properties for the AM receiver.}
\end{figure}
\end{frame}

\begin{frame}{Electromagnetic waves: Electromagnetic spectrum and energy}
We can show that the \textit{bandwidth} of the RLC response is
\begin{equation}
\frac{\Delta f}{f_0} = \omega_0 \tau = 2\pi f_0 \tau
\end{equation}
Where $\tau = RC$ and $f_0 = 1/(2\pi \sqrt{LC})$.  This simplifies to
\begin{equation}
\frac{\Delta f}{f_0} = R \sqrt{\frac{C}{L}}
\end{equation}
Thus, bandwidth is proportional to $R$, as shown in Fig. \ref{fig:rx}.
\end{frame}

\begin{frame}{Electromagnetic waves: Electromagnetic spectrum and energy}
Suppose an audio signal exists primarily at 5 kHz, and we need an AM carrier at 1100 kHz.  This implies our sidebands will be at 1095 kHz and 1105 kHz, making our bandwidth 10 kHz centered around 1100 kHz.  If our resistance $R$ is such that we are capturing the carrier, but not the sidebands, we should:
\begin{itemize}
\item A: Decrease $R$
\item B: Increase $R$
\item C: Leave $R$ unchanged 
\item D: Set $R$ to 0 Ohms
\end{itemize}
\end{frame}

\begin{frame}{Electromagnetic waves: Electromagnetic spectrum and energy}
\footnotesize
\begin{columns}[T]
\begin{column}{0.4\textwidth}
The local oscillator (LO) is a tunable oscillator set to be 455 kHz above the AM channel.  For example:
\begin{itemize}
\item AM channel: 1200 kHz
\item LO: 1655 kHz
\item Mixer: 1655 kHz + 1200 kHz, 1655 kHz, and 1655 - 1200 kHz
\item 455 kHz is the \textit{intermediate frequency} (IF)
\item IF amplifier: responds only to 1655 - 1200 = 455 kHz
\end{itemize}
\end{column}
\begin{column}{0.6\textwidth}
\begin{figure}
\centering
\includegraphics[width=\textwidth]{figures/SuperHet2.pdf}
\caption{\label{fig:amspec3} \footnotesize The superheterodyne radio scheme.  Channel input is amplified, mixed with a local oscillator, and moved to the IF.  The IF is filtered and amplified, then demodulated in the detector.}
\end{figure}
\end{column}
\end{columns}
\end{frame}

\begin{frame}{Electromagnetic waves: Electromagnetic spectrum and energy}
Suppose an audio signal exists primarily at 10 kHz, and is mixed with a carrier at 1000 kHz.  What should our LO be if our IF is 455 kHz?
\begin{itemize}
\item A: 1000 kHz
\item B: 455 kHz
\item C: 545 kHz
\item D: 1455 kHz
\end{itemize}
\end{frame}

\begin{frame}{Electromagnetic waves: Electromagnetic spectrum and energy}
\small
\textbf{\alert{Frequency modulation}} (FM) radio transmission converts audio signals into frequency deviations in the carrier.
\begin{figure}
\centering
\includegraphics[width=0.35\textwidth]{figures/FM.png}
\includegraphics[width=0.55\textwidth]{figures/FMSpec.pdf}
\caption{\label{fig:radio2} \footnotesize (Left) Frequency modulation. (Right) Capacitance microphone.}
\end{figure}
\end{frame}

\begin{frame}{Electromagnetic waves: Electromagnetic spectrum and energy}
\begin{figure}
\centering
\includegraphics[width=0.475\textwidth]{figures/FMSpec2.pdf}
\includegraphics[width=0.475\textwidth]{figures/FMSpec3.pdf}
\caption{\label{fig:radio3} \footnotesize (Left) Lowering $C$ raises $f_0$, and raising $C$ lowers $f_0$. (Right) Changes in pressure correspond to changes in $C$.}
\end{figure}
\end{frame}

\begin{frame}{Electromagnetic waves: Electromagnetic spectrum and energy}
\begin{columns}
\begin{column}{0.5\textwidth}
\small
\textbf{\alert{In summary,}}
\begin{enumerate}
\item The $C$ in the LC oscillator can be made to depend on audio amplitude
\item The audio amplitude corresponds to the frequency deviation
\item The rate at which the frequency changes depends on audio frequency.
\item \textbf{For those interested,} a great final project is to assemble a DIY AM transistor radio
\end{enumerate}
\end{column}
\begin{column}{0.5\textwidth}
\begin{figure}
\centering
\includegraphics[width=0.95\textwidth]{figures/FMSpec4.pdf}
\caption{\label{fig:radio4} \footnotesize Audio (modulation) frequency determines \textit{how often} the frequency deviates, not the frequency deviation itself.}
\end{figure}
\end{column}
\end{columns}
\end{frame}

\begin{frame}{Unit 5 Summary}
\begin{enumerate}
\item Electromagnetic waves - \textbf{Chapters 24.1 - 24.4}
\begin{itemize}
\item Maxwell's Equations
\item Electromagnetic wave production
\item Electromagnetic spectrum and energy
\end{itemize}
\item Geometric optics - \textbf{Chapters 25.1 - 25.3, 25.6}
\begin{itemize}
\item Ray-tracing
\item Reflection
\item Refraction
\item Lens optics
\end{itemize}
\end{enumerate}
\end{frame}

\begin{frame}{Unit 5 Summary}
\begin{enumerate}
\item Wave optics - \textbf{Chapters 27.1 - 27.3}
\begin{itemize}
\item Wave interferance
\item Wave diffraction
\item Double slit experiments
\end{itemize}
\item Nuclear physics in medicine - \textbf{32.1 - 32.4}
\begin{itemize}
\item Diagnostics and medical imaging
\item Biological effects of ionizing radiation
\item Therapeutic uses of ionizing radiation
\item Food irradiation
\end{itemize}
\end{enumerate}
\end{frame}

\section{Geometric optics: Ray-tracing}

\begin{frame}{Geometric optics: Ray-tracing}
\begin{figure}
\includegraphics[width=0.375\textwidth]{figures/geo5.png}
\includegraphics[width=0.375\textwidth]{figures/geo3.png}
\includegraphics[width=0.375\textwidth]{figures/geo4.png}
\includegraphics[width=0.375\textwidth]{figures/geo1.png}
\caption{\label{fig:geo1} \footnotesize (Top left) Specular reflection (Top right) Diffuse reflection (Bottom left) Smooth surface (Bottom right) Rough surface.}
\end{figure}
\end{frame}

\begin{frame}{Geometric optics: Ray-tracing}
\begin{figure}
\includegraphics[width=0.375\textwidth]{figures/geo6.png}
\includegraphics[width=0.375\textwidth]{figures/geo2.png}
\caption{\label{fig:geo2} \footnotesize (Left) Your image in a mirror is due to specular reflection. (Right) The image of the moon on the ocean is due to diffuse reflection.}
\end{figure}
\small
\textbf{\alert{Specular reflection}} rule: the incident angle and reflected angle are equal.
\begin{equation}
\theta_{\rm i} = \theta_{\rm r}
\end{equation}
Both angles are usually measured with respect to the direction orthogonal to the reflecting surface.
\end{frame}

\begin{frame}{Geometric optics: Ray-tracing}
\textbf{Group exercise:} Light shows staged with lasers use moving mirrors to swing beams and create colorful effects. Show that a light ray reflected from a mirror changes direction by $2\theta$ when the mirror is rotated by an angle $\theta$.
\begin{figure}
\centering
\includegraphics[width=0.45\textwidth]{figures/laserface.jpg}
\includegraphics[width=0.45\textwidth]{figures/mirror.png}
\end{figure}
\end{frame}

\begin{frame}{Geometric optics: Ray-tracing}
\small
\textbf{\alert{The speed of light}} can be measured independently of electromagnetism, with an \textit{interferometer.}
\begin{figure}
\centering
\includegraphics[width=0.45\textwidth]{figures/michelson.png}
\caption{\label{fig:mich} A schematic of the 1887 experiment by Albert Michelson.}
\end{figure}
\end{frame}

\begin{frame}{Geometric optics: Ray-tracing}
\begin{tcolorbox}[colback=white,colframe=black!40!black,title=The Speed of Light]
\alert{The speed of light in a vacuum is
\begin{equation}
c = 2.99792458 \times 10^8~~m~s^{-1}
\end{equation}
When light travels through a material with \textit{index of \\ refraction, $n$}, the speed is
\begin{equation}
v = \frac{c}{n}
\end{equation}
}
\end{tcolorbox}
\end{frame}

\begin{frame}{Geometric optics: Ray-tracing}
\textbf{In the Michelson interferometer}, light had to travel 35 km down to a mirror, and 35 km back from the mirror.  Assuming the speed of light in a vacuum ($3\times 10^8$ m/s), how long does it take for the light to make the round trip?
\begin{itemize}
\item A: 116 $\mu$s
\item B: 116 ns
\item C: 233 $\mu$s
\item D: 233 ns
\end{itemize}
\end{frame}

\begin{frame}{Geometric optics: Ray-tracing}
\textbf{In airborne radar}, aircraft must transmit a radio pulse and record the reflection time to determine the position of other aircraft.  If an aircraft records a reflection time of 700 $\mu$s, how far away is the object?
\begin{itemize}
\item A: 210 km
\item B: 210 m
\item C: 105 m
\item D: 105 km
\end{itemize}
\end{frame}

\begin{frame}{Geometric optics: Ray-tracing}
\textbf{\alert{The index of refraction}} is a property of materials and substances that depends on the atomic or molecular structure of electrons.
\begin{table}
\centering
\begin{tabular}{| c | c | c |}
\hline
\textbf{Air} & Radio & 1.000368 \\ \hline
\textbf{Air} & Optical & 1.000293 \\ \hline
\textbf{Water, fresh} & Optical & 1.333 \\ \hline
\textbf{Ice, fresh} & Optical & 1.31 \\ \hline
\textbf{Ice, fresh} & Radio & 1.78 \\ \hline
\end{tabular}
\caption{\label{tab:n} Indices of refraction for several substances in different parts of the electromagnetic spectrum.}
\end{table}
\end{frame}

\begin{frame}{Geometric optics: Ray-tracing}
\begin{table}
\footnotesize
\centering
\begin{tabular}{| c | c | c |}
\hline
\textbf{Ice, fresh} & Optical & 1.31 \\ \hline
\textbf{Ice, fresh} & Radio & 1.78 \\ \hline
\end{tabular}
\caption{\label{tab:n} \footnotesize Indices of refraction for several substances.}
\end{table}
\footnotesize
Suppose we are measuring the ice shelf thickness in Greenland with \textbf{radio waves} to determine changes in ice volume due to climate change.  If in 2016 we observed a radio echo time of 1150 ns, and in 2020 we observed a time of 1125 ns, by how much has the ice thickness been reduced?
\begin{itemize}
\item A: 5.25 cm
\item B: 52.5 cm
\item C: 525 cm
\item D: 0.525 cm
\end{itemize}
\footnotesize
Hint: Assume the radio signal begins from the ice surface, travels down, and reflects from the bottom.
\end{frame}

\section{Geometric optics: Reflection}

\section{Geometric optics: Refraction}

\section{Geometric optics: Lens optics}

\section{Wave optics: Wave interferance}

\section{Wave optics: Wave diffraction}

\section{Wave optics: Double slit experiments}

\section{Nuclear physics in medicine: Diagnostics and medical imaging}

\section{Nuclear physics in medicine: Biological effects of ionizing radiation}

\section{Nuclear physics in medicine: Therapeutic uses of ionizing radiation}

\section{Nuclear physics in medicine: Food irradiation}

\section{Conclusion}

\begin{frame}{Unit 5 Summary}
\begin{enumerate}
\item Electromagnetic waves - \textbf{Chapters 24.1 - 24.4}
\begin{itemize}
\item Maxwell's Equations
\item Electromagnetic wave production
\item Electromagnetic spectrum and energy
\end{itemize}
\item Geometric optics - \textbf{Chapters 25.1 - 25.3, 25.6}
\begin{itemize}
\item Ray-tracing
\item Reflection
\item Refraction
\item Lens optics
\end{itemize}
\end{enumerate}
\end{frame}

\begin{frame}{Unit 5 Summary}
\begin{enumerate}
\item Wave optics - \textbf{Chapters 27.1 - 27.3}
\begin{itemize}
\item Wave interferance
\item Wave diffraction
\item Double slit experiments
\end{itemize}
\item Nuclear physics in medicine - \textbf{32.1 - 32.4}
\begin{itemize}
\item Diagnostics and medical imaging
\item Biological effects of ionizing radiation
\item Therapeutic uses of ionizing radiation
\item Food irradiation
\end{itemize}
\end{enumerate}
\end{frame}

\end{document}
