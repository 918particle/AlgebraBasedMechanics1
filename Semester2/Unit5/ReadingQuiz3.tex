\documentclass{article}
\usepackage{graphicx}
\usepackage[margin=1.5cm]{geometry}
\usepackage{amsmath}

\begin{document}

\title{Bonus Warm Up!  Unit 5, Electromagnetism}
\author{Prof. Jordan C. Hanson}

\maketitle

\section{Memory Bank}

\begin{itemize}
\item \textbf{Optics:} Snell's Law states that, for an interface between two media with \textit{indices of refraction} $n_1$ and $n_2$, light leaves one medium with an angle $\theta_1$ and enters the second medium with an angle $\theta_2$ as follows:
\begin{equation}
n_1 \sin\theta_1 = n_2\sin\theta_2
\end{equation}
\noindent The two angles are defined with respect to vertical.
\item \textbf{Optics:} Fresnel's equations give the fraction of power (in light) that is reflected ($R$) or transmitted ($T$) by an interface between media with indices of refraction $n_1$ and $n_2$:
\begin{equation}
R = \left| \frac{n_1 - n_2}{n_1 + n_2} \right|^2 \label{eq:2}
\end{equation}
\noindent Note that, to conserve power, $T = 1 - R$.
\end{itemize}

\section{Light Can Refract and Reflect}

\begin{enumerate}
\item Suppose a ray of sunlight hits a portion of the ocean that is momentarily still and flat.  If the sun is 45 degrees away from vertical, at what angle (from vertical) do the rays enter the water?  Assume the index of refraction of air is $n_1 = 1.0$, and that of the water is $n_2 = 1.336$. \\ \vspace{2cm}
\item Suppose a ray of sunlight hits a portion of the ocean that is momentarily still and flat.  (a) If the sun is 0 degrees away from vertical, what fraction of power is reflected?  Assume the index of refraction of air is $n_1 = 1.0$, and that of the water is $n_2 = 1.336$. (b) What fraction of power is transmitted into the ocean?  (c) Think about the last time you were at the beach.  Does the image of the Sun on the water look brighter when the sun is \textit{lower} or \textit{higher} in the sky?  How should Eq. \ref{eq:2} be modified (qualitatively) to account for this? \\ \vspace{2cm}
\end{enumerate}

\end{document}
