\documentclass{article}
\usepackage{graphicx}
\usepackage[margin=1.5cm]{geometry}
\usepackage{amsmath}

\begin{document}

\title{Wednesday Reading Assessment: Unit 1, Ohm's Law, Resistors in Complex Circuits}
\author{Prof. Jordan C. Hanson}

\maketitle

\section{Memory Bank}

\begin{itemize}
\item $V = I R$ ... Ohm's Law, with $V$ for voltage, $I$ for current, and $R$ for resistance.
\item $R_{tot} = R_1 + R_2$ ... Total resistance of two resistors in series.
\item $R_{tot}^{-1} = R_1^{-1} + R_2^{-1}$ ... Total resistance of two resistors in parallel.
\item $P = IV$ ... The power consumed by a device that draws a current $I$ at a voltage $V$.
\item $I = \Delta Q/\Delta t$ ... The definition of current, a change in charge versus time.
\end{itemize}

\section{Current from Resistance and Voltage}

\begin{enumerate}
\item (a) Suppose an electrical circuit is comprised of a 5V battery, and two 1k$\Omega$ resistors \textit{in series}.  What is the current flowing from the battery? (b) Suppose an electrical circuit is comprised of a 5V battery, and two 1k$\Omega$ resistors \textit{in parallel}.  What is the current flowing from the battery? \\ \vspace{1.5cm}
\end{enumerate}

\section{Power}

\begin{enumerate}
\item (a) Compute the power consumption for the circuits in parts (a) and (b) of the previous problem.  (b) If the battery has 10 A hr of charge, how long will the battery last in each case?
\end{enumerate}

\end{document}
