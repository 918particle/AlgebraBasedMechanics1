\documentclass{article}
\usepackage{graphicx}
\usepackage[margin=1.5cm]{geometry}
\usepackage{amsmath}

\begin{document}

\title{Tuesday Reading Assessment: Unit 1, Ohm's Law and Batteries, DC Circuits and Power}
\author{Prof. Jordan C. Hanson}

\maketitle

\section{Memory Bank}

\begin{itemize}
\item $C_{\rm tot} = C_1 + C_2 + ...$ ... Capacitors in \textit{parallel}.
\item $U = \frac{1}{2}C V^2$ ... Stored energy in a capacitor.
\item $P = \Delta U/\Delta t$ ... Power is the consumption or change in stored energy versus time.
\item $y(x) = mx + b$ ... Linear function with slope $m$, and y-intercept $b$
\item $m = \Delta y / \Delta x$ ... Formula for slope.
\item $V = i R$ ... Ohm's Law, with $V$ for voltage, $i$ for current, and $R$ for resistance.
\item $R_{\rm tot} = R_1 + R_2 + ...$ ... Resistors in \textit{series.}
\end{itemize}

\section{Power and Capacitance}

\begin{enumerate}
\item Suppose a $1$ $\mu$F capacitor is fully charged with 5 Volts.  (a) How much energy is stored? (b) If the energy is released in 25 ms, what is the power delivered, in mW? (c) If a bank of 10 such capacitors were charged \textit{in parallel,} what power would be delivered? \\ \vspace{1cm}
\end{enumerate}

\section{Ohm's Law: Calculating Slope from Data}

\begin{enumerate}
\item Suppose you encounter the data in Tab. \ref{tab:t}.  If you treat the voltage as the $y$-variable, and current as the $x$-variable, what is the slope of the data?  What are the units of the slope? \\ \vspace{2cm}
\begin{table}[ht]
\centering
\begin{tabular}{| c | c |}
\hline
Current (mA) & Volts (V) \\ \hline
5 & 1 \\
10 & 2 \\
15 & 3 \\
20 & 4 \\
25 & 5 \\
30 & 6 \\ \hline
\end{tabular}
\caption{\label{tab:t} A measurement of current through a resistor, given a voltage dropped across the resistor.}
\end{table}
\item If two resistors with the resistance implied by the previous exercise are connected \textit{in series,} what is the total resistance?
\end{enumerate}

\end{document}
