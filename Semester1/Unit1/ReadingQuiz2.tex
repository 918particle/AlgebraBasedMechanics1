\documentclass{article}
\usepackage{graphicx}
\usepackage[margin=1.5cm]{geometry}
\usepackage{amsmath}

\begin{document}

\title{Wednesday Reading Assessment: Unit 2, Kinematics}
\author{Prof. Jordan C. Hanson}

\maketitle

\section{Memory Bank}

\begin{itemize}
\item $a = g$ (m/s$^2$)
\item $v_f(t) = gt + v_{i,y}$ (m/s)
\item $y(t) = \frac{1}{2}gt^2 + v_{i,y} t + y_{i}$ (m)
\item $v_f^2 = v_i^2 + 2g\Delta y$ (m/s)$^2$.
\end{itemize}

\section{Chapter 2 - Kinematics}

\begin{enumerate}
\item Solve the second equation above in the memory bank for $t$, and just take the magnitude of the vectors. $t=?$ \\ \vspace{3cm}
\item Insert $t$ from the prior question into the third equation from the memory bank, and solve for $v_f^2$.  What relationship do you find? \\ \vspace{4cm}
\item \textbf{Example from KNS}: Imagine a sprinter preparing for a race.  He is starting from rest, and the race begins at $t=0$.  He accelerates up to 10 m/s at a rate of 3 m/s$^2$.  How far has he traveled? (Choose the correct equation from the memory bank before beginning). \\ \vspace{3cm}
\item If he travels at 10 m/s for another 20 seconds, what additional distance does he cover?
\end{enumerate}

\end{document}
