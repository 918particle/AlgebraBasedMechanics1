\documentclass{article}
\usepackage{graphicx}
\usepackage[margin=1.5cm]{geometry}
\usepackage{amsmath}

\begin{document}
\twocolumn

\title{Wednesday warm-up: Kinematics, II}
\author{Prof. Jordan C. Hanson}

\maketitle

\begin{figure}[ht]
\centering
\includegraphics[width=0.3\textwidth]{figures/run.jpeg}
\caption{\label{fig:1} (Left) A system moves with constant velocity.  Velocity is the slope on this plot. (Right) A system moves with non-constant velocity.}
\end{figure}

\section{Memory Bank}

\begin{enumerate}
\item $v = \frac{\Delta x}{\Delta t}$ ... Average velocity.
\item $x(t) = \frac{1}{2}at^2 + v_i t + x_i$ ... Position versus time with constant accerlation.
\item $a = \frac{\Delta v}{\Delta t}$ ... Acceleration is the change in velocity.
\item $v_f^2 = v_i^2 + 2 a \Delta x$ ... Kinematic equation without time.
\end{enumerate}

\section{Graphical Analysis of Kinematics}

\begin{enumerate}
\item Consider the motion of the runner depicted in Fig. \ref{fig:1}.  (a) What is the speed of the system after $t = 4$ seconds? (b) What is the acceleration between $t=0$ and $t=4$ seconds? (c) What is the speed of the runner between $t=0$ and $t=4$ seconds? \\ \vspace{3cm}
\item Now change the y-axis units in Fig. \ref{fig:1} to velocity, in meters per second.  Answer parts (a)-(c) from the previous question again.  For part (c), write your answer as a function of time. \\ \vspace{4cm}
\item Suppose a runner accelerates at 3 m s$^{-2}$, starting from rest.  (a) \textit{Where} does the runner reach 10 m s$^{-1}$?  (b) \textit{When} does the runnder reach 10 m s$^{-1}$?
\end{enumerate}

\end{document}
