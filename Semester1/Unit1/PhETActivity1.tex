\documentclass{article}
\usepackage{graphicx}
\usepackage[margin=1.5cm]{geometry}
\usepackage{amsmath}
\usepackage{url}

\begin{document}

\title{Laboratory Activity: Unit 1, Measuring $g$}
\author{Prof. Jordan C. Hanson}

\maketitle

\section{The Acceleration of Gravity, First Measurement}

The goal of this laboratory activity is to measure $g$, the acceleration due to gravity.  Let $g$ be the acceleration downward, and assume it is constant.  Let $v_{i,y}$ be the initial velocity in the y-axis, and assume the y-axis is vertical.  Let $y_i$ be the initial vertical position.  The position of a vertically accelerating object, in general, is given by
\begin{equation}
y(t) = -\frac{1}{2}gt^2 + v_{i,y} t + y_i \label{eq:1}
\end{equation}
In Eq. \ref{eq:1}, $-g$ is the acceleration.  The vector form of acceleration points down, so we give $g$ a minus sign.  We begin to observe the system at time $t=0$.  If a stationary marble is dropped and Eq. \ref{eq:1} is used to predict the position $y(t)$, then $v_{i,y} = 0$.  Let the change in height be $h = y(t) - y_i$.  Show that
\begin{equation}
h = -\frac{1}{2}g t^2
\end{equation}
Use this equation to solve for $g$.  The result should be
\begin{equation}
g = \frac{-2h}{t^2} \label{eq:2}
\end{equation}
Use the following procedure to measure $g$:
\begin{enumerate}
\item Use the ruler to measure the vertical displacement of the marble.
\item Use a stopwatch to time the descent of the marble.
\item Insert the measured values of $h$ and $t$ into Eq. \ref{eq:2} to calculate $g$.
\item Repeat 10 times and compute the average for $g = g_{ave}$.
\item Calculate the \textit{percent error} of $g$, using the $g = 9.81$ m s$^{-2}$.
\begin{equation}
\Delta g\% = \frac{g_{ave} - g}{g} \times 100
\end{equation}
\end{enumerate}

\section{The Acceleration of Gravity, Second Measurement}

Now measure $g$ using a pendulum, constructed in the same fashion as the previous lab activity.  Let $T$ be the period of the pendulum, and $L$ be the length.  The relationship between $T$, $L$, and $g$ is

\begin{equation}
T = 2\pi \sqrt{L/g} \label{eq:3}
\end{equation}

Solve Eq. \ref{eq:3} for $g$, and repeat the above procedure to obtain $g_{ave}$ and the percent error using the pendulum.  Compare the results for $g_{ave}$ from each technique.

\end{document}
