\title{Syllabus for Algebra-Based Physics-2: Mechanics (PHYS135A)}
\author{Dr. Jordan Hanson - Whittier College Dept. of Physics and Astronomy}
\date{\today}
\documentclass[10pt]{article}
\usepackage[a4paper, total={18cm, 27cm}]{geometry}
\usepackage{outlines}
\usepackage[sfdefault]{FiraSans}
\usepackage{hyperref}
\begin{document}
\maketitle

\begin{abstract}
The concepts of algebra-based mechanics will be presented within the context of interactive problem-solving.  First, the concepts of displacement, velocity, and acceleration in one and two dimensions will be introduced, building up to Newton's Laws of motion.  Next, the concepts of friction and rotational motion will be added.  More complex problems will be introduced through the conservation of energy and linear momentum, followed by the rotational equivalents.  The course work will include analytic textbook problems, interactive computational exercises, and lab-based activities.
\end{abstract}
\noindent
\textit{\textbf{Pre-requisites}: None} \\
\textit{\textbf{Course credits, Liberal Arts Categorization}: 4 Credits, None} \\
\textit{\textbf{Regular course hours and location}: Tuesday and Thursday, 8:50-10:50, SLC 232, predominantly synchronous time.} \\
\textit{\textbf{Instructor contact information}: 
\begin{enumerate}
\item Email: jhanson2@whittier.edu
\item Cell: 562.351.0047
\item Zoom ID / pass: 796 092 0745 / 667725
\item YouTube Channel: \url{www.youtube.com/918particle}
\item Book online appointments: \url{https://fgucmvjkylvmgqfsco.10to8.com}
\end{enumerate}}
\noindent
\textit{\textbf{Office hours}: Booking service to schedule meeting: \url{10to8.com} as above, and indicate in-person or online.  Office Location: SLC 212.} \\
\textit{\textbf{Attendance/Absence}: Students are required by Whittier College to sign in to class.} \\ 
\textit{\textbf{Late work policy}: Late work is generally not accepted, but is left to the discretion of the instructor.} \\
\textit{\textbf{Text}: OpenStax \textit{College Physics}. Go to \url{https://openstax.org/details/books/college-physics} for access (free).} \\
\textit{\textbf{Homework and Reading}: OpenStax Tutor: \url{https://tutor.openstax.org/enroll/099326/College-Physics-Fall-2021}.} Homework sets will usually require 10 exercises per week, and is integrated with the open-access text from which reading assignments will be given. This system costs 10 dollars, but the textbook is free. \\
\textit{\textbf{Online Laboratories}: Some laboratory activities will be through \url{pivotinteractives.com}.  There is no cost to the student.  Click to join: \url{https://app.pivotinteractives.com/join-class?classKey=ck-9f74f665}. Tutorial: \url{https://help.pivotinteractives.com/en/articles/3389724-student-guide-to-using-pivot-interactives}.} \\
\textit{\textbf{Grading}: There will be one midterm, one (optional) final exam, and daily warm-up exercises.  There will also be homework problem sets, submitted via OpenStax Tutor.  Finally, there will be a self-designed project and presentation given at the end of the term.  Grading percentages are shown in Tab. \ref{tab:grades}.  \textit{Note that the final exam will be optional.} The final project is a good opportunity for the use of Digital Storytelling.  For more information, see \url{https://diglibarts.whittier.edu} and contact Sonia Chaidez: \url{schaidez@whittier.edu}. } \\

\begin{table}[h]
\centering
\begin{tabular}{| c | c |}
\hline
Item & Percentage \\ \hline \hline
Daily exercises & 10 \% \\ \hline
Homework sets and labs & 30 \% \\ \hline
Midterm & 20 \% \\ \hline
Final & 20 \% \\ \hline
Final Project + Presentation & 20\% \\ \hline
\end{tabular}
\begin{tabular}{| c | c |}
\hline
Item & Percentage \\ \hline \hline
Daily exercises & 15 \% \\ \hline
Homework sets and labs & 35 \% \\ \hline
Midterm & 25 \% \\ \hline
Final & -~- \\ \hline
Final Project + Presentation & 25\% \\ \hline
\end{tabular}
\caption{\label{tab:grades} (Left) These are the grade settings with the final exam included. (Right) These are the grade settings without the final exam.  The final exam is optional. \textbf{Grade Settings}: $\geq 60\%, <70\%$ = D, $\geq 70\%, <80\%$ = C, $\geq 80\%, <90\%$ = B, $\geq 90\%, <100\%$ = A. Pluses and minuses: 0-3\% minus, 3\%-6\% straight, 6\%-10\% plus (e.g. 79\% = C+, 91\% = A-).}
\end{table}
\noindent
\textit{\textbf{Student Disability Services}: Whittier College is committed to make learning experiences as accessible as possible. If you experience physical or academic barriers due to a disability, you are encouraged to contact Student Disability Services (SDS) to discuss options. To learn more about academic accommodations, email disabilityservices@whittier.edu, call (562) 907-4825, or go to SDS which is located on the ground floor of Wardman Library.} \\
\textit{\textbf{Academic Honesty Policy}: \url{http://www.whittier.edu/academics/academichonesty}} \\

\clearpage

\textit{\textbf{Policy due to COVID-19}:
\begin{enumerate}
\item Students are required to be vaccinated against COVID-19 and wear a mask indoors.
\item Group project results will be designed, constructed, and executed outside of class.  Projects will be presented \textit{in-person} during the last week of class.
\item Students may opt-out of taking the final exam.  If one does not take it, the assignment weights for the final grade will be those given in Tab. \ref{tab:grades} (right).
\item The final project can be created in one of two options.  \textbf{Option A}: A 10 minute traditional presentation with several minutes for questions.  \textbf{Option B}: Digital liberal arts style, in the form of video or digital book form that educates the class on a topic.
\end{enumerate}}
\noindent
\textit{\textbf{Course Objectives}:}
\begin{itemize}
\item To practice written and oral expression of scientifically technical ideas.
\item To solve word problems pertaining to physics and mathematics.
\item To construct mathematical models of mechanical systems.
\item To apply logical thinking to conceptually-posed physics problems.
\item To practice scientific experimentation, data analysis, and reporting of results.
\end{itemize}

\textit{\textbf{Course Outline}:}
\begin{outline}[enumerate]
\1 Unit 0 - \textbf{Chapters 1.2 - 1.4}
\2 Introduction to iClicker, class procedures, reading syllabus.
\2 \textit{Warm-up activity: gas mileage, speed, calories and energy.}
\2 Unit-analysis, approximation, and coordinate systems.
\2 Adding and subtracting vectors, displacement and translational motion.
\1 Unit 1 - \textbf{Chapters 2.1 - 2.4}
\2 Tuesday reading quiz: chapter 2.3
\2 Thursday reading quiz: chapter 2.4
\2 Topics covered:
\3 Vector displacement, scalar distance, and coordinate systems.
\3 Time, velocity vector, acceleration vector.
\1 Unit 2 - \textbf{Chapters 2.5 - 2.8}
\2 Tuesday reading quiz: chapter 2.5.
\2 Thursday reading quiz: chapter 2.5.
\2 Topics covered:
\3 Kinematic equations.
\3 Kinematic equations with vectors.
\3 Graphical analysis of kinematics.
\1 Unit 3 - \textbf{Chapters 3.1 - 3.4}
\2 Tuesday reading quiz: chapter 3.1
\2 Thursday reading quiz: chapter 3.3
\2 Topics covered:
\3 Kinematics in 2D.
\3 Addition and subtraction of vectors: a review.
\3 Projectile motion.

\1 \textbf{Midterm: October 6th, 2021.} This exam is worth 20-25\% of the grade (see Tab. \ref{tab:grades}), and will cover Units 0-3.  The emphasis is on working with units, vectors, and applying kinematic equations in 1D. 

\1 Unit 4 - \textbf{Chapters 4.1-4.4}
\2 Tuesday reading quiz: chapter 4.2
\2 Thursday reading quiz: chapter 4.3
\2 Topics covered:
\3 Newton's First Law
\3 Newton's Second Law
\3 Newton's Third Law
\1 Unit 5 - \textbf{Chapters 4.5, 5.1 - 5.3}
\2 Tuesday reading quiz: chapter 4.5
\2 Thursday reading quiz: chapter 5.1
\2 Topics covered:
\3 Normal forces and tension.
\3 Drag and friction.
\3 Elasticity
\1 Unit 6 - \textbf{Chapters 6.1 - 6.3, 6.5 - 6.6}
\2 Tuesday reading quiz: chapter 6.1
\2 Thursday reading quiz: chapter 6.5
\2 Topics covered:
\3 Rotational kinematics
\3 Centripetal force
\3 \textit{Special topic: gravity and Kepler's Laws}
\1 Unit 7 - \textbf{Chapters 7.1 - 7.4, 7.6 - 7.8}
\2 Tuesday reading quiz: chapter 7.2
\2 Thursday reading quiz: chapter 7.3
\2 Topics covered:
\3 Work, kinetic energy, work-energy theorem.
\3 Potential energy.
\3 Energy conservation and power.
\3 \textit{Special topic: work, energy and power in human beings}

\1 \textbf{Final project proposal due: November 12th, 2021.}  The final project proposal is a 1-2 page report describing the hypothesis, data to me collected, experimental procedures, and parts required.  Submit one document per group, and include all names and student IDs.

\1 Unit 8 - \textbf{Chapters 8.1, 8.3 - 8.6}
\2 Tuesday reading quiz: chapter 8.1
\2 Thursday reading quiz: chapter 8.4
\2 Topics covered:
\3 \textit{Math review: solving systems of equations}
\3 Linear momentum, conservation of momentum.
\3 Elastic scattering.
\3 Inelastic scattering.
\3 Scattering in 2D
\1 Unit 9 - \textbf{Chapters 10.1 - 10.4}
\2 Tuesday reading quiz: chapter 10.1
\2 Thursday reading quiz: chapter 10.4
\2 Topics covered:
\3 Angular acceleration and rotational kinematics.
\3 Inertia and rotational kinetic energy.
\1 Unit 10 - \textbf{Chapters 10.5 - 10.6}
\2 Tuesday reading quiz: chapter 10.5
\2 Thursday reading quiz: chapter 10.6
\2 Topics covered:
\3 Angular momentum conservation.
\3 Collisions with rotations.

\1 \textbf{December 1st, 2021: students receive final exam study guide.} The final exam is optional and the emphasis is on energy and power, energy and momentum conservation, and angular momentum.

\1 Unit 11 - \textbf{Class Presentations and Final Review}
\2 No reading quizzes
\2 Group presentations:
\3 Worth 20-25\% of the course grade (See Tab. \ref{tab:grades}).
\3 Given as a group
\3 10-15 minute presentation with slides or board work
\2 Final exam reviews will be given Dec. 3rd.
\end{outline}

\end{document}
