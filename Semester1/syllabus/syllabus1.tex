\title{Syllabus for Algebra-Based Physics-1: Mechanics (PHYS135A-01)}
\author{Dr. Jordan Hanson - Whittier College Dept. of Physics and Astronomy}
\date{\today}
\documentclass[10pt]{article}
\usepackage[a4paper, total={18cm, 27cm}]{geometry}
\usepackage{outlines}
\usepackage[sfdefault]{FiraSans}
\usepackage{hyperref}

\begin{document}
\maketitle

\begin{abstract}
The concepts of algebra-based mechanics will be presented within the context of interactive problem-solving.  First, the concepts of displacement, velocity, and acceleration in one and two dimensions will be introduced, building up to Newton's Laws of motion.  Next, the concepts of friction and rotational motion will be added.  More complex problems will be introduced through the conservation of energy and linear momentum, followed by the rotational equivalents.  The course work will include interactive computational exercises, analytic textbook problems, and lab-based activities.
\end{abstract}
\noindent
\textit{\textbf{Pre-requisites}: None.} \\
\textit{\textbf{Course credits, Liberal Arts Categorization}: 4 Credits, None} \\
\textit{\textbf{Regular course hours}: Monday and Wednesday from 8:50 - 10:50 in SLC 232} \\
\textit{\textbf{Instructor contact information}: jhanson2@whittier.edu, tel. 562.907.5130} \\
\textit{\textbf{Office hours}: Mondays from 1-5pm in SLC 212} \\
\textit{\textbf{Attendance/Absence}: Students needing to reschedule midterms and exams should notify the professor a reasonable time beforehand.  Students are allowed up to three absences with no penalty.  Each additional absence will result in a 10\% reduction in the in-class participation grade.  Arriving more than 10 minutes late to class will count as an absence for grading purposes. (see \textit{\textbf{Grading}}}). \\
\textit{\textbf{Late work policy}: Late work will not be accepted.} \\
\textit{\textbf{Technology policy}: The use of mobile, tablet, or laptop devices during class is not allowed, with the exception of note-taking.} \\
\textit{\textbf{Text}: College Physics (openstax.org) -  https://openstax.org/details/books/college-physics} \\
\textit{\textbf{Grading}: There will be three tests, each examining conceptual understanding in step-by-step problems.  Each midterm is worth 15\% of the final grade.  The weekly online homework is worth 20\% of the grade.  Interactive in-class activities will be worth 10\% of the final grade.  Lab groups will present results of a group project worth 10\% of the grade.  The final exam will be held on December 11th from 1-3pm, and will be worth 15\% of the grade.} \\
\textit{\textbf{Grade Settings}: $<60\%$ = F, $>60\%,\leq 70\%$ = D, $>70\%,\leq80\%$ = C, $>80\%,\leq 90\%$ = B, $<90\%,\leq 100\%$ = A.  Pluses and minuses: 0-3\% minus, 3\%-6\% straight, 6\%-10\% plus (e.g. 79\% = C+, 91\% = A-)} \\
\textit{\textbf{Homework sets}: Typically 10 online problems per week, assigned via the ExpertTA system on Fridays, due the following Fridays.  Follow this link to register for the online homework software: \url{http://goeta.link/USB06CA-AF2720-1PE}} \\
\textit{\textbf{Just in Time Teaching}: Just in Time Teaching (JITT) are assignments beginning the second week of class based on course reading.  The day before class, students will receive conceptual physics questions based on the reading.  Based on the response rate and accuracy, the JITT exercise will be used to tailor the lecture and discussions the following day.  Answering the JITT questions are part of the in-class participation grade.} \\
\textit{\textbf{ADA Statement on Disability Services}: Whittier College is committed to make learning experiences as accessible as possible. If you experience physical or academic barriers due to a disability, you are encouraged to contact Student Disability Services (SDS) to discuss options. To learn more about academic accommodations, email disabilityservices@whittier.edu, call (562) 907-4825, or go to SDS which is located on the ground floor of Wardman Library.} \\
\textit{\textbf{Academic Honesty Policy}: \url{http://www.whittier.edu/academics/academichonesty}} \\
\textit{\textbf{Course Objectives}:}
\begin{itemize}
\item Written expression of quantitative and numerical ideas and arguments.
\item Oral expression of quantitative and numerical ideas and arguments.
\item \textbf{Problem solving using numerical skills.}
\item Mathematical modeling.
\item Logical thinking.
\item Analysis of data and results.
\end{itemize}
\textit{\textbf{Course Outline}:}
\begin{outline}[enumerate]
\1 Week 1 - September 5th through September 8th - \textbf{Chapter 1}
\2 Estimations, approximations, unit-analysis, and coordinate systems
\2 Adding and subtracting vectors, displacement and translational motion

\1 Week 2 - September 9th through September 15th - \textbf{Chapter 2.1-2.3}
\2 Displacement, scalar quantities, coordinate systems
\2 Time, velocity and speed

\1 Week 3 - September 16th through September 22nd - \textbf{Chapter 2.4-2.8}
\2 Acceleration
\2 Kinematic equations with constant acceleration
\2 Analysis of graphs

\1 Week 4 - September 23rd through September 29th - \textbf{Chapter 3.1-3.3}
\2 Kinematics with vectors, more than one dimension
\2 More on graphs and vectors

\1 Midterm 1 - Beginning of Week 5

\1 Week 5 - September 30th through October 6th - \textbf{Chapter 3.4-3.5}
\2 Projectile motion
\2 Addition of velocities, relativity

\1 Week 6 - October 7th through October 13th - \textbf{Chapter 4.1-4.4}
\2 Forces
\2 Newton's First, Second, and Third Laws of Motion

\1 Week 7 - October 14th through October 20th - \textbf{Chapters 4.5-4.8}
\2 Examples and applications of Newton's Laws
\2 \textit{Special topic: the Four Forces of The Universe}

\1 Mid-semester break, October 19th

\1 Midterm 2 - Beginning of Week 8

\1 Week 8 - October 21st through October 27th - \textbf{Chapter 5}
\2 Frictional forces
\2 Drag forces
\2 Elastisity: Stress and Strain

\1 Week 9 - October 28th through November 3rd - \textbf{Chapters 6.1-6.3}
\2 Rotational displacement, velocity, and acceleration
\2 Centripetal acceleration and force

\1 Week 10 - November 4th through November 10th - \textbf{Chapters 6.5-6.6}
\2 Newton's Universal Law of Gravitation
\2 Kepler's Laws

\1 Week 11 - November 11th through November 17th - \textbf{Chapters 7.1-7.5}
\2 Definition of Work, kinetic energy
\2 The Work Energy Theorem
\2 Potential and kinetic energy, conservative and non-conservative forces

\1 Midterm 3 - Beginning of Week 12

\1 Week 12 - November 18th and November 24th - \textbf{Chapters 7.6-7.9}
\2 Conservation of energy
\2 \textit{Special topic: World energy use}
\2 \textit{Special topic: Work, energy, and power in the human body}

\1 Week 13 - November 25th through December 1st  - \textbf{Chapters 8.1, 8.3-8.5}
\2 Linear momentum and momentum conservation
\2 Collisions, elastic and inelastic
\2 \textit{Angular momentum, time-permitting: Chapter 10}

\1 Week 14 - December 2nd through December 8th - \textbf{Class Presentations and Final Review}
\end{outline}
\end{document}
