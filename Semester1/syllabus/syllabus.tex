\title{Syllabus for Algebra-Based Physics-1: Mechanics (PHYS135A-01)}
\author{Dr. Jordan Hanson - Whittier College Dept. of Physics and Astronomy}
\date{\today}
\documentclass[10pt]{article}
\usepackage[a4paper, total={18cm, 27cm}]{geometry}
\usepackage{outlines}
\usepackage[sfdefault]{FiraSans}
\usepackage{hyperref}

\begin{document}
\maketitle

\begin{abstract}
The concepts of algebra-based mechanics will be presented within the context of interactive problem-solving.  First, the concepts of displacement, velocity, and acceleration in one and two dimensions will be introduced, building up to Newton's Laws of motion.  Next, the concepts of friction and rotational motion will be added.  More complex problems will be introduced through the conservation of energy and linear momentum, followed by the rotational equivalents.  The course work will include analytic textbook problems, interactive computational exercises, and lab-based activities.
\end{abstract}
\noindent
\textit{\textbf{Pre-requisites}: None.} \\
\textit{\textbf{Course credits, Liberal Arts Categorization}: 4 Credits, None} \\
\textit{\textbf{Regular course hours}: Monday and Wednesday from 8:50 - 10:50 in SLC 228} \\
\textit{\textbf{Instructor contact information}: jhanson2@whittier.edu, tel. 562.907.5130} \\
\textit{\textbf{Office hours}: Mondays, 16:30-17:30, and Tuesdays from 13:00-16:00 in SLC 212} \\
\textit{\textbf{Attendance/Absence}: In-class activities will serve as attendance (see \textit{\textbf{Grading}}}).  Students needing to reschedule midterms and exams should notify the professor a reasonable time beforehand. \\
\textit{\textbf{Late work policy}: Acceptance of late work is at the discretion of the professor.} \\
\textit{\textbf{Text}: College Physics (openstax.org) -  https://openstax.org/details/books/college-physics} \\
\textit{\textbf{Grading}: There will be three tests, each examining conceptual understanding in step-by-step problems. Each test is worth 15\% of the final grade. The weekly online homework is worth 20\% of the grade. Interactive in-class activities will be worth 10\% of the final grade. Lab groups will present results of a group project worth 10\% of the grade. The final exam will be held on December 10th from 10:30-12:30, and will be worth 15\% of the grade.} \\
\textit{\textbf{Final Exam:} The final exam will be held on December 10th from 10:30-12:30.} \\
\textit{\textbf{Grade Settings}: $<60\%$ = F, $>60\%,\leq 70\%$ = D, $>70\%,\leq80\%$ = C, $>80\%,\leq 90\%$ = B, $<90\%,\leq 100\%$ = A.  Pluses and minuses: 0-3\% minus, 3\%-6\% straight, 6\%-10\% plus (e.g. 79\% = C+, 91\% = A-)} \\
\textit{\textbf{Homework sets}: Typically 10 problems per week, assigned on Monday and collected the following Monday.  Please follow the link \url{http://goeta.link/USB06CA-F1461C-1X2} to access the online homework system.  The system requires \$32.50 for access (remember that the textbook is free).  The online system will give clues to struggling students and provide the professor with useful class statistics to aid in class management.} \\
\textit{\textbf{ADA Statement on Disability Services}: The Americans with Disabilities Act (ADA) is a federal anti-discrimination statute that provides comprehensive civil rights protection for persons with disabilities. Among other things, this legislation requires that all students with disabilities be guaranteed a learning environment that provides for reasonable accommodation of their disabilities. If you believe you have a disability requiring an accommodation, please contact Disability Services: disabilityservices@whittier.edu, tel. 562.907.4825.} \\
\textit{\textbf{Academic Honesty Policy}: \url{http://www.whittier.edu/academics/academichonesty}} \\
\textit{\textbf{Course Objectives}:}
\begin{itemize}
\item To practice written expression of quantitative and numerical ideas and arguments.
\item To practice expression of quantitative and numerical ideas and arguments.
\item Improvement of numerical analysis and problem solving.
\item Improvement of problem solving via computer simulations.
\item To practice the analysis of scientific data and results.
\end{itemize}
\textit{\textbf{Course Outline}:}
\begin{outline}[enumerate]
\1 Unit 0 - \textbf{Chapters 1.2 - 1.4}
\2 Introduction to iClicker, class procedures, reading syllabus.
\2 \textit{Warm-up activity: gas mileage, speed, calories and energy.}
\2 Unit-analysis, approximation, and coordinate systems.
\2 Adding and subtracting vectors, displacement and translational motion.
\1 Unit 1 - \textbf{Chapters 2.1 - 2.4}
\2 Monday reading quiz: chapter 2.3
\2 Wednesday reading quiz: chapter 2.4
\2 Topics covered:
\3 Vector displacement, scalar distance, and coordinate systems.
\3 Time, velocity vector, acceleration vector.
\1 Unit 2 - \textbf{Chapters 2.5 - 2.8}
\2 Monday reading quiz: chapter 2.5.
\2 Wednesday reading quiz: chapter 2.5.
\2 Topics covered:
\3 Kinematic equations.
\3 Kinematic equations with vectors.
\3 Graphical analysis of kinematics.
\1 Unit 3 - \textbf{Chapters 3.1 - 3.4}
\2 Monday reading quiz: chapter 3.1
\2 Wednesday reading quiz: chapter 3.3
\2 Topics covered:
\3 Kinematics in 2D.
\3 Addition and subtraction of vectors: a review.
\3 Projectile motion.

\1 \textbf{First midterm: October 2nd, 2019 during normal class time.} This exam is worth 15\% of the class grade, and will cover Units 0-3.  The emphasis is on working with units, vectors, and applying kinematic equations in 1D. 

\1 Unit 4 - \textbf{Chapters 4.1-4.4}
\2 Monday reading quiz: chapter 4.2
\2 Wednesday reading quiz: chapter 4.3
\2 Topics covered:
\3 Newton's First Law
\3 Newton's Second Law
\3 Newton's Third Law
\1 Unit 5 - \textbf{Chapters 4.5, 5.1 - 5.3}
\2 Monday reading quiz: chapter 4.5
\2 Wednesday reading quiz: chapter 5.1
\2 Topics covered:
\3 Normal forces and tension.
\3 Drag and friction.
\3 Elasticity
\1 Unit 6 - \textbf{Chapters 6.1 - 6.3, 6.5 - 6.6}
\2 Monday reading quiz: chapter 6.1
\2 Wednesday reading quiz: chapter 6.5
\2 Topics covered:
\3 Rotational kinematics
\3 Centripetal force
\3 \textit{Special topic: gravity and Kepler's Laws}

\1 \textbf{Second midterm: November 4th, 2019 during normal class time.} This exam is worth 15\% of the class grade, and will cover Units 4-6.  The emphasis is on Newton's second law and rotational kinematics.

\1 Unit 7 - \textbf{Chapters 7.1 - 7.4, 7.6 - 7.8}
\2 Monday reading quiz: chapter 7.2
\2 Wednesday reading quiz: chapter 7.3
\2 Topics covered:
\3 Work, kinetic energy, work-energy theorem.
\3 Potential energy.
\3 Energy conservation and power.
\3 \textit{Special topic: work, energy and power in human beings}
\1 Unit 8 - \textbf{Chapters 8.1, 8.3 - 8.6}
\2 Monday reading quiz: chapter 8.1
\2 Wednesday reading quiz: chapter 8.4
\2 Topics covered:
\3 \textit{Math review: solving systems of equations}
\3 Linear momentum, conservation of momentum.
\3 Elastic scattering.
\3 Inelastic scattering.
\3 Scattering in 2D
\1 Unit 9 - \textbf{Chapters 10.1 - 10.4}
\2 Monday reading quiz: chapter 10.1
\2 Wednesday reading quiz: chapter 10.4
\2 Topics covered:
\3 Angular acceleration and rotational kinematics.
\3 Inertia and rotational kinetic energy.
\1 Unit 10 - \textbf{Chapters 10.5 - 10.6}
\2 Monday reading quiz: chapter 10.5
\2 Wednesday reading quiz: chapter 10.6
\2 Topics covered:
\3 Angular momentum conservation.
\3 Collisions with rotations.

\1 \textbf{Third midterm: November 25th, 2019 during normal class time.} This exam is worth 15\% of the class grade, and will cover Units 4-6.  The emphasis is on energy and power, energy and momentum conservation, and angular momentum.

\1 Unit 11 - \textbf{Class Presentations and Final Review}
\2 No reading quizzes
\2 Group presentations:
\3 Worth 10\% of the course grade
\3 Given as a gropu
\3 10-15 minute presentation with slides or board work
\2 Final exam reviews will be given between Dec. 6th - 9th, for the convenience of the students.
\end{outline}
\end{document}
