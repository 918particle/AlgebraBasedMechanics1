\title{Syllabus for Algebra-Based Physics-1: Mechanics (PHYS135A)}
\author{Dr. Jordan Hanson - Whittier College Dept. of Physics and Astronomy}
\date{\today}
\documentclass[10pt]{article}
\usepackage[a4paper, total={18cm, 27cm}]{geometry}
\usepackage{outlines}
\usepackage{hyperref}
\begin{document}
\maketitle

\begin{abstract}
The concepts of algebra-based mechanics will be presented within the context of interactive problem-solving.  First, the concepts of displacement, velocity, and acceleration in one and two dimensions will be introduced, building up to Newton's Laws of motion.  Next, the concepts of friction and rotational motion will be added.  More complex problems will be introduced through the conservation of energy and linear momentum, followed by the rotational equivalents.  The course work will include analytic textbook problems, interactive computational exercises, and lab-based activities.
\end{abstract}
\noindent
\textit{\textbf{Course credits, Liberal Arts Categorization}: 4 Credits, None} \\
\textit{\textbf{Regular course hours and location}: Tuesday and Thursday, SLC232, 8:50-10:50.} \\
\textit{\textbf{Instructor contact information}: 
\begin{enumerate}
\item Email/Office Location: jhanson2@whittier.edu / SLC 212
\item Discord handle: 918particle\#5083
\end{enumerate}}
\noindent
\textit{\textbf{Office hours}: Connect with instructor via Discord server: \url{https://discord.gg/Rj6pTscv}} \\
\textit{\textbf{Attendance/Absence}: In-class activities will serve as attendance (see \textbf{Grading}). Students needing to reschedule midterms and exams should notify the professor a reasonable time beforehand}.\\ 
\textit{\textbf{Late work policy}: Late work is generally not accepted, but is left to the discretion of the instructor.} \\
\textit{\textbf{Text}: OpenStax \textit{College Physics}: \url{https://openstax.org/details/books/college-physics}.} \\
\textit{\textbf{Homework}: \url{https://tutor.openstax.org/enroll/059769/Algebra-Based-Physics-A-Fall-2022}.} Homework sets will usually require 10 exercises per week, and is integrated with the open-access text from which reading assignments will be given. This system costs 10 dollars, but the textbook is free. \\
\textit{\textbf{Grading}: Grading percentages are shown in Tab. \ref{tab:grades}.  The final project is a good opportunity for the use of Digital Storytelling.  For more information, see \url{https://diglibarts.whittier.edu} and contact Sonia Chaidez: \url{schaidez@whittier.edu}. } \\

\begin{table}[h]
\small
\centering
\begin{tabular}{| c | c |}
\hline
Item & Percentage \\ \hline \hline
Daily exercises & 15 \% \\ \hline
Homework sets and labs & 30 \% \\ \hline
Midterms & 30 \% \\ \hline
Final Project + Presentation & 25\% \\ \hline
\end{tabular}
\caption{\label{tab:grades} (Left) These are the grading percentages. \textbf{Grade Settings}: $\geq 60\%, <70\%$ = D, $\geq 70\%, <80\%$ = C, $\geq 80\%, <90\%$ = B, $\geq 90\%, <100\%$ = A. Pluses and minuses: 0-3\% minus, 3\%-6\% straight, 6\%-10\% plus (e.g. 79\% = C+, 91\% = A-).}
\end{table}
\noindent
\textit{\textbf{Student Accessibility Services}: Whittier College is committed to make learning experiences as accessible as possible. If you experience physical or academic barriers due to a disability, you are encouraged to contact Student Accessibility Services (SAS) to discuss options. To learn more about academic accommodations, drop by our center (ground floor of Wardman Library), email sas@whittier.edu, or contact 562-907-4825.} \\
\textit{\textbf{Academic Honesty Policy}: \url{http://www.whittier.edu/academics/academichonesty}} \\ \\
\noindent
\textit{\textbf{Course Objectives}:}
\begin{itemize}
\item To practice written and oral expression of scientifically technical ideas.
\item To solve word problems pertaining to physics and mathematics.
\item To construct mathematical models of mechanical systems.
\item To apply logical thinking to conceptually-posed physics problems.
\item To practice scientific experimentation, data analysis, and reporting of results.
\end{itemize}
\noindent
\textit{\textbf{Course Outline}:}
\begin{outline}[enumerate]
\1 Unit 0 - \textbf{Chapters 1.2 - 1.4}
\2 Introduction to iClicker, class procedures, reading syllabus.
\2 \textit{Warm-up activity: gas mileage, speed, calories and energy.}
\2 Unit-analysis, approximation, and coordinate systems.
\2 Adding and subtracting vectors, displacement and translational motion.
\1 Unit 1 - \textbf{Chapters 2.1 - 2.4}
\2 Tuesday warm up: chapter 2.3
\2 Thursday warm up: chapter 2.4
\2 Topics covered:
\3 Vector displacement, scalar distance, and coordinate systems.
\3 Time, velocity vector, acceleration vector.
\1 Unit 2 - \textbf{Chapters 2.5 - 2.8}
\2 Tuesday warm up: chapter 2.5.
\2 Thursday warm up: chapter 2.5.
\2 Topics covered:
\3 Kinematic equations.
\3 Kinematic equations with vectors.
\3 Graphical analysis of kinematics.
\1 Unit 3 - \textbf{Chapters 3.1 - 3.4}
\2 Tuesday warm up: chapter 3.1
\2 Thursday warm up: chapter 3.3
\2 Topics covered:
\3 Kinematics in 2D.
\3 Addition and subtraction of vectors: a review.
\3 Projectile motion.

\1 \textbf{First midterm: distributed October 7th, 2022, take-home and open-book.}  This exam is worth 15\% of the class grade, and will cover Units 0-3. The emphasis is on working with units, vectors, and applying kinematic equations in 1D.

\1 Unit 4 - \textbf{Chapters 4.1-4.4}
\2 Tuesday warm up: chapter 4.2
\2 Thursday warm up: chapter 4.3
\2 Topics covered:
\3 Newton's First Law
\3 Newton's Second Law
\3 Newton's Third Law
\1 Unit 5 - \textbf{Chapters 4.5, 5.1 - 5.3}
\2 Tuesday warm up: chapter 4.5
\2 Thursday warm up: chapter 5.1
\2 Topics covered:
\3 Normal forces and tension.
\3 Drag and friction.
\3 Elasticity
\1 Unit 6 - \textbf{Chapters 6.1 - 6.3, 6.5 - 6.6}
\2 Tuesday warm up: chapter 6.1
\2 Thursday warm up: chapter 6.5
\2 Topics covered:
\3 Rotational kinematics
\3 Centripetal force
\3 \textit{Special topic: gravity and Kepler's Laws}
\1 Unit 7 - \textbf{Chapters 7.1 - 7.4, 7.6 - 7.8}
\2 Tuesday warm up: chapter 7.2
\2 Thursday warm up: chapter 7.3
\2 Topics covered:
\3 Work, kinetic energy, work-energy theorem.
\3 Potential energy.
\3 Energy conservation and power.
\3 \textit{Special topic: work, energy and power in human beings}

\1 \textbf{Second midterm: distributed November 18th, 2022, take-home and open-book.} This exam is worth 15\% of the class grade, and will cover Units 4-7. The emphasis will be on Newton's Second Law, free-body diagrams and friction, and the definitions of work and energy.

\1 \textbf{Final project proposal due: November 18th, 2022.}  The final project proposal is a 1-2 page report describing the hypothesis, data to me collected, experimental procedures, and parts required.  Submit one document per group, and include all names and student IDs.

\1 Unit 8 - \textbf{Chapters 8.1, 8.3 - 8.6}
\2 Tuesday warm up: chapter 8.1
\2 Thursday warm up: chapter 8.4
\2 Topics covered:
\3 \textit{Math review: solving systems of equations}
\3 Linear momentum, conservation of momentum.
\3 Elastic scattering.
\3 Inelastic scattering.
\3 Scattering in 2D
\1 Unit 9 - \textbf{Chapters 10.1 - 10.4}
\2 Tuesday warm up: chapter 10.1
\2 Thursday warm up: chapter 10.4
\2 Topics covered:
\3 Angular acceleration and rotational kinematics.
\3 Inertia and rotational kinetic energy.
\1 Unit 10 - \textbf{Chapters 10.5 - 10.6}
\2 Tuesday warm up: chapter 10.5
\2 Thursday warm up: chapter 10.6
\2 Topics covered:
\3 Angular momentum conservation.
\3 Collisions with rotations.
\1 Unit 11 - \textbf{Class presentations and Final Review}
\2 No warm-up lectures
\2 Group presentations:
\3 Worth 25\% of the course grade.
\3 Given as a group.
\3 15 minute presentation with slides or digital storytelling format.
\3 \textbf{Option A:} standard presentation with PDF, LibreOffice, or PowerPoint presentation slides. Presenters speak in person demonstrating their results, practicing technical communication.
\3 \textbf{Option B:} creation of a pre-recorded video that uses tools from digital storytelling.  Whittier College uses WeVideo for such projects: \url{https://www.wevideo.com/}.  The video illustrates the scientific results such that the audience can understand them.
\end{outline}

\end{document}
