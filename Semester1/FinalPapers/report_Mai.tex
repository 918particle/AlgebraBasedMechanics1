\title{History of Science Paper: Brandon Mai}
\author{Dr. Jordan Hanson - Whittier College Dept. of Physics and Astronomy}
\date{\today}
\documentclass[10pt]{article}
\usepackage[a4paper, total={18cm, 27cm}]{geometry}
\usepackage{outlines}
\usepackage[sfdefault]{FiraSans}
\usepackage{hyperref}

\begin{document}
\maketitle

\begin{abstract}
The paper was a recaounting of history of the lives of three people: Tycho Brache, Johannes Kepler, and Sir Isaac Newton.  Several scientific details about Brache were included, and also important moments in his career.  The connection between Brache and Kepler was evident, however I would have liked to see more detail about the connection between Isaac Newton and Kepler.  There were interesting historical details included; the Thirty Years' War serves as historical backdrop, and it is an interesting challenge the scientists had to face.  I appreciated the inclusion of personal details about the character of the scientists, however, I would like to know more about their inventions and how they worked.  Newton's prism experiment and gravity calculations, or Kepler's discovery of the Law of Harmonies are interesting scientifically and mathematically.
\end{abstract}

\textit{Assessment} -  Replaces lowest midterm score with a 100. \\ \\

\textit{Notes on Paper}
\begin{outline}[enumerate]
\1 Astrology is trying to predict someone's future from constellations, \textbf{astronomy} is the science of the stars.
\1 Tycho Brache was Danish, interesting, I did not know that.
\1 Brache was kidnapped: many famous scientists and other thinkers turn out to have winding paths towards success.  This does not surprise me that he had this to motivate him.
\1 I laughed out loud when I read that he had his nose shot off
\1 ``The Kind of Germany gave him the opportunity to build an observatory to study astronomy.  King Frederick II granted him the small island of Hven near Copenhagen, where he funded all the expenses...the best observatory ever in Europe.''  \textbf{This shows me that the relationship between new science and the interests of the powerful has persisted throughout the centuries.  What is interesting is that today, the powerful happen to be less interested in science than in previous decades, but more interested in economics and business.}  This is an interesting detail to include.
\2 There are a few remarks about the inventions of Brache for observing stars.  I would have liked to see more detail about how they worked since they pre-date the spyglass.
\2 Was Brache a Protestant?  The emperor didn't seem to mind!  He appointed him Imperial Astronomer, in what remained of the Holy Roman Empire.
\2 Then Brache observed a supernova...it seemed like there were philosophical issues at stake here, but it wasn't clear what they were other than that it seemed to challenge Aristotelean views for some reason.  I appreciated the inclusion of the key scientific observation: Brache saw no parallax in the supernova, after observing for a year.  Do you know how parallax works?
\1 Regarding Kepler's relationship with his parents: this is actually not uncommon among famous scientists.  They have issues with either family or the place where they were born, and this drives them to seek answers.  In the search for answers, they end up doing wonderful science.  The reconciliation with their origin is always interesting.
\2 I appreciated the inclusion of details about the advisor-student relationship between Brache and Kepler.  I'm not sure why Brache left his Imperial Post, but thankfully it led him to Kepler.
\1 Was \textit{Tabulae Rudolphinae} named for the emperor?
\1 How did Kepler tease out the law of harmonies?  In your reading did you encounter \textbf{the eureka moment?}
\1 Sir Isaac Newton was not connected to Kepler in the paper.  I learned that he obtained his first degree six decades after Kepler died, so perhaps they are not.
\2 But do you understand how Isaac Newton built off of the work of people that came before him, rather than coming up with it out of thin air?  This is a common misconception about ``great big people.''
\2 There was an interesting remark that Isaac Newton's studies of light inspired artists.  This is a classic hallmark of wonderful science, that it causes new art to be created.
\end{outline}
\end{document}
