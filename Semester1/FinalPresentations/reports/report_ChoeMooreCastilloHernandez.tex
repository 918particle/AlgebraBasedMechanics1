\title{Project Overview and Assessment: Elizabeth Castillo, Esther Choe, Megan Hernandez, and Jake Moore}
\author{Dr. Jordan Hanson - Whittier College Dept. of Physics and Astronomy}
\date{\today}
\documentclass[10pt]{article}
\usepackage[a4paper, total={18cm, 27cm}]{geometry}
\usepackage{outlines}
\usepackage[sfdefault]{FiraSans}
\usepackage{hyperref}

\begin{document}
\maketitle

\begin{abstract}
The goal of this project was to determine the kinematic conditions that lead to the fastest overall run time for a distance of 20 meters.  There is a connection between physics, and biomechanics and kinesiology.  That connection was explored in this project by recording kinematic data of a runner in the same fashion as it would be recorded in a biomechanical context.  While the hypothesis needs to be formulated more carefully and quantitatively, the data is sound.  The faster run times come from runs during which a higher constant velocity was sustained for longer periods by the runner.
\end{abstract}

\textit{Score} - \textbf{9 of 10 points.}  Formulate hypotheses quantitatively rather than qualitatively, and quantify words like \textit{steady velocity} with the slope of the data or the fluctuation in the data about an average velocity.

\textit{Project Assessment}
\begin{outline}[enumerate]
\1 Introduction of Concepts, Hypothesis
\2 ``\textit{We believe that the best running time will have the fastest peak acceleration and a constant velocity throughout the rest of the run}.''  This is a clear but qualitative hypothesis.
\1 Explanation of the Experiment, with Diagram or Picture
\2 The explanation of the setup with words, diagrams, and pictures was clear.
\2 Might have been possible to record number of steps at each interval, but ultimately step-number was not used.
\1 Presentation of Data and Systematics
\2 The data was presented in a clear fashion.
\2 The overall experiment would have benefited from more bins, that is, a longer run with more intervals.  Alternatively, the experiment could have benefitted from combining the data from different runs to assess the effect of initial acceleration on the run time.
\1 Conclusion
\2 Although there was no quantitative prediction, there was a qualitative prediction that the fastest runs would have the steadiest velocities and the highest peak accelerations.  The data bore this out, with the fastest time coming from the highest peak acceleration, and the second fastest coming from the longest sustained constant velocity.
\end{outline}
\end{document}
