\title{Project Overview and Assessment: Brandon Mai, and Bryanna Wertz}
\author{Dr. Jordan Hanson - Whittier College Dept. of Physics and Astronomy}
\date{\today}
\documentclass[10pt]{article}
\usepackage[a4paper, total={18cm, 27cm}]{geometry}
\usepackage{outlines}
\usepackage[sfdefault]{FiraSans}
\usepackage{hyperref}

\begin{document}
\maketitle

\begin{abstract}
The goal of this project was to accurately predict the time required for a pumpkin to descend from a given height with an acceleration $g = 9.8$ m/s$^2$.  There were two trials, each having a different systematic error.  This could be due to an offset in stopwatch time.  The main improvement to this presentation recommended is the use of tables.  When there are several types of numbers floating around (predictions, measurements, different units, errors) it is helpful to organize them into tables.
\end{abstract}

\textit{Score} - \textbf{9 of 10 points.}

\textit{Project Assessment}
\begin{outline}[enumerate]
\1 Introduction of Concepts, Hypothesis
\2 There were quantitative predictions for times based on kinematics equations.
\1 Explanation of the Experiment, with Diagram or Picture
\2 The explanation of the setup with words, diagrams, and pictures was clear.
\1 Presentation of Data and Systematics
\2 See abstract.  The data could be better organized, however, it is clear that the data matched the hypothesis for the lower story.  For the higher story, the agreement was not as good, but this could be due to the real height of the building not being exactly 11.37 meters.
\1 Conclusion
\2 The conclusion was both quantitative and honest in assessment of errors.
\end{outline}
\end{document}
