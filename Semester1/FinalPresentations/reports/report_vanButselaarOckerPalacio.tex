\title{Project Overview and Assessment: Elmer van Butselaar, Sigi Ocker, and Chris Palacio}
\author{Dr. Jordan Hanson - Whittier College Dept. of Physics and Astronomy}
\date{\today}
\documentclass[10pt]{article}
\usepackage[a4paper, total={18cm, 27cm}]{geometry}
\usepackage{outlines}
\usepackage[sfdefault]{FiraSans}
\usepackage{hyperref}

\begin{document}
\maketitle

\begin{abstract}
This was a highly organized, detailed, and carefully planned experiment that didn't work.  The measured k-value was discovered to depend on the mass hung from it, indicating several regimes of force per unit length.  Intriguingly, another group found that nylon is actually more linear and predictable.  Regardless of the difficulties encountered, I felt that I learned something valuable here, and there is no question that this was a complete scientific investigation that relied upon both quantitative predictions and measurements.  The lesson for the students here is that sometimes even the most detailed experiments do not work out, because we simply don't know how an object or system will behave until we test it, and thus this was a useful exercise.
\end{abstract}

\textit{Score} - \textbf{10 of 10 points.}

\textit{Project Assessment}
\begin{outline}[enumerate]
\1 Introduction of Concepts, Hypothesis
\2 Both the concepts and hypothesis were introduced quantitatively, and given subsequent explanation.
\1 Explanation of the Experiment, with Diagram or Picture
\2 The diagrams and pictures clearly depicted the experimental apparatus.
\1 Presentation of Data and Systematics
\2 The presentation of data and systematic errors was (and had to be) thorough, due to an unforseen challenge.  However, I understood completely what subsequent measurements were made, and what the data indicated in the end.  The release clamp was a nice detail, in that we could not attribute systematic errors to fingers or release inconsistencies.
\1 Conclusion
\2 The hypothesis was not confirmed, but it it seemed evident that the reason was the k-value was not reliable.
\end{outline}
\end{document}
