\documentclass{article}
\usepackage{graphicx}
\usepackage[margin=1.5cm]{geometry}
\usepackage{amsmath}

\begin{document}

\title{Warm-Up: Units 5 and 6, Kepler's Law, Orbits, Work, and Energy}
\author{Prof. Jordan C. Hanson}

\maketitle

\section{Memory Bank}

\begin{itemize}
\item $T^2 \propto r^3$ ... Kepler's Third Law
\item Given two planets, we can use this like:
\begin{equation}
\left(\frac{T_2}{T_1}\right)^2 = \left(\frac{r_2}{r_1}\right)^3 \label{eq:orb}
\end{equation}
\item Let $\vec{u} = u_x \hat{i} + u_y \hat{j}$, and $\vec{v} = v_x \hat{i} + v_y \hat{j}$, and $\theta$ be the angle between them.  The \textit{dot-product} of $\vec{u}$ and $\vec{v}$ is
\begin{equation}
\vec{u} \cdot \vec{v} = u_x v_x + u_y v_y = |\vec{u}||\vec{v}|\cos\theta
\end{equation}
\item Definition of the work, $W$, done by a net external force $\vec{F}$ through a displacement $\Delta\vec{x}$ is:
\begin{equation}
W = \vec{F} \cdot \Delta \vec{x} = F \Delta x \cos\theta
\end{equation}
\end{itemize}

\section{Kepler's Laws}

\begin{enumerate}
\item Suppose we define a unit called an ``Astronomical Unit'' that is equal to $1.496\times 10^8$ km.  This is the distance between the Earth and the Sun.  So we can say that the Earth is 1 AU from the Sun.  It turns out that Venus is 0.72 AU from the Sun (it's closer).  The orbit of the Earth is 1 year.  Let $T_1 = 1$ year, $r_1 = 1.0$ AU for the Earth, and $r_2 = 0.72$ AU for Venus.  Use Eq. \ref{eq:orb}. to find the orbital period of Venus, $T_2$. \\ \vspace{1cm}
\item The orbital period of Jupiter is observed to be 11.8 years.  How far in AU is Jupiter from the Sun?  (\textit{Hint: it's the same procedure as the prior problem using Earth's numbers, but solving for $T_2$}). \\ \vspace{1cm}
\end{enumerate}

\section{Work and Energy}

\begin{enumerate}
\item Suppose a person pushes a box with a force $\vec{F} = 40 \hat{i} - 30 \hat{j}$ N of force, and the box moves $\Delta \vec{x} = 5\hat{i}$ m.  (a) What is the work done?\footnote{Note the units are the product of force and displacement, so Newton meters.  One Newton meter is called a Joule.} (b) Draw a diagram of the crate, force vector, and displacement vector. (c) What is the angle between $\vec{F}$ and $\Delta \vec{x}$? (d) What can the person do to increase the work done?
\end{enumerate}

\end{document}
