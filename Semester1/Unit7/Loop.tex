\documentclass{article}
\usepackage{graphicx}
\usepackage[margin=1.5cm]{geometry}
\usepackage{amsmath}

\begin{document}

\title{Loop Activity: Conservation of Energy}
\author{Prof. Jordan C. Hanson}

\maketitle

\section{Memory Bank}

\begin{enumerate}
\item $KE = \frac{1}{2}m v^2$ ... Definition of kinetic energy
\item $PE_G = mgh$ ... Definition of gravitational potential energy
\item $KE_{rot} = \frac{1}{2}I \omega^2$ ... Definition of kinetic energy for system rotating with $\omega$.
\item $I = \frac{2}{5}mr^2$ ... \textit{Moment of inertia} for a solid sphere with radius $r$.
\item $a_C = mv^2/r$ ... Centripetal acceleration.
\end{enumerate}

\section{Fixing the Loop the Loop Experiment}

\begin{enumerate}
\item Start with energy conservation, and assume $y$ is the initial marble height and that $R$ is the loop radius.  (a) If the initial marble height is $y$, and it has mass $m$, what is the initial $PE_G$ of the system? (b) Write the general expression for the $KE$ of the marble with speed $v$, and add to it the $PE_G$ the marble has when it is at the top of the loop (height of $2R$). (c) Set the results from (a) and (b) equal to each other.  This is \textbf{energy conservation}. \\ \vspace{1.5cm}
\item Note that the marble is moving in a circle, meaning it has centripetal acceleration $a_C$.  (a) Show that the normal force on the marble at the top of the loop is $N + mg = mv^2/R$. (b) For $N = 0$, derive an expression for $v$, representing the minimum velocity necessary to proceed through the loop. \\ \vspace{1.5cm}
\item Substitute $v$ into the equation for \textbf{energy conservation} from (1), and solve for $y/R$ (the result should be a unitless fraction). \\ \vspace{1.5cm}
\end{enumerate}

\section{Bonus Round: Adding angular kinetic energy}

\begin{enumerate}
\item Repeat the entire process, but adding $KE_{rot}$ from the memory bank to the kinetic energy.  How does this affect the result for $y/R$?
\end{enumerate}

\end{document}