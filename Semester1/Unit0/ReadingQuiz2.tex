\documentclass{article}
\usepackage{graphicx}
\usepackage[margin=1.5cm]{geometry}
\usepackage{amsmath}

\begin{document}
\twocolumn

\title{Wednesday warm-up: unit analysis and vectors}
\author{Prof. Jordan C. Hanson}

\maketitle

\section{Chapter 1 - Unit analysis, \\ Estimation}

\begin{enumerate}
\item What is 25 m s$^{-1}$ in km hr$^{-1}$?
\begin{itemize}
\item A: 15 km hr$^{-1}$
\item B: 25 km hr$^{-1}$
\item C: 60 km hr$^{-1}$
\item D: 90 km hr$^{-1}$
\end{itemize}
\item Suppose a ship accelerates from 0 km hr$^{-1}$ to 10 km hr$^{-1}$ in 60 seconds.  If acceleration is the change in velocity divided by the change in time, what is the acceleration of the ship? \\ \vspace{1cm}
\item Estimate the area of the North Quad of Whittier College (the open space outside the SLC):
\begin{itemize}
\item A: 5000 m$^2$
\item B: 5000 cm$^2$
\item C: 500 m$^2$
\item D: 500 cm$^2$
\end{itemize}
\item A coffee bean is about 0.5 cm$^3$ in volume.  How many could fit in a 2 liter bottle?
\begin{itemize}
\item A: $4\times 10^{1}$
\item B: $4\times 10^{2}$
\item C: $4\times 10^{3}$
\item D: $4\times 10^{4}$
\end{itemize}
\end{enumerate}

\vspace{5cm}

\section{Chapter 2 - Vectors}

\begin{enumerate}
\item Recently, we have represented 2D vectors like this: $\vec{v} = (v_x,v_y)$.  The $v_x$ is the x-component, and the $v_y$ is the y-component.  Let us exchange this notation for a different one.  Let $\vec{v} = v_x \hat{i} + v_y \hat{j}$.  The $\hat{i}$ and the $\hat{j}$ are \textbf{unit vectors}, each with length 1.  The $\hat{i}$ points in the x-direction, and the $\hat{j}$ points in the y-direction. (a) Let $\vec{v} = -2\hat{i} + 2\hat{j}$, and $\vec{w} = 2\hat{i} - 2\hat{j}$.  Draw each in a 2D coordinate system below. (b) What is $\vec{v} + \vec{w}$?  (c) What is $\vec{v} - \vec{w}$? \\ \vspace{3cm}

\end{enumerate}

\end{document}
