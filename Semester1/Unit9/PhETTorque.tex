\documentclass{article}
\usepackage{graphicx}
\usepackage[margin=1.5cm]{geometry}
\usepackage{amsmath}
\usepackage{url}

\begin{document}

\title{PhET Activity: Torque}
\author{Prof. Jordan C. Hanson}

\maketitle

\section{Introduction}

\begin{enumerate}
\item Navigate to \url{https://phet.colorado.edu/en/simulations/balancing-act}.
\item For rotating an object, not only force but distance of the force from rotation point also matters. For rotational equilibrium, total torque about rotation point must be zero. This means, about rotation point, total left-side torque should be equal to total right-side torque.
\item Recall that torque is $\tau = r F\sin\theta$, where $F$ is the force, $r$ is the level arm, and $\theta$ is the angle between them.
\end{enumerate}

\section{Explore}

\begin{enumerate}
\item Remove the support rods. Does the rod rotate? Can you identify rotation point?
\item Keep one fire extinguisher on rod, at a point. Does the rod rotate? If yes, why? Notice direction of rotation (Clockwise/Anti-clockwise) also.
\item Keep it at different positions and observe change in rotation speed of the rod.
\item Place another fire extinguisher on other side. Try to balance rod. Notice positions of fire extinguishers when rod becomes horizontal.  Use the rulers to check the positions.
\item Calculate the left-side and right-side torque. What is the relation between values of torques when rod is stable?
\item Adjust one of the extinguishers a little from its position. What happens to movement of the rod? Explain reason to support your observation.
\end{enumerate}

\section{Balance Lab}

\begin{enumerate}
\item Place mystery object A at 1 m position on rod. Balance rod by using 20 kg boy. How can you
find mass of mystery object A? Have you ever thought of how does seesaw scale at grocery
shop works?
\end{enumerate}

\section{Game Time}

Play the game tab to test your understanding of force and torque!

\end{document}