\documentclass{article}
\usepackage{graphicx}
\usepackage[margin=1.5cm]{geometry}
\usepackage{amsmath}

\begin{document}

\title{Forces and Inclines}
\author{Prof. Jordan C. Hanson}

\maketitle

\section{Review of Spring Force}
The spring force is $\vec{s} = -k\Delta \vec{x}$.  That is, the force is directly proportional to the \textit{change} in length.  Use the given weights, spring, ruler and hook to measure the spring constant $k$ for your spring.  That is, after drawing an appropriate free-body diagram, we find that 
\begin{equation}
|mg| = k|\Delta x|
\end{equation}
Solving for $k$:
\begin{equation}
k = \frac{mg}{\Delta x}
\end{equation}
The constant $k$ has units of Newtons per meter.  Enter your data and corresponding value for $k$ below: \\ \vspace{2cm}
\section{Inclined Surfaces}
Now place your weights on the ruler and align the spring with the ruler such that the ruler serves as an \textit{inclined plane} for the weights.  Use enough weight so that you can observe the increase in length of the spring, even when the spring is nearly horizontal.  Draw a free-body diagram for the weight attached to the spring: \\ \vspace{2cm}
\section{Net Force on Weight}
The ruler should be providing a normal force $N = mg \cos\theta$ to keep the weights on the ruler.  The net force \textit{down the ruler} should be $F_{net} = mg\sin\theta$, where $\theta$ is the angle between the ruler and the table.  Use the protractor to measure the angle between the ruler and the table.  Create a graph below of the $\Delta x$ (change in length) of the spring versus $\theta$.  Does it follow the expected $\sin\theta$ dependence?
\end{document}
