\documentclass{article}
\usepackage{graphicx}
\usepackage[margin=1.5cm]{geometry}
\usepackage{amsmath}

\begin{document}

\title{Warm Up Exercises: Circular Motion}
\author{Prof. Jordan C. Hanson}

\maketitle

\section{Memory Bank}

\begin{itemize}
\item Let $\Delta\theta$ be the \textit{angular displacement}, $\Delta\theta = \theta_f - \theta_i$.  Let the time duration be $\Delta t = t_f - t_i$.  Let the angular velocity be $\omega = \Delta \theta / \Delta t$.  If $t_i = 0$ seconds and $\theta_i = 0$ degrees, then we can use $omega$ to write $\theta = \omega t$ (just like $x = v t$.  If an object is rotating with angular velocity $\omega$ on a circle of radius $r$, then the position versus time is:
\begin{equation}
\vec{r}(t) = r\cos(\omega t)\hat{i} + r\sin(\omega t)\hat{j} \label{eq:1}
\end{equation}
\item $v = r\omega$ ... Radial velocity.
\item $a_{\rm C} = v^2/r$ ... Centripetal acceleration.
\item $a_{\rm C} = r \omega^2$ ... Centripetal acceleration.
\item $\vec{F}_{\rm C} = m a_{\rm C}$ ... Centripetal force.
\end{itemize}



\section{Centripetal Force}
\begin{enumerate}
\item Suppose a system is rotating about the origin with a radius $r = 1.0$ m, and angular speed $\omega = 50$ rotations per second. (a) What is the angular speed in radians per second?  (b) Where is the system at $t = 0.75$ seconds?  (c) What are the radial velocity and centripetal acceleration? (d) If a mass of 0.05 kg is attached to the end of the radius, what is $\vec{F}_{\rm C}$? \\ \vspace{3cm}
\item In the prior problem, what would $\vec{F}_{\rm C}$ be if the angular velocity was doubled?
\end{enumerate}

\end{document}
