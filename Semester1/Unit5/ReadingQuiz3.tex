\documentclass{article}
\usepackage{graphicx}
\usepackage[margin=1.5cm]{geometry}
\usepackage{amsmath}

\begin{document}

\title{Warm Up Exercises: Drag, Circular Motion}
\author{Prof. Jordan C. Hanson}

\maketitle

\section{Memory Bank}

\begin{itemize}
\item Force of drag, in air or other gas: $F_D = \frac{1}{2}C \rho A v^2$.
\item In the above formula, $C$ is an empirical constant, $\rho$ is the density of the air or gas, $A$ is the area of the object, and $v$ is the object's velocity.
\item Let $\Delta\theta$ be the \textit{angular displacement}, $\Delta\theta = \theta_f - \theta_i$.  Let the time duration be $\Delta t = t_f - t_i$.  Let the angular velocity be $\omega = \Delta \theta / \Delta t$.  If $t_i = 0$ seconds and $\theta_i = 0$ degrees, then we can use $omega$ to write $\theta = \omega t$ (just like $x = v t$.  If an object is rotating with angular velocity $\omega$ on a circle of radius $r$, then the position versus time is:
\begin{equation}
\vec{r}(t) = r\cos(\omega t)\hat{i} + r\sin(\omega t)\hat{j} \label{eq:1}
\end{equation}
\item $a_{\rm C} = r \omega^2$ ... Centripetal force.
\item $v = r\omega$ ... Radial velocity.
\end{itemize}



\section{Drag Forces, Circular Motion}
\begin{enumerate}
\item Suppose a cyclist with $A = 0.5$ m$^2$, $C = 1.0$, and total mass $m = 70$ kg is pedalling at $20$ m/s.  Assume the density of air is $\rho = 1.2$ kg m$^{-3}$.  (a) What is the drag force on the system? (b) What is the drag force if the speed drops to $10$ m/s? (c)  Suppose the speed is now $20$ m/s again, but the cyclist ducks down to lower the area to $A = 0.25$ m$^2$.  What is the new drag force? \\ \vspace{1.5cm}
\item Prove Eq. \ref{eq:1} using trigonometry.  \textit{It is helpful to draw a graph and recall that $\theta = \omega t$.} \\ \vspace{2cm}
\item Suppose a system is rotating about the origin with a radius $r = 1.0$ m, and angular speed $\omega = 2\pi/10$ radians per second. (a) Where is the system at $t = 0$ seconds?  (b) Where is the system at $t=5$ seconds? (c) What are the radial velocity and centripetal acceleration? \\ \vspace{2cm}
\item Find the time $t$ that makes the position $\vec{r} = -1.0\hat{j}$ m.
\end{enumerate}

\end{document}
